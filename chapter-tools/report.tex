\documentclass[11pt,letterpaper]{article}

\addtolength{\oddsidemargin}{-.875in}
\addtolength{\evensidemargin}{-.875in}
\addtolength{\textwidth}{1.75in}

\addtolength{\topmargin}{-.875in}
\addtolength{\textheight}{1.75in}

\usepackage[utf8]{inputenc}
\usepackage{caption} % for table captions
\usepackage{amsmath} % for multi-line equations and piecewises
\DeclareMathOperator{\sign}{sign}
\usepackage{graphicx}
\usepackage{relsize}
\usepackage{xspace}
\usepackage{verbatim} % for block comments
\usepackage{subcaption} % for subfigures
\usepackage{enumitem} % for a) b) c) lists
\newcommand{\Cyclus}{\textsc{Cyclus}\xspace}%
\newcommand{\Cycamore}{\textsc{Cycamore}\xspace}%
\newcommand{\deploy}{\texttt{d3ploy}\xspace}%
\newcommand{\Deploy}{\texttt{D3ploy}\xspace}%
\usepackage{tabularx}
\usepackage{color}
\usepackage{multirow}
\usepackage{float}
\usepackage[acronym,toc]{glossaries}
\newacronym{BSD}{BSD}{Berkeley Software Distribution}
\newacronym{CR}{CR}{control rod}
\newacronym{EOEC}{EOEC}{end of equilibrium cycle}
\newacronym{FDM}{FDM}{Finite Difference Method}
\newacronym{FEM}{FEM}{Finite Element Method}
\newacronym{GRS}{GRS}{Gesellschaft für Anlagen und Reaktorsicherheit}
\newacronym{He}{He}{helium}
\newacronym{HZDR}{HZDR}{Helmholtz-Zentrum Dresden-Rossendorf}
\newacronym{HTGR}{HTGR}{High Temperature Gas-Cooled Reactor}
\newacronym{HTR}{HTR}{High Temperature Reactor}
\newacronym{HTTR}{HTTR}{High Temperature Test Reactor}
\newacronym{IPyC}{IPyC}{inner pyrolitic carbon}
\newacronym{INL}{INL}{Idaho National Laboratory}
\newacronym{JFNK}{JFNK}{Jacobian-Free Newton-Krylov}
\newacronym{KAERI}{KAERI}{Korea Atomic Energy Research Institute}
\newacronym{LGPL}{LGPL}{Lesser GNU Public License}
\newacronym{LWR}{LWR}{Light Water Reactor}
\newacronym{MC}{MC}{Monte Carlo}
\newacronym{MHTGR}{MHTGR}{Modular High-Temperature Gas-Cooled Reactor}
\newacronym{MOOSE}{MOOSE}{Multiphysics Object-Oriented Simulation Environment}
\newacronym{MSR}{MSR}{Molten Salt Reactor}
\newacronym{NEA}{NEA}{Nuclear Energy Agency}
\newacronym{NEM}{NEM}{Nodal Expansion Method}
\newacronym{NRC}{NRC}{Nuclear Regulatory Commission}
\newacronym{NSC}{NSC}{Nuclear Science Committee}
\newacronym{OECD}{OECD}{Organisation for Economic Co-operation and Development}
\newacronym{OPyC}{OPyC}{outer pyrolitic carbon}
\newacronym{PBMR}{PBMR}{Pebble Bed Modular Reactor}
\newacronym{PDE}{PDE}{partial differential equation}
\newacronym{PMR}{PMR}{Prismatic Modular Reactor}
\newacronym{RSD}{RSD}{Relative Standard Deviation}
\newacronym{SD}{SD}{Standard Deviation}
\newacronym{SiC}{SiC}{silicon carbide}
\newacronym{SNU}{SNU}{Seoul National University}
\newacronym{TRISO}{TRISO}{Tristructural Isotropic}
\newacronym{UIUC}{UIUC}{University of Illinois at Urbana-Champaign}
\newacronym{UNIST}{UNIST}{Ulsan National Intitute of Science and Technology}
\newacronym{UK}{UK}{United Kingdom}
\newacronym{UMICH}{UMICH}{Universtiy of Michigan}
\newacronym{US}{US}{United States}
\newacronym{VHTR}{VHTR}{very high temperature reactor}
%\newacronym{<++>}{<++>}{<++>}
%\newacronym{<++>}{<++>}{<++>}

\definecolor{bg}{rgb}{0.95,0.95,0.95}
\newcolumntype{b}{X}
\newcolumntype{f}{>{\hsize=.15\hsize}X}
\newcolumntype{s}{>{\hsize=.5\hsize}X}
\newcolumntype{m}{>{\hsize=.75\hsize}X}
\newcolumntype{r}{>{\hsize=1.1\hsize}X}
\usepackage{titling}
\usepackage[hang,flushmargin]{footmisc}
\renewcommand*\footnoterule{}
\usepackage{tikz}

\usetikzlibrary{shapes.geometric,arrows}
\tikzstyle{process} = [rectangle, rounded corners,
minimum width=1cm, minimum height=1cm,text centered, draw=black,
fill=blue!30]
\tikzstyle{arrow} = [thick,->,>=stealth]

\graphicspath{}

\begin{document}

\section{MOOSE}

% intro
\gls{MOOSE}\cite{gaston_moose_2009} is a computational framework whose purpose is to support the development of engineering analysis applications.
In a nuclear reactor, several \glspl{PDE} describe its physical behavior.
These equations are typically nonlinear and they are often strongly coupled to each other.
\gls{MOOSE} targets such systems and solves them in a fully coupled manner.

% more details about MOOSE
\gls{MOOSE} is an open source code under a \gls{LGPL}.
The code itself relies on LibMesh \cite{kirk_libmesh_2006}, an LGPL finite element library, and PetSc, a \gls{BSD}-licensed toolkit for solving nonlinear equations \cite{balay_petsc_2016}.
MOOSE applications define weak forms of the governing equations.
They modularize the physics expressions into "Kernels".
Kernels are C++ classes that contain methods for computing the residual and Jacobian contributions of individual pieces of the governing equations.
\gls{MOOSE} and LibMesh translate them into residual and Jacobian functions.
These functions become inputs into PetSc solution routines.

\gls{MOOSE} utilizes the mathematical structure present in \gls{JFNK} methods \cite{knoll_jacobian-free_2004}.
\gls{JFNK} methods are synergistic combinations of Newton-type methods for superlinearly convergent solution of nonlinear equations and Krylov subspace methods for solving the Newton correction equations.
The link between the two methods is the Jacobian-vector product.
\gls{JFNK} methods compute such product approximately without forming and storing the elements of the true Jacobian.
The ability to perform a Newton iteration without forming the Jacobian gives \gls{JFNK} methods potential for application throughout problems governed by nonlinear partial differential equations.

All the software built on the \gls{MOOSE} framework shares a common code base.
The applications by default use monolithic and implicit methods.
This feature facilitates relatively easy coupling between different phenomena and allows for great flexibility even with large variance in time scales \cite{novak_pronghorn_2018}.
Additionally, all codes use MPI for parallel communication and allow for deployment on massively-parallel cluster-computing platforms.

\section{Moltres}

Moltres \cite{lindsay_introduction_2018} is a \gls{MOOSE} based application originally designed for modeling fluid-fuelled \glspl{MSR}.
This simulation tool is open source and counts with an LGPL license.
It uses \texttt{git} for version control, emphasizing its openness and promoting quality through peer review.

Moltres solves arbitrary-group neutron diffusion, precursors, and temperature governing equations.
It can solve the equations in a fully-coupled way or solve each system independently allowing for great flexibility and making it applicable to a wide range of nuclear engineering problems.

\subsection{Diffusion and precursors equations}

% Equations
Equations \ref{eq:diffusion} and \ref{eq:precursors} describe the time dependent behavior of the neutronics.

\begin{align}
  \frac{1}{v_g}\frac{\partial}{\partial t} \phi_g &= \nabla \cdot D_g \nabla \phi_g -
  \Sigma_g^r \phi_g + \sum_{g \ne g'}^G \Sigma_{g'\rightarrow g}^s \phi_{g'} +
  \chi_g^p \sum_{g' = 1}^G (1 - \beta) \nu \Sigma_{g'}^f \phi_{g'} +
  \chi_g^d \sum_i^I \lambda_i C_i
\label{eq:diffusion}
        \intertext{where}
        v_g &= \mbox{group $g$ neutron speed} \notag \\
        \phi_g &= \mbox{group $g$ neutron flux} \notag \\
        t &= \mbox{time} \notag \\
        D_g &= \mbox{group $g$ diffusion coefficient} \notag \\
        \Sigma_g^r &= \mbox{group $g$ macroscopic removal cross-section} \notag \\
        \Sigma_{g'\rightarrow g}^s &= \mbox{group $g'$ to group $g$ macroscopic scattering} \notag \\
        &\phantom{{}=1} \mbox{cross-section} \notag \\
        \chi_g^p &= \mbox{group $g$ prompt fission spectrum} \notag\\
        G &= \mbox{number of discrete energy groups} \notag \\
        \nu &= \mbox{number of neutrons produced per fission} \notag \\
        \Sigma_g^f &= \mbox{group $g$ macroscopic fission cross-section} \notag \\
        \chi_g^d &= \mbox{group $g$ delayed fission spectrum} \notag \\
        I &= \mbox{number of delayed neutron precursor groups} \notag \\
        \beta &= \mbox{delayed neutron fraction} \notag \\
        \lambda_i &= \mbox{average decay constant of delayed neutron} \notag \\
        &\phantom{{}=1} \mbox{precursors in precursor group $i$} \notag \\
        C_i &= \mbox{concentration of delayed neutron precursors} \notag \\
        &\phantom{{}=1} \mbox{in precursor group $i$}. \notag
\end{align}

\begin{align}
        \frac{\partial}{\partial t} C_i &= \sum_{g'= 1}^G \beta_i \nu
        \Sigma_{g'}^f \phi_{g'} - \lambda_i C_i
\label{eq:precursors}
\end{align}

For the eigenvalue calculations, the equations lose the time dependency.
Using \ref{eq:chit} we combine both equations.
We introduce the parameter $k_{eff}$ that divides the fission term obtaining equation \ref{eq:eigenvalue}.

\begin{align}
    \chi_g^t &= \chi_g^p (1 - \beta) + \chi_g^d \sum_i^I \beta_i  \label{eq:chit} \\
  \nabla \cdot D_g \nabla \phi_g - \Sigma_g^r \phi_g &+
  \sum_{g \ne g'}^G \Sigma_{g'\rightarrow g}^s \phi_{g'} + \chi_g^t \sum_{g' = 1}^G \frac{1}{k_{eff}}\nu \Sigma_{g'}^f \phi_{g'} = 0
  \label{eq:eigenvalue}
\end{align}

% Boundary conditions
We apply the vacuum boundary condition to the diffusion equation for both transient and eigenvalue problems.
The vacuum boundary condition states that no neutrons penetrate the boundary in the inward direction.
In other words, the incoming current density ($J^-(r_s, t)$) is equal to zero, equation \ref{eq:vacuumbc1}.

\begin{align}
   J^-(r_s, t) = \frac{1}{4} \phi(r_s, t) + \frac{D}{2} \hat{n_s} \cdot \nabla \phi (r_s, t) = 0
\label{eq:vacuumbc1}
\end{align}

% Precursor BC?
% The precursors equation is an IVP, so it doesn't require the definition of BC
% When accounting for the precursors drift, that's a different story.

\subsection{Thermal-Hydraulics}

The governing equation for the temperature of the solids is the 3-D heat conduction equation \cite{melese_thermal_1984}.
Equations \ref{eq:tempfuel} and \ref{eq:heatsource} allow for solving the temperature in the fuel.

\begin{align}
    \rho_f c_{p,f} \frac{\partial}{\partial t} T_f &= k_f \nabla^2 T_f + Q_f \label{eq:tempfuel} \\
    Q_f &= \sum_{g = 1}^{G} \epsilon_g^f \Sigma_g^f \phi_g \label{eq:heatsource}
  \intertext{where}
  \rho_f &= \mbox{fuel density} \notag \\
  c_{p,f} &= \mbox{fuel specific heat capacity} \notag \\
  k_f &= \mbox{fuel thermal conductivity } \notag \\
  T_f &= \mbox{fuel temperature} \notag \\
  Q_f &= \mbox{volumetric heat source} \notag \\
  \epsilon_g^f &= \mbox{energy released per fission} \notag \\
  \Sigma_g^f &= \mbox{group $g$ macroscopic fission cross-section} \notag \\
  \phi_g &= \mbox{group $g$ neutron flux.} \notag
\end{align}

Equations \ref{eq:tempgraphite} solve the temperature of the moderator and reflector.

\begin{align}
    \rho_i c_{p,i} \frac{\partial}{\partial t} T_i &= k_i \nabla^2 T_i \label{eq:tempgraphite}
  \intertext{where}
  i &= \mbox{m (moderator) or r (reflector)} \notag \\
  \rho_i &= \mbox{material $i$ density} \notag \\
  c_{p,i} &= \mbox{material $i$ specific heat capacity} \notag \\
  k_i &= \mbox{material $i$ thermal conductivity} \notag \\
  T_i &= \mbox{material $i$ temperature.} \notag
\end{align}

% coolant: need to do this section, several things missing
% The governing equation of the coolant is the 1-D form of the continuity, momentum, and energy conservation equations, equations \ref{eq:continuity} to \ref{eq:temperature} \cite{melese_thermal_1984}.
% \begin{align}
%  	\frac{\partial}{\partial t} \rho + \nabla \cdot (\rho u) &= 0 \label{eq:continuity} \\
%   \frac{\partial}{\partial t}(\rho u) + \frac{\partial}{\partial z}(\rho u^2) &= - \frac{\partial}{\partial z}p - \tau \frac{\epsilon}{A} - \rho g \label{eq:momentum} \\
%  	\rho_c c_{p,c} \left( \frac{\partial}{\partial t} T_c + u \frac{\partial}{\partial z} T_c \right) &= q'''_{conv} 	\label{eq:temperature}
%   \intertext{where}
%   \rho_c &= \mbox{coolant density} \notag \\
%   u &= \mbox{coolant velocity} \notag \\
%   p &= \mbox{coolant pressure} \notag \\
%   \tau &= \mbox{shear stress} \notag \\
%   \epsilon &= \mbox{wetted perimeter} \notag \\
%   A &= \mbox{cross-sectional area} \notag \\
%   g &= \mbox{gravity} \notag \\
%   c_{p,c} &= \mbox{coolant specific heat capacity} \notag \\
%   k_c &= \mbox{coolant thermal conductivity} \notag \\
%   T_c &= \mbox{coolant temperature} \notag \\
%   q'''_{conv} &= \mbox{convective heat transfer.} \notag
% \end{align}

% In the steady-state limit, the continuity equation becomes:
% \begin{align}
%  	\rho_c u & = \frac{\dot{m}}{A} = \mbox{constant} \label{eq:continuity2}
%   \intertext{where}
%   \dot{m} &= \mbox{mass flow rate} \notag
% \end{align}

% Introducing equation \ref{eq:continuity2} into \ref{eq:temperature} gives:
% \begin{align}
% 	c_{p,c} \frac{\dot{m}}{A} \frac{\partial}{\partial z} T_c = q'''_{conv}
% 	\label{eq:temperature2}
% \end{align}

% % momentum equation: correlations
% \begin{align}
%   \tau &= \frac{1}{2} f \rho_c u^2 \label{eq:friction} \\
%   f &=
%   \intertext{where}
%   f &= \mbox{friction factor} \notag
% \end{align}

% % momentum equation: incompressible flow \cite{tak_practical_2012}
% % This one neglects the fact that $u$ changes along $z$.
% \begin{align}
%   \Delta P &= \frac{1}{2} \rho u^2 \left(f\frac{L}{D} + K\right) + \rho g H
%   \intertext{where}
%   f &= \mbox{} \notag \\
% \end{align}

% % momentum equation: compressible flow \cite{melese_thermal_1984}
% \begin{align}
%   % \Delta P = P_i - P_o = \frac{\dot{m}^2}{2 \rho_i A^2} \left[ K_v &+ K_i + \frac{4fL}{D}\left( \frac{\bar{T}}{T_i} \right) + \frac{T_o-T_i}{T_i} + \sum_j K_j \left( \frac{T_j}{T_i} \right) + K_o \left( \frac{T_o}{T_i} \right) \right] \\
%   \Delta P = P_i - P_o = \frac{\dot{m}^2}{2 \rho_i A^2} & \left( K_v + K_i + \frac{4fL}{D} \frac{\bar{T}}{T_i} + \frac{T_o-T_i}{T_i} + \sum_j K_j \frac{T_j}{T_i} + K_o \frac{T_o}{T_i} \right) \\
%   \bar{T} &= (T_i+T_o)/2
%   \intertext{where}
%   K_v &= \mbox{loss coefficient due to flow control valve} \notag \\
%   T_j &= \mbox{coolant temperature at the bottom of block j} \notag \\
%   K_i &= \mbox{entrance pressure drop coefficient} \notag \\
%   K_j &= \mbox{offset loss coefficient at element interfaces} \notag \\
%   K_o &= \mbox{exit pressure drop coefficient} \notag
% \end{align}

% % convective heat:
% Dittus-Boelter equation

% \begin{align}
%   q''_{conv} &= h A (T_m - T_c) \label{eq:convheat} \\
%   h &= Nu \frac{k_c}{D_h} \\
%   Nu &= 0.023 Re^{0.8} Pr^{0.4} \label{eq:dittus}
% \end{align}

% How to obtain the constants and the values ?

\section{Serpent}



\section{Benchmark specification}



\subsection{Cross section handling}

\subsection{Homogenization}

\begin{align}
  \phi &= \frac{\sum_i \phi_i V_i}{\sum_i V_i}  \\
  \Sigma_x &= \frac{\sum_i \Sigma_{x,i} \phi_i V_i}{\sum_i \phi_i V_i}  \\
  \chi_t &= \frac{\sum_{t,i} \chi_i \nu \Sigma_{f,i} \phi_i V_i}{\sum_i \nu \Sigma_{f,i} \phi_i V_i}
  \intertext{where}
  \phi &= \mbox{neutron flux} \notag \\
  V_i &= \mbox{volume $i$} \notag \\
  \Sigma_x &= \{\Sigma_t, \Sigma_s^{g \rightarrow g'}, \nu\Sigma_f\} \notag
\end{align}

\subsection{Collapsing}

\begin{align}
  \phi_{G} &= \sum_g \phi_g \\
  \Sigma_x^G &= \frac{\sum_g \Sigma_x^g \phi_g}{\sum_g \phi_g} \\
  \Sigma_s^{G \rightarrow G'} &= \frac{\sum_g \sum_{g'} \Sigma_s^{g \rightarrow g'} \phi_g}{\sum_g \phi_g} \\
  \chi_t^G &= \sum_g \chi_t^g
  \intertext{where}
  \Sigma_x &= \{\Sigma_t, \nu\Sigma_f\} \notag \\
  \phi_{G} &= \mbox{group G neutron flux} \notag
\end{align}

\subsection{Translate to Moltres format}

% Info provided by the benchmark
Normalized flux, \Sigma_t, D, \nu\Sigma_f, \Sigma_f, \chi_T, P0 Scattering Matrix \Sigma_{s0, g \rightarrow g'}, P1 Scattering Matrix \Sigma_{s1, g \rightarrow g'}, \Kappa = 200 MeV

% Info required by Moltres
\Sigma_r, \nu\Sigma_f, D, \beta_{eff}, \Sigma_f, \Kappa, v^{-1}, \chi_T, \chi_D, P0 Scattering Matrix \Sigma_{s0, g \rightarrow g'}, \lambda

% Translation
\Sigma_r^g= \Sigma_t^g-\Sigma_s^{g \rightarrow g}

\section{Energy groups}

\begin{table}[htbp!]
  \centering
    \caption{Energy group structure \cite{gougar_prismatic_2010}.}
  \begin{tabular}{ll}
    \toprule
  Group   & Upper Energy (eV) \\ \midrule
  1   & 1.49E+07      \\
  2   & 7.41E+06      \\
  3   & 3.68E+06      \\
  4   & 6.72E+05      \\
  5   & 1.11E+05      \\
  6   & 1.93E+04      \\
  7   & 3.35E+03      \\
  8   & 1.58E+03      \\
  9   & 7.48E+02      \\
  10    & 2.75E+02      \\
  11    & 1.30E+02      \\
  12    & 6.14E+01      \\
  13    & 2.90E+01      \\
  14    & 1.37E+01      \\
  15    & 8.32E+00      \\
  16    & 5.04E+00      \\
  17    & 2.38E+00      \\
  18    & 1.29E+00      \\
  19    & 6.50E-01      \\
  20    & 3.50E-01      \\
  21    & 2.00E-01      \\
  22    & 1.20E-01      \\
  23    & 8.00E-02      \\
  24    & 5.00E-02      \\
  25    & 2.00E-02      \\
  26    & 1.00E-02      \\
    \bottomrule
  \end{tabular}
    \label{tab:energygroups}
\end{table}

\cline{1-2}

\pagebreak
\bibliographystyle{plain}
\bibliography{bibliography}

\end{document}

	% \begin{figure}[htbp!]
	% 	\centering
	% 	\begin{subfigure}[t]{0.4\textwidth}
	% 		\centering
	% 		\includegraphics[width=\linewidth]{figures/radial-layout.png}
	% 		\caption{XY-plane.}
	% 	\end{subfigure}
	% 	\begin{subfigure}[t]{0.4\textwidth}
	% 		\centering
	% 		\includegraphics[width=\linewidth]{figures/axial-layout.png}
	% 		\caption{YZ-plane.}
	% 	\end{subfigure}
	% 	\hfill
	% 	\caption{MHTGR reactor layout.}
	% 	\label{fig:layout}
	% \end{figure}

	% \begin{figure}[htbp!]
	% 	\centering
	% 	\includegraphics[width=0.6\linewidth]{figures/axial1.png}
	% 	\hfill
	% 	\caption{Neutron flux on the specified fuel channel.}
	% 	\label{fig:axial}
	% \end{figure}
