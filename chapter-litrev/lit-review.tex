Most important ones are:
% tak_numerical_2008


% tak_numerical_2008

The complex geometry of the hexagonal fuel assembly hinders accurate evaluations of the temperature profile  without elaborate numerical calculations.

The unit cell model is an effective approach that reduces the computational efforts.
However, such model neglects the effects of the bypass gap flow and the radial power distribution within the fuel assembly.

HTGR advantages are an inherent safety, high thermal efficiency, and a high temperature process heat potential.
Higher fuel temperature achieves higher outlet temperature.
However, undesirably high temperatures jeopardize the integrity of the TRISO particles (as well as the fission products release).
The design limit in HTGR is 1250 C. (check)

Complex geometry of the fuel blocks hinders accurate evaluations of the fuel temperatures requiring elaborate numerical calculations.

The analyses and design of prismatic reactors has widely used simplified computational models, among which we find the equivalent cylinder model and the unit cell model.

The equivalent cylinder model has been widely adopted in the thermo-fluid analyses using one-dimensional and multi-dimensional system codes:
* NO et al. 2007 % no_multi-component_2007
* Reza et al. 2006 % reza_design_2006
* Nakano et al. 2007 % nakano_conceptual_2008

Solved the multi-dimensional head conduction based on the unit cell model:
* Bays et al. 2006
* In et al. 2006 % in_three-dimensional_2006
* Richards et al. 2007 % richards_thermal_2007
The unit cell model assumes that the heat generated in the fuel is only removed by the coolant.

Such simplified approaches are helpful to understand some basic aspects of a heat transfer in prismatic fuel blocks, and they are economical by reducing the computational efforts.
However, the simplified approaches cannot consider a heat transfer within a fuel assembly as well as a coolant flow through a gap between the fuel assemblies, which may significantly affect the fuel temperature.

Evaluations of the thermal behavior of prismatic fuels using more detailed thermo-fluid models than the unit cell model are rare, particularly in the open literatures.
General Atomics analyzed detailed thermal behaviors of the fuel blocks of HTGRs using a model named DEMISE but limited information is available in the book written by Melese and Katz (1984).
* Melese and Katz 1984

Using a CFD code named Trio_U, Cioni et al. (2004) made three-dimensional thermo-fluid simulations of prismatic fuel assemblies of a HTGR.
They modeled seven fuel assemblies in detail and paid attention to a heat transfer between the fuel assemblies with/without a blockage of the coolant channels.
* Cioni et al. 2004

Simoneau et al. (2005) numerically analyzed the passive heat removal characteristics of a prismatic VHTR using a commercial CFD code.
They applied a simplified three-dimensional reactor model.
They used homogenized triangular posts since they were interested in the overall system behavior.
* Simoneau et al. 2005

PMR600: the reference reactor is GT-MHR.
The heat generated in the fuel compacts is conducted through the graphite block and it is finally removed by the helium coolant.

Considering the effects of the dowel holes to be negligible, a 1/12 section of the fuel is enough to simulate the fuel assembly due to its symmetry.
The model includes the graphite plugs.
Uniform power density profile.
The burnup flattens out the axial power profile making it close to uniform at the end of cycle.
Amount of heat generation in the graphite plugs is negligibly small.

Re = 40000 in coolant channel and 2000 in bypass gap.
Increasing the bypass flow gap increases the fuel temperature.
The unit cell model is a reasonable approximation if the bypass flow is small and the radial power profile within the assembly is uniform.

What they did:
* three-dimensional computational fluid dynamics (CFD) analysis carried out on a typical fuel assembly of a prismatic HTGR
* results used to assess the accuracy of the unit cell model
* examine the deviation from the results by the unit cell approach due to the bypass flow between fuel assemblies and radial power variations within the fuel assembly
* flow split a the upper plenum into the flow channels is obtained by a 1D calculation by assuming the same pressure drop across the entire height of the reactor core including top and bottom reflector blocks.
* calculated flow rate is used as an inlet boundary condition for the CFD analysis.
* CFD analysis using CFX 11 (ANSYS Inc.)
* Highest temperature: 1119 C.
* Max coolant velocity 51 m/s
* Pressure drop: 25 kPa

% no_multi-component_2007
The VHTR faces some technical and economical challenges, particularly reactor safety and costs.
The air ingress event following a LOCA is a cause of concern.
Following the depressurization of helium in the core, there exists the potential for air to enter the core through the break and oxidize the in-core graphite structure.

What they did:
* Development of the code GAMMA
* Solves equations for both the gas and the solid parts.
* Use porous media model (Nield and Bejan, 1999) to consider heat transport in solid-fluid components.
* Gas flow medium: multi-dimensional governing equations for a chemically reacting flow (Poinsot and Veynante,
2001) that consist of the spatially-averaged conservation equations for continuity, momentum, energy of the gas mixture, and the mass of each species.
* Heat transfer between the fluid and the solid: h(TWall - TFluid)
* The boundary condition for the enclosure (recinto) is the radiative heat transfer. It uses the irradiation/radiosity method (Holman, 1986). The radiation exchange between surfaces is gray and diffuse.
* Improvement of RELAP5
* They added the capability to simulate air ingress following LOCAs due to the effects of molecular diffusion.

% reza_design_2006
The temperatures required for efficient hydrogen production present a unique design challenge for the reactor pressure vessel during steady-state operation.

What they did:
* Use ATHENA (that relies on RELAP5-3D) to solve MHRs
* peak steady state reactor vessel temperature of 541 C
* Propose an alternative coolant inlet flow configuration
* Calculated peak temperatures in HPCC and LPCC accidents with the new config
high pressure conduction cooldown (HPCC): 5.03 MPa
low pressure conduction cooldown (LPCC): 1 atm
* Max core velocity ~53 m/s
* Total pressure drop 57.87 kPa
* Peak fuel temperature around ~1106C at SS
* Design fuel temperature limit of 1600C

% nakano_conceptual_2008
At the beginning of burnup, the deep insertion of control rods makes the power density larger in the bottom region.
The withdrawal of control rods at the end of cycle leads larger power density at the top of the core.

The heat transfer areas are different between the actual unit cell and the analytical model.
The unit cell is asymmetric making the temperature distribution asymmetric in the graphite block.
It leads to temperature difference between actual and analytical model.

What they did:
* The fuel has a temp profile when the outlet coolant temp is 850.
* The idea is to restrain the bypass flow to leave that temp profile unchanged while increasing the outlet temp to 950.
* 3D FEM model developed on ANSYS v.10 code.
* It calculates the pressure and flow rate distribution using the Bernoulli equation.
* The 3D model includes RPV and core internals and core fuels.
* Calculates 1/3rd of the reactor: 120 degrees.
* Nuclear analysis using DELIGHT-8 to make macro cross-section sets at various burnup points.
* CITATION for the code analysis.
* Flow distribution analysis using FLOWNET: to obtain the fuel channel flow rate and pressure drop/
* Fuel temperature analysis with TAC-2D:
	- Power density from the nuclear analysis.
	- flow rate calculated in the flow distribution analysis.
	- calculation model: 2D cylindrical model for a hot channel fuel pin cell
	- the coolant channel is modeled into an annular channel around the graphite block
* FP release: Diffusion in the particles, sorption, and diffusion in the fuel matrix and in the graphite region, evaporation and mass transfer on the graphite surface into the coolant.

% in_three-dimensional_2006
What they did:
* Predict hot-spot fuel temperature distributions in a PMR at a steady state
* CFD code: CFX-10 used to perform a 3D analysis
* GT-MHR600 is the reference design
* unit cell model
* uniform cooling and constant pressure assumed at the inlet and the exit of the coolant channel
* helium mass flow rate in a coolant channel is 0.0176 kg/s
* inlet/outlet helium temperatures are 399C and 950C.
* q''' = 33.3, 33.1, 35.0 MW/m3 at BOC, MOC, and EOC
* Tmax in the fuel is 1295.3 C at EOC at the exit
* The temperature difference between the coolant (bulk temperature) and the fuel is approx 150C

% richards_thermal_2007

% simoneau_three-dimensional_2007