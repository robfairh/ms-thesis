\section{Contribution}

% Ch1: prismatic htgrs
HTGRs have several desirable features that make them a good candidate for deployment in the near-term.
Some of those features are reliance on passive heat transfer mechanisms, use of TRISO particles, and high operating temperatures.
Higher temperatures offer increased thermal cycle efficiencies and enable a wide range of process heat applications, such as hydrogen production.
Hydrogen can be a decisive response to energy and climate challenges, as it can decarbonize the transport and power sectors.

% Ch1: motivation
To support the evolution of the HTGR technology, this work focused on the modeling of prismatic HTGRs.
Modeling the multi-physics of prismatic HTGRs enables the prediction of the reactor thermal-fluid behavior which is necessary for assessing the safety characteristics of the reactor.
The HTGR complex geometry requires numerical tools to conduct the analyses.
Moltres is a simulation tool suitable for multi-physics problems.
Although its original development targeted MSRs, this work studied its applicability to prismatic HTGRs.

% Ch4: Neutronics
Multi-physics simulators need to resolve the double heterogeneities present in the prismatic HTGR fuel assemblies.
Monte Carlo simulators are capable of explicitly modeling TRISO particles.
Although using such a capability is computationally expensive, Chapter \ref{ch:neutronics} proved it necessary for obtaining group constants for diffusion solvers.
Diffusion solvers rely on different levels of homogenization.
Moltres previous work focused on MSRs which allow for a more heterogeneous homogenization.
Nonetheless, HTGRs require a higher level of homogenization making Moltres not easily applicable to HTGRs.
This work studied using Moltres as a homogeneous solver for carrying out neutronics stand-alone simulations of prismatic HTGRs.
The first study comprised an analysis of the effects of the energy group structure on the simulations of a fuel column.
Such a study compared Moltres results to Serpent results.
Based on that study, the following section conducted a full-core simulation in Moltres using a 15-energy group structure.
We compared Moltres results to Serpent results, and overall, the results showed good agreement.
Finally, Moltres carried out Phase I Exercise 1 of the OECD/NEA MHTGR-350 MW Benchmark.
The exercise provides the group constants which required the development of a tool to adapt them to Moltres format.
The exercise models one-third of the MHTGR-350 and uses periodic boundary conditions on the sides.
Imposing those boundary conditions in Moltres was not possible because of the simulation high memory requirements.
To circumvent this barrier, Moltres conducted the exercise using reflective boundary conditions.
We also analyzed the effects of such an approximation as some discrepancies in the results arose.
We concluded that the boundary condition approximation was responsible for causing such discrepancies.

% Ch5: Thermal-fluids
Multi-physics simulators need to resolve the temperature in the different regions of the reactor.
Moltres is a MSR simulator, and its thermal-fluids kernels applicability to prismatic HTGRs is not straightforward.
Chapter \ref{ch:thermalfluids} focused on developing a thermal-fluids model for Moltres.
A first preliminary study verified the proposed model by comparing the numerical solution to a known analytical solution.
A second preliminary study applied the thermal-fluids model to a unit cell problem and compared Moltres results to published article.
Both preliminary studies showed good results.
Another study demonstrated the use of Moltres on a larger-scale problem, a fuel column.
We compared Moltres results to a published article for two cases, a no-bypass-gap case and a 3mm-bypass-gap case.
Both cases exhibited good results.
As part of the study, two analyses focused on different aspects of the simulations.
The first analysis studied the modeling of the mass flow distribution of the coolant in the different coolant paths.
The second analysis studied the mesh convergence of the full-fuel column problem.
This analysis exposes that a high level of detail in our model may restrict the model's applicability to a full-core problem.
The next study used a different approach for the thermal-fluids modeling of the MHTGR-350.
This study solved Phase I Exercise 2 of OECD/NEA MHTGR-350 MW Benchmark.
The exercise revealed that a more accurate mass flow distribution and the consideration of the radiative heat transfer mechanism are necessary for the correct modeling of prismatic HTGRs.

% Ch6: Hydrogen
World-wide climate change demands a large scale deployment of \gls{CO2}-free energy sources.
UIUC is fighting climate change by actively reducing GHG emissions on its campus.
This work comprises efforts that align with UIUC's goals to reduce \gls{CO2} emissions from the electricity generation and the transportation sectors.
This work proposed an alternative: the deployment of a nuclear reactor and hydrogen production.
Regarding hydrogen production methods, we surveyed three different processes: LTE, HTE, and SI.
We developed a tool to calculate their energy requirements, regarding electricity and heat, and hydrogen production rates.
This tool is applicable to a stand-alone hydrogen plant and a nuclear power plant that produces both electricity and hydrogen.

\section{Future Work}

This section introduces some possible future work as a continuation of this thesis.

% Ch4: Neutronics
As mentioned earlier, the periodic boundary conditions impose a high memory requirement on Moltres simulations.
Moltres relies on PetSc routines for solving the equation systems.
Future work will look for a PetSc routine to enable the exact modeling of Phase I Exercise 1 of the benchmark, i.e. conduct the simulations with the periodic boundary conditions.

% Ch5: Thermal-fluids
The right modeling of Phase I Exercise 2 of the benchmark ensures the right modeling of the coupled exercises.
Moltres results for exercise 2a showed some discrepancies to INL benchmark results.
As discussed earlier, the flat flow approximation may be the cause.
Future work will add mass flow distribution capabilities to Moltres.
% This simplified algorithms allow for solving the mass flow rate distribution in the core for steady-state cases, and for transient cases as an approximation.
% However, in coupled analyses, the flow distribution depends on the temperature and will change with respect to time.
% This creates the necessity for developing tools integrated into Moltres.
% Currently, MOOSE has a module for modeling the incompressible Navier-Stokes equations.
% Integrating that module into the solver could improve the accuracy.
% This task will be part of the future work.
Additionally, Section \ref{sec:ph1ex2} first analysis intended to reproduce INL results.
That analysis reveals that Moltres modeling fails to capture the right heat transfer between different assemblies.
Not modeling the radiative heat transfer between elements might be causing this behavior.
Future analyses will try add the radiate heat transfer modeling capabilities to Moltres.

% Ch5: thermal-fluids
- 3D Thermal-fluids
	- porous media model
	- Multi-physics

% Thermo-mechanics capabilities in Moltres

% Ch6: Hydrogen
As \ref{ch:hydro} mentioned, high temperatures enable efficient hydrogen production.
However, the current fleet of nuclear reactors are LWRs with outlet temperatures around 300$^{\circ}$C.
Additionally, HTE and SI are viable processes for temperatures well above 300$^{\circ}$C.
In the short term, the development of hydrogen economies demand mature technologies.
Related future work will study the steam reforming method, which could use or not carbon-capture sequestration systems.
The future work will analyze the feasibility of using a carbon-capture sequestration system from an economics perspective.
For the case the process does not use a carbon-capture sequestration system, we will compare the \gls{CO2} savings versus the \gls{CO2} production.
Moreover, related future work will study the viability of combining LWRs to HTE and SI processes via steam boosting systems.
These systems rely on the use of electric heaters to enhance the steam temperature.
These steam boosting systems could enable the coupling of hydrogen plants to the LWR fleet, and it is a promising short term solution to ease climate change.
