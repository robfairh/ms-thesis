\section{Contribution}

% Ch1: prismatic htgrs
The development of the HTGR technology begun almost 60 years ago.
HTGRs have several desirable features that make them a good candidate for deployment in the near-term.
Some of those features are reliance on passive heat transfer mechanisms, use of TRISO particles, and high operating temperatures.
Higher temperatures offer increased cycle efficiencies and enable a wide range of process heat applications, such as hydrogen production.
Hydrogen can be a decisive response to energy and climate challenges, as it can decarbonize the transport and power sectors.
Several hydrogen production processes benefit from high temperatures, such as high-temperature electrolysis or thermochemical water splitting.
Utilizing high-temperature nuclear power plants as the energy source of the process eliminates the need to burn fossil fuels.

% Ch1: motivation
To support the evolution of the HTGR technology, this work focused on the extension of Moltres capabilities to prismatic HTGRs.
Modeling and prediction of core thermal-hydraulic behavior is necessary for assessing the safety characteristics of a reactor.
The fuel blocks’ complex geometry requires elaborate numerical calculations.
The characteristics of an HTGR are different from those of conventional Light Water Reactors (LWRs).
Such differences demand for new reactor analysis tools.
Historically, the operator-splitting technique allowed for coupling stand-alone neutronics solver to a thermal-hydraulics solver and modeling a reactor.
However, solving the fully-coupled system is preferable over using the operator-splitting technique.
Moltres can solve systems of equations in a fully-coupled way or solve systems of equations independently. This feature makes Moltres suitable for solving multi-physics problems and a wide range of nuclear engineering problems.

% Ch1: Objectives
This work extends Moltres modeling capabilities to prismatic HTGRs.
% left here

% Ch4: Neutronics


% Ch5: Thermal-fluids


% Ch6: Hydrogen
World-wide climate change demands a large scale deployment of \gls{CO2}-free energy sources.
UIUC is fighting climate change by actively reducing GHG emissions on its campus.
This work comprises efforts that align with UIUC's goals to reduce \gls{CO2} emissions from the electricity generation and the transportation sectors.
This work proposed an alternative: the deployment of a nuclear reactor and hydrogen production.
Regarding hydrogen production methods, we surveyed three different processes: LTE, HTE, and SI.
We developed a tool to calculate their energy requirements, regarding electricity and heat, and hydrogen production rates.
This tool is applicable to a stand-alone hydrogen plant and a nuclear power plant that produces both electricity and hydrogen.

\section{Future Work}

This section introduces some possible future work as a continuation of this thesis.

% Moltres extensions
- 3D Thermal-fluids
	- radiative heat transfer
	- porous media model
- Multi-physics
- Thermo-mechanics

% Thermal-fluids
This simplified algorithms allow for solving the mass flow rate distribution in the core for steady-state cases, and for transient cases as an approximation.
However, in coupled analyses, the flow distribution depends on the temperature and will change with respect to time.
This creates the necessity for developing tools integrated into Moltres.
Currently, MOOSE has a module for modeling the incompressible Navier-Stokes equations.
Integrating that module into the solver could improve the accuracy.
This task will be part of the future work.



% Ch4: Neutronics


% Ch6: Hydrogen
As \ref{ch:hydro} mentioned, high temperatures enable efficient hydrogen production.
However, the current fleet of nuclear reactors are LWRs with outlet temperatures around 300$^{\circ}$C.
Additionally, HTE and SI are viable processes for temperatures well above 300$^{\circ}$C.
In the short term, the development of hydrogen economies demand mature technologies.
Related future work will study the steam reforming method, which could use or not carbon-capture sequestration systems.
The future work will analyze the feasibility of using a carbon-capture sequestration system from an economics perspective.
For the case the process does not use a carbon-capture sequestration system, we will compare the \gls{CO2} savings versus the \gls{CO2} production.
Moreover, related future work will study the viability of combining LWRs to HTE and SI processes via steam boosting systems.
These systems rely on the use of electric heaters to enhance the steam temperature.
These steam boosting systems could enable the coupling of hydrogen plants to the LWR fleet, and it is a promising short term solution to ease climate change.
