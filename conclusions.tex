\section{Contribution}

% Ch1: prismatic htgrs
The development of the HTGR technology begun almost 60 years ago.
HTGRs have several desirable features that make them a good candidate for deployment in the near-term.
Some of those features are reliance on passive heat transfer mechanisms, use of TRISO particles, and high operating temperatures.
Higher temperatures offer increased cycle efficiencies and enable a wide range of process heat applications, such as hydrogen production.
Hydrogen can be a decisive response to energy and climate challenges, as it can decarbonize the transport and power sectors.
Several hydrogen production processes benefit from high temperatures, such as high-temperature electrolysis or thermochemical water splitting.
Utilizing high-temperature nuclear power plants as the energy source of the process eliminates the need to burn fossil fuels.

% Ch1: motivation
To support the evolution of the HTGR technology, this work focused on the extension of Moltres capabilities to prismatic HTGRs.
Modeling and prediction of core thermal-hydraulic behavior is necessary for assessing the safety characteristics of a reactor.
The fuel blocks’ complex geometry requires elaborate numerical calculations.
The characteristics of an HTGR are different from those of conventional Light Water Reactors (LWRs).
Such differences demand for new reactor analysis tools.
Historically, the operator-splitting technique allowed for coupling stand-alone neutronics solver to a thermal-hydraulics solver and modeling a reactor.
However, solving the fully-coupled system is preferable over using the operator-splitting technique.
Moltres can solve systems of equations in a fully-coupled way or solve systems of equations independently. This feature makes Moltres suitable for solving multi-physics problems and a wide range of nuclear engineering problems.

% Ch1: Objectives
This work extends Moltres modeling capabilities to prismatic HTGRs.
% left here

% Neutronics
% Thermal-fluids

% Hydrogen
Additionally, this work studied different hydrogen production that can be coupled to a nuclear power plant.


\section{Future Work}

This section introduces some possible future work as a continuation of this thesis.

% Moltres extensions
- 3D Thermal-fluids
	- radiative heat transfer
	- porous media model
- Multi-physics
- Thermo-mechanics

% Hydrogen extensions
- HTE with steam temperature boosting