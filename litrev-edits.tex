Neutronics:

- CAPP
lee_development_2008
reitsma_oecd-neansc_2008
lee_development_2011
stammler_helios_1998
tak_cappgamma_2016
lim_gamma_2006

- Pronghorn and rattlesnake
wang_rattlesnake_2019
strydom_inl_2013
j_ortensi_initial_2012
oecd_nea_coupled_2020
strydom_inl_2013

Thermal-fluids
- Simple approaches

- More elaborated approaches


Multi-physics:
- studer_cast3marcturus_2007
- sanchez_apollo2_1999
- cavalier_presentation_2005
- lee_development_2011
- tak_coupled_2016
- kaeri_decart_2007
- tak_cappgamma_2016

















Stainsby's approach
% stainsby_investigation_2008
based on Stainsby's approach \cite{stainsby_investigation_2008}.
This approach uses different lenght scale models to solve the temperature distribution.


Why the temperature has to be averaged:
% stainsby_investigation_2008
Each fuel compact contains thousand of TRISO coated particles, and ultimately we require the average and maximum temperatures of the fuel kernels and surrounding coatings, both for use in neutronics calculations and to be able to estimate fuel failure fractions and fission product release rates.