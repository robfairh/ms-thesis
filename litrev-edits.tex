Neutronics:
- CRONOS
lautard_cronos_1990
damian_vhtr_2008

- CAPP
lee_development_2008
reitsma_oecd-neansc_2008
lee_development_2011
stammler_helios_1998
tak_cappgamma_2016
lim_gamma_2006
yuk_time-dependent_2020

- Pronghorn and rattlesnake
wang_rattlesnake_2019
strydom_inl_2013
j_ortensi_initial_2012
oecd_nea_coupled_2020
strydom_inl_2013

Thermal-fluids
- Simple approaches

- More elaborated approaches


Multi-physics:
- damian_vhtr_2008
- studer_cast3marcturus_2007
- sanchez_apollo2_1999
- cavalier_presentation_2005
- lee_development_2011
- tak_coupled_2016
- kaeri_decart_2007
- tak_cappgamma_2016
- yuk_time-dependent_2020

% CRONOS
\gls{CEA} developed CRONOS2 \cite{lautard_cronos_1990} as part of the SAPHYR system.
It conducts steady-state and transient multi-group calculations, taking into account thermal-fluids feedback effects.
CRONOS2 solves either the diffusion equation or the transport equation using the S$_N$ method and an \gls{FDM} or a \gls{FEM} discretization.
In 2008, Damian et al. \cite{damian_vhtr_2008} conducted a study aimed at understanding the physical aspects of the annular core and the passive safety features of a standard block type \gls{HTGR}.
For the study, the authors developed the code suite NEPTHIS/CAST3M, which relied on CRONOS2.

% damian_vhtr_2008
In 2008, Damian et al. \cite{damian_vhtr_2008} conducted a study aimed to understand the passive safety features of a prismatic \gls{HTGR}.
They performed analyses on various geometrical scales, including: unit cell and fuel columns located at the core hot-spot and two-dimensional and three-dimensional core configurations, including the coupling between neutronics and thermal-fluids.
The first part of the assessment concerns thermal calculations on steady-state core configurations and used CAST3M \cite{studer_cast3marcturus_2007} to solve the three-dimensional heat conduction equation in the solid and the one-dimensional thermal-fluid equations in the coolant.
The second part of the assessment used the transport code APOLLO2 \cite{sanchez_apollo2_1999} on a two-dimensional core configuration to minimize the radial power peaking factor.
The analyses included the variation of several parameters, such as fuel enrichment, fuel loading, and the fuel management scheme.
The fuel enrichment variation had the most substantial impact.
The last part of the study analyzed a three-dimensional core model using the coupled software NEPTHIS \cite{cavalier_presentation_2005} and CAST3M/Arcturus for calculating the neutronics and the thermal-fluids.
NEPHTIS used a transport-diffusion calculation scheme that relied on APOLLO2 and the diffusion code CRONOS2.
The CAST3M/Arcturus model used a two-level approach.
On the first level, the porous media model solved the homogenized system and the coolant.
On the second level, CAST3M solved the thermal-fluids on the homogenized geometry.
The authors conducted several parametric studies and assessed their impact on the power distribution.
The studies included the variation of the helium bypass fraction, average power density, core geometry, reflector materials, and fuel loading strategy.
Their results exhibited that with the reduction of the bypass fraction, the average reflector temperature rises.









Stainsby's approach
% stainsby_investigation_2008
based on Stainsby's approach \cite{stainsby_investigation_2008}.
This approach uses different lenght scale models to solve the temperature distribution.


Why the temperature has to be averaged:
% stainsby_investigation_2008
Each fuel compact contains thousand of TRISO coated particles, and ultimately we require the average and maximum temperatures of the fuel kernels and surrounding coatings, both for use in neutronics calculations and to be able to estimate fuel failure fractions and fission product release rates.