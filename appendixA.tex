\label{appendix:equations}

\section{Neutron flux equations}
\label{appendix:equations-n}

Equations \ref{eq:diffusion} and \ref{eq:precursors} describe the time dependent behavior of the neutron flux and the concentration of the delayed neutron precursors

\begin{align}
  % diffusion
  & \frac{1}{v_g}\frac{\partial}{\partial t} \phi_g = \nabla \cdot D_g \nabla \phi_g - \Sigma_g^r \phi_g +
  \sum_{g \ne g'}^G \Sigma_{g'\rightarrow g}^s \phi_{g'} + \chi_g^p \sum_{g' = 1}^G (1 - \beta) \nu \Sigma_{g'}^f \phi_{g'} + 
  \chi_g^d \sum_i^I \lambda_i C_i \label{eq:diffusion} \\
  % precursors
  & \frac{\partial}{\partial t} C_i = \sum_{g'= 1}^G \beta_i \nu \Sigma_{g'}^f \phi_{g'} - \lambda_i C_i \label{eq:precursors}
  \intertext{where}
  & v_g = \mbox{group $g$ neutron speed } [cm \cdot s^{-1}] \notag \\
  & \phi_g = \mbox{group $g$ neutron flux } [n \cdot cm^{-2} \cdot s^{-1}] \notag \\
  & t = \mbox{time } [s] \notag \\
  & D_g = \mbox{group $g$ diffusion coefficient } [cm] \notag \\
  & \Sigma_g^r = \mbox{group $g$ macroscopic removal cross-section } [cm^{-1}] \notag \\
  & \Sigma_{g'\rightarrow g}^s = \mbox{group $g'$ to group $g$ macroscopic scattering cross-section } [cm^{-1}] \notag \\
  & \chi_g^p = \mbox{group $g$ prompt fission spectrum } [-] \notag\\
  & G = \mbox{number of discrete energy groups } [-] \notag \\
  & \nu = \mbox{number of neutrons produced per fission } [-] \notag \\
  & \Sigma_g^f = \mbox{group $g$ macroscopic fission cross-section } [cm^{-1}] \notag \\
  & \chi_g^d = \mbox{group $g$ delayed fission spectrum } [-] \notag \\
  & I = \mbox{number of delayed neutron precursor groups } [-] \notag \\
  & \beta = \mbox{delayed neutron fraction } [-] \notag \\
  & \lambda_i = \mbox{average decay constant of delayed neutron precursors in precursor group $i$ } [s^{-1}] \notag \\
  & C_i = \mbox{concentration of delayed neutron precursors in precursor group $i$ } [cm^{-3}]. \notag
\end{align}

The following equation relates $\chi_g^t$ to $\chi_g^p$ and $\chi_g^d$ \cite{hetrick_dynamics_1973}
\begin{align}
  & \chi_g^t = \chi_g^p (1 - \beta) + \chi_g^d \sum_i^I \beta_i  \label{eq:chit} \\
  \intertext{where}
  & \chi_g^t = \mbox{group $g$ total fission spectrum } [-]. \notag
\end{align}

The combination of the steady-state of equations \ref{eq:diffusion}, \ref{eq:precursors}, \ref{eq:chit}, and replacing $\nu\Sigma_g^f$ by $\frac{\nu\Sigma_g^f}{k_{eff}}$ yields the eigenvalue equation \cite{duderstadt_nuclear_1976}
\begin{align}
  \nabla \cdot D_g \nabla \phi_g - \Sigma_g^r \phi_g & + \sum_{g \ne g'}^G \Sigma_{g'\rightarrow g}^s \phi_{g'} +
  \chi_g^t \sum_{g' = 1}^G \frac{1}{k_{eff}}\nu \Sigma_{g'}^f \phi_{g'} = 0 \label{eq:eigenvalue}
  \intertext{where}
  k_{eff} = \mbox{multiplication factor } [-]. \notag
\end{align}

\section{Thermal-fluids equations}
\label{appendix:equations-th}

% solids
The three-dimensional heat conduction equation \cite{melese_thermal_1984} allows for solving the temperature in the fuel, helium gap, moderator, coolant film, and reflector.
\begin{align}
  & \rho_i c_{p,i} \frac{\partial}{\partial t} T_i = k_i \nabla^2 T_i + Q_i \label{eq:app-solid} \\
  \intertext{where}
  & i = \mbox{f (fuel), g (helium gap), m (moderator), cf (coolant film), r (reflector)} \notag \\
  & \rho_i = \mbox{material $i$ density } [kg \cdot cm^{-3}] \notag \\
  & c_{p,i} = \mbox{material $i$ heat capacity } [J \cdot kg^{-1} \cdot K^{-1}] \notag \\
  & k_i = \mbox{material $i$ thermal conductivity  } [W \cdot cm^{-1} \cdot K^{-1}] \notag \\
  & T_i = \mbox{material $i$ temperature } [^{\circ}C] \notag \\
  & Q_i = \mbox{material $i$ volumetric heat source } [W \cdot cm^{-3}]. \notag
\end{align}

Equations \ref{eq:app-heatsource1} and \ref{eq:app-heatsource2} define the fuel heat source in the stand-alone and coupled calculations
\begin{align}
  & Q_f = Q_0 \label{eq:app-heatsource1} \\
  & Q_f = \sum_{g = 1}^{G} \epsilon_g^f \Sigma_g^f \phi_g \label{eq:app-heatsource2} \\
  & Q_g = Q_m = Q_{cf} = Q_r = 0 \label{eq:heatsource3}
  \intertext{where}
  & Q_i = \mbox{material $i$ volumetric heat source } [W \cdot cm^{-3}] \notag \\
  & \epsilon_g^f = \mbox{energy released per fission } [J] \notag \\
  & \Sigma_g^f = \mbox{group $g$ macroscopic fission cross-section } [cm^{-1}] \notag \\
  & \phi_g = \mbox{group $g$ neutron flux } [n \cdot cm^{-2} \cdot s^{-1}]. \notag
\end{align}

% coolant
The governing equation of the coolant is the one-dimensional form of the continuity, momentum, and energy conservation equations \cite{white_viscous_2006}

\begin{align}
  & \frac{\partial}{\partial t} \rho_c + \nabla \cdot (\rho_c u) = 0 \label{eq:app-continuity} \\
  & \rho_c \left(\frac{\partial}{\partial t} u + u\frac{\partial}{\partial z}u \right) = - \frac{\partial}{\partial z}p - \tau \frac{\varepsilon}{A} - \rho_c g \label{eq:app-momentum} \\
  & \rho_c \left( \frac{\partial}{\partial t} (c_{p,c} T_c) + u\frac{\partial}{\partial z} (c_{p,c} T_c) \right) = \frac{\partial}{\partial t} p + u\frac{\partial}{\partial z} p +  q'''_{conv}  \label{eq:app-tempcool} \\
  & \tau = \frac{f}{2} \rho_c u^2 \label{eq:app-friction} \\
  & q'''_{conv} = h\frac{\varepsilon}{A} (T_i-T_c) \label{eq:app-convection}
  \intertext{where}
  & \rho_c = \mbox{coolant density } [kg \cdot cm^{-3}] \notag \\
  & u = \mbox{coolant velocity } [cm \cdot s^{-1}] \notag \\
  & p = \mbox{coolant pressure } [\times 10^{-2} Pa] \notag \\
  & \tau = \mbox{shear stress } [\times 10^{-2} Pa] \notag \\
  & \varepsilon = \mbox{wetted perimeter } [cm] \notag \\
  & A = \mbox{cross-sectional area } [cm^2] \notag \\
  & g = \mbox{gravity } [m \cdot s^{-2}] \notag \\
  & c_{p,c} = \mbox{coolant specific heat capacity } [J \cdot kg^{-1} \cdot K^{-1}] \notag \\
  & T_c = \mbox{coolant temperature } [^{\circ}C] \notag \\
  & k_c = \mbox{coolant thermal conductivity } [W \cdot cm^{-1} \cdot K^{-1}] \notag \\
  & q'''_{conv} = \mbox{convective heat transfer } [W \cdot cm^{-3}] \notag \\
  & f = \mbox{friction factor } [-] \notag \\
  & h = \mbox{heat transfer coefficient } [W \cdot cm^{-2} \cdot K^{-1}] \notag \\
  & T_i = \mbox{solid temperature } [^{\circ}C]. \notag
\end{align}

Equation \ref{eq:app-churchill} \cite{churchill_friction-factor_1977} determines the friction factor $f$

\begin{align}
  & f = 8\left[ \left( \frac{8}{Re} \right)^{12} + \frac{1}{(A+B)^{3/2}} \right]^{1/12} \notag \\
  & A = \left\{ 2.457 \quad ln \left( \frac{1}{\left(\frac{7}{Re_{D_h}}\right)^{0.9}+0.27\frac{\varepsilon}{D_h}} \right) \right\}^{16} \label{eq:app-churchill} \\
  & B = \left\{ \frac{37530}{Re} \right\}^{16} \notag
  \intertext{where}
  & \varepsilon = \mbox{surface roughness } [-] \notag \\
  & Re = \mbox{Reynolds number } [-] \notag \\
  & D_h = \mbox{hydraulic diameter } [cm]. \notag
\end{align}

Equation \ref{eq:film-conduc} calculates the film thermal conductivity $k_f$ \cite{melese_thermal_1984}

\begin{align}
  & Nu = 0.023 Re^{0.8} Pr^{0.4} \label{eq:dittus} \\
  & h = \frac{Nu \cdot k_c}{D_h}  \\
  & k_f = h R_{cf} ln(R_{cf}/R_c) \label{eq:film-conduc}
  \intertext{where}
  & Nu = \mbox{Nusselt number } [-] \notag \\
  & Pr = \mbox{Prandtl number } [-] \notag \\
  & h = \mbox{heat transfer coefficient } [W \cdot cm^{-2} \cdot s^{-1}] \notag \\
  & D_h = \mbox{hydraulic diameter } [cm] \notag \\
  & k_c = \mbox{coolant thermal conductivity } [W \cdot cm^{-1} \cdot K^{-1}] \notag \\
  & k_{cf} = \mbox{coolant film thermal conductivity } [W \cdot cm^{-1} \cdot K^{-1}] \notag \\
  & R_{cf} = \mbox{coolant film radius } [cm] \notag \\
  & R_c = \mbox{coolant channel radius } [cm]. \notag
\end{align}

In the steady-state limit, equation \ref{eq:app-continuity} becomes
\begin{align}
  & \nabla \cdot (\rho_c u) = 0
  \intertext{wich leads to } 
  & \rho_c u (z) = \rho_{c,i} u_i 
  \intertext{where} 
  & \rho_{c,i} = \mbox{inlet coolant density} \notag \\
  & u_i = \mbox{inlet coolant velocity.} \notag
\end{align}

In the steady-state limit, the temperature equations (equations \ref{eq:app-solid} and \ref{eq:app-tempcool}) become \cite{tak_practical_2012}
\begin{align}
  & k_i \nabla^2 T_i + Q_i = 0 \\
  & \rho_{c,i} u_i\frac{\partial}{\partial z} (c_{p,c} T_c) = q'''_{conv}.
\end{align}


\section{Coolant distribution equation}
\label{appendix:equations-fluid}

The pressure drop in a coolant channel is proportinal to the mass flow squared 
\begin{align}
  & \Delta P = B_i \dot{m}_i^2 \label{eq:app-itersolver1}
  \intertext{where}
  & \Delta P = \mbox{pressure drop } [Pa] \notag \\
  & \dot{m}_i = \mbox{channel $i$ mass flow rate } [kg \cdot s^{-1}] \notag \\
  & B_i = \mbox{constant specified by the chosen model, value that depends on $\dot{m}_i$} [kg \cdot m^{-1}] \notag
\end{align}
which is equivalent to
\begin{align}
  & \dot{m}_i = \sqrt{\frac{\Delta P}{B_i}}. \label{eq:app-itersolver2}
\end{align}

Assuming constant $\Delta P$ across all coolant paths, and using equation \ref{eq:app-itersolver2}, it is possible to calculate the total mass flow rate $m_T$
\begin{align}
  & \dot{m}_T = \sum_i \dot{m}_i = \sum_i \sqrt{\frac{\Delta P}{B_i}} = \Delta P \frac{1}{\sum_i \sqrt{B_i}}
  \intertext{where}
  & \dot{m}_T = \mbox{total mass flow rate } [kg \cdot s^{-1}]. \notag
\end{align}

Finally, equation \ref{eq:app-itersolver3} gives $\Delta P$
\begin{align}
  & \Delta P = \frac{\dot{m}_T}{\sum_i \frac{1}{\sqrt{B_i}}}. \label{eq:app-itersolver3}
\end{align}

These equations solve $\dot{m}_i$ using an iterative scheme as $B_i$ depends on $\dot{m}_i$ \cite{melese_thermal_1984}
\begin{align}
  & B_i^{(n)} = f(\dot{m}_i^{(n)})
  & \Delta P^{(n)} = \frac{\dot{m}_T}{\sum_i \frac{1}{\sqrt{B_i^{(n)}}}}
  & \dot{m}_i^{(n+1)} = \sqrt{\frac{\Delta P^{(n)}}{B_i^{(n)}}}.
\end{align}