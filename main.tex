\documentclass[edeposit,fullpage]{uiucthesis2018}

\usepackage[acronym,toc]{glossaries}
\newacronym{BSD}{BSD}{Berkeley Software Distribution}
\newacronym{CR}{CR}{control rod}
\newacronym{EOEC}{EOEC}{end of equilibrium cycle}
\newacronym{FDM}{FDM}{Finite Difference Method}
\newacronym{FEM}{FEM}{Finite Element Method}
\newacronym{GRS}{GRS}{Gesellschaft für Anlagen und Reaktorsicherheit}
\newacronym{He}{He}{helium}
\newacronym{HZDR}{HZDR}{Helmholtz-Zentrum Dresden-Rossendorf}
\newacronym{HTGR}{HTGR}{High Temperature Gas-Cooled Reactor}
\newacronym{HTR}{HTR}{High Temperature Reactor}
\newacronym{HTTR}{HTTR}{High Temperature Test Reactor}
\newacronym{IPyC}{IPyC}{inner pyrolitic carbon}
\newacronym{INL}{INL}{Idaho National Laboratory}
\newacronym{JFNK}{JFNK}{Jacobian-Free Newton-Krylov}
\newacronym{KAERI}{KAERI}{Korea Atomic Energy Research Institute}
\newacronym{LGPL}{LGPL}{Lesser GNU Public License}
\newacronym{LWR}{LWR}{Light Water Reactor}
\newacronym{MC}{MC}{Monte Carlo}
\newacronym{MHTGR}{MHTGR}{Modular High-Temperature Gas-Cooled Reactor}
\newacronym{MOOSE}{MOOSE}{Multiphysics Object-Oriented Simulation Environment}
\newacronym{MSR}{MSR}{Molten Salt Reactor}
\newacronym{NEA}{NEA}{Nuclear Energy Agency}
\newacronym{NEM}{NEM}{Nodal Expansion Method}
\newacronym{NRC}{NRC}{Nuclear Regulatory Commission}
\newacronym{NSC}{NSC}{Nuclear Science Committee}
\newacronym{OECD}{OECD}{Organisation for Economic Co-operation and Development}
\newacronym{OPyC}{OPyC}{outer pyrolitic carbon}
\newacronym{PBMR}{PBMR}{Pebble Bed Modular Reactor}
\newacronym{PDE}{PDE}{partial differential equation}
\newacronym{PMR}{PMR}{Prismatic Modular Reactor}
\newacronym{RSD}{RSD}{Relative Standard Deviation}
\newacronym{SD}{SD}{Standard Deviation}
\newacronym{SiC}{SiC}{silicon carbide}
\newacronym{SNU}{SNU}{Seoul National University}
\newacronym{TRISO}{TRISO}{Tristructural Isotropic}
\newacronym{UIUC}{UIUC}{University of Illinois at Urbana-Champaign}
\newacronym{UNIST}{UNIST}{Ulsan National Intitute of Science and Technology}
\newacronym{UK}{UK}{United Kingdom}
\newacronym{UMICH}{UMICH}{Universtiy of Michigan}
\newacronym{US}{US}{United States}
\newacronym{VHTR}{VHTR}{very high temperature reactor}
%\newacronym{<++>}{<++>}{<++>}
%\newacronym{<++>}{<++>}{<++>}


\usepackage{xspace}
\usepackage{graphics}

\usepackage{placeins}
\usepackage{booktabs} % nice rules (thick lines) for tables
\usepackage{microtype} % improves typography for PDF

\usepackage[hyphens]{url}
\usepackage{hyperref}
\usepackage{subfig}
\usepackage{hhline}
\usepackage{amsmath}
\usepackage{color}
\usepackage{multirow}
\usepackage{siunitx}
\sisetup{
    input-decimal-markers = .,input-ignore = {,},table-number-alignment = right,
    group-four-digits = true
}
\usepackage{fourier}
\usepackage{booktabs}
\newcommand\tab[1][1cm]{\hspace*{#1}}

\usepackage{threeparttable, tablefootnote}

%tikzpicture fit to page width
\usepackage{environ}
\makeatletter
\newsavebox{\measure@tikzpicture}
\NewEnviron{scaletikzpicturetowidth}[1]{%
  \def\tikz@width{#1}%
  \def\tikzscale{1}\begin{lrbox}{\measure@tikzpicture}%
  \BODY
  \end{lrbox}
  \pgfmathparse{#1/\wd\measure@tikzpicture}%
  \edef\tikzscale{\pgfmathresult}%
  \BODY
}

\usepackage{tabularx}
\newcolumntype{b}{>{\hsize=1.0\hsize}X}
\newcolumntype{q}{>{\hsize=0.5\hsize}X}
\newcolumntype{R}{>{\raggedleft\arraybackslash\hsize=0.5\hsize}X}
\newcolumntype{z}{>{\hsize=0.75\hsize}X}
\newcolumntype{s}{>{\hsize=.5\hsize}X}
\newcolumntype{m}{>{\hsize=.75\hsize}X}

\usepackage{cleveref}
\usepackage{datatool}
\usepackage[numbers]{natbib}
\usepackage{notoccite}

\usepackage{tikz}
\usetikzlibrary{positioning, arrows, decorations, shapes}

\usetikzlibrary{shapes.geometric,arrows}
\tikzstyle{process} = [rectangle, rounded corners, minimum width=2.5cm, minimum height=1cm,text centered, draw=black, fill=blue!30]

\tikzstyle{object} = [ellipse, rounded corners, minimum width=3cm, minimum height=1cm,text centered, draw=black, fill=green!30]
\tikzstyle{objectr} = [ellipse, rounded corners, minimum width=3cm, minimum height=1cm,text centered, draw=black, fill=red!30]

\tikzstyle{empty} =  [rectangle, rounded corners, minimum width=2.5cm, minimum height=0.7cm,text centered, draw=black, fill=white!30]
\tikzstyle{arrow} = [thick,->,>=stealth]

%% Added by me
\usepackage{tabularx}
\usepackage{float}
\usepackage{enumitem}
% \usepackage{subcaption}
% \usepackage[titletoc]{appendix}
\usepackage{appendix}

%\title{Moltres application to prismatic gas-cooled reactors}
%\title{Moltres application to prismatic gas-cooled reactors and high-temperature hydrogen production}
\title{Multi-physics and technical analysis of high-temperature gas-cooled reactors for hydrogen production}
\author{Roberto E. Fairhurst Agosta}
\department{Nuclear, Plasma, and Radiological Engineering}
\concentration{Computational Science and Engineering}
% \schools{B.S., University of Illinois - Urbana Champaign, 2017}
\msthesis
\advisor{Kathryn D. Huff}
\degreeyear{2020}
\committee{Assistant Professor Kathryn D. Huff, Advisor \\ Associate Professor and Associate Head for Undergraduate Programs, Tomasz Kozlowski}

\begin{document}
\maketitle

\frontmatter
%% Create an abstract that can also be used for the ProQuest abstract.
%% Note that ProQuest truncates their abstracts at 350 words.
\begin{abstract}

% climate change and HTGRs role
The future energy needs require the development of clean energy sources to ease the increasing environmental concerns.
High-Temperature Gas-cooled Reactors have several desirable features that make them ideal candidates for the near-future large-scale deployment.
Some of these features are a high temperature and high thermal cycle efficiency, which enable a wide range of process heat applications, such as hydrogen production.
Implementing hydrogen economies can decarbonize the transport and power sectors, offering an alternative to ease climate change.

% what did this work do
This work uses Moltres as the primary simulation tool.
Although Moltres original development targeted Molten Salt Reactors, this work studies Moltres applicability to multi-physics simulations of prismatic High-Temperature Gas-cooled Reactors.
Multi-physics simulations are necessary for assessing reactor safety characteristics.
Ensuring Moltres' multi-physics modeling capabilities requires assessing the independent modeling capabilities of the different involved physical phenomena.
Therefore, this thesis breaks down the analysis into three parts: stand-alone neutronics, stand-alone thermal-fluids, and coupled neutronics/thermal-fluids.

% results: ch4
Regarding stand-alone neutronics, several analyses compare the results calculated by Moltres and Serpent on a model of the MHTGR-350.
The first analysis studies the energy group structure effects on the simulation of a fuel column.
The results of the study suggest using a 15-energy group structure for attaining a desirable accuracy.
The following analysis focuses on the full-core problem and compares different aspects of the simulations, concluding that Moltres obtains reasonably accurate results.
The final study on stand-alone neutronics describes Moltres results of Phase I Exercise 1 of the OECD/NEA MHTGR-350 Benchmark.
The benchmark exercise proved to be a modeling challenge, requiring the implementation of several approximations.
For the most part, this thesis demonstrates Moltres' capability to simulate stand-alone neutronics of prismatic High-Temperature Gas-cooled Reactors.

% results: ch5
Regarding stand-alone thermal-fluids, several studies compare Moltres results to previously published results.
These studies focus on local models such as the unit cell and the fuel column problems, for which Moltres results differ by less than 2\% from the published results.
Further studies analyze the possibility of extending the thermal-fluids model implemented in the previous problems to a full-core simulation, finding a high memory requirement imposed by the simulations.
The full-core simulations focus on Phase I Exercise 2 of the benchmark, for which the implementation of a two-level approach in Moltres was necessary.
The study's results were within an 11.3\% difference to the published results, concluding that further analysis is required.

% results: ch5-coupled
Regarding coupled neutronics/thermal-fluids, the analysis describes Phase I Exercise 3 of the benchmark.
The exercise uses a simplified model that helps visualize some of the essential aspects of multi-physics simulations in Moltres.
This exercise finds some flaws in Moltres' model and sets a basis for future work.

% results: ch 6
This thesis aligns with the University of Illinois' goals to reduce carbon emissions from its campus's electricity generation and transportation sectors.
This work focuses on two main analysis by introducing a nuclear reactor coupled to a hydrogen plant as a solution.
The first analysis evaluates the conversion of the university fleet along with the mass transit transport system in Urbana-Champaign to Fuel Cell Electric Vehicles.
The second analysis investigates the duck curve phenomenon in the university's grid and introduces a mitigation strategy that may reduce the reliance on dispatchable sources.
These studies emphasize how nuclear energy and hydrogen production can potentially mitigate climate change.

\end{abstract}

\begin{dedication}
A mi familia, for their unconditional support.
\end{dedication}

\chapter*{Acknowledgments}

I wish to express my gratitude to my advisor Dr. Kathryn Huff, who guided me from my academic beginnings.
I would also like to thank Dr. Tomasz Kozlowski for his help and invaluable comments as the second reader of this thesis.
I wish to acknowledge Dr. Paul Fischer, Dr. Rizwan Uddin, and Dr. Caleb Brooks for their time and discussions helpful to this work.
% This work was possible thanks to the financial support of the Nuclear Regulatory Commission Faculty Development Program (award NRC-HQ-84-14-G-0054 Program B) and the UIUC Nuclear, Plasma, and Radiological Engineering Department.
% https://docs.google.com/spreadsheets/d/1VybV0oMPiqpInTwgO0ICq0EC4AL_fGj-zbXGJAhRiVU/edit#gid=0

Besides the support from several faculty, this work was possible due to day-to-day motivation from my fellow group-mates Sam Dotson, Greg Westphal, Kip Kleimenhagen, Andrei Rykhlevskii, Gwen Chee, and Sun Myung Park.
I also much appreciate Gavin Davis' help with proofreading my writing.
Finally, I want to acknowledge my family, who, even though we were more than five thousand miles apart, their support and encouragement were always present.
I also wanted to express my gratitude to Daniela Calero for her moral support throughout the pandemic.


%% The thesis format requires the Table of Contents to come
%% before any other major sections, all of these sections after
%% the Table of Contents must be listed therein (i.e., use \chapter,
%% not \chapter*).  Common sections to have between the Table of
%% Contents and the main text are:
%%
%% List of Tables
%% List of Figures
%% List Symbols and/or Abbreviations
%% etc.

\tableofcontents
\listoftables
\listoffigures

%% Create a List of Abbreviations. The left column
%% is 1 inch wide and left-justified
%\chapter{List of Abbreviations}
%\printglossaries
%% Create a List of Symbols. The left column
%% is 0.7 inch wide and centered

\pagebreak
\mainmatter

% different chapters
\chapter{Introduction}
\section{Prismatic Gas-Cooled Reactors}

This section will talk about different aspects of prismatic HTGRs:

\begin{itemize}
	\item history
	\item features
	\item co-generation possibilities (highlight hydrogen production)
\end{itemize}

\section{Motivation}

This section will express the motivation behind this work.

\section{Objectives}

This section will summarize the objectives of this work.

\chapter{Literature Review}
\section{Prismatic HTGR Diffusion Solvers}

Nowadays, several codes solve the neutronics of prismatic \glspl{HTGR}.
Most of these codes rely on one of the following methods: stochastic transport (Monte Carlo), deterministic transport, or deterministic diffusion.
We focus our interest on the last method.
% If time permits work on a brief description of the Monte Carlo and deterministic transport maybe.
% Why? Deterministic diffusion solvers have lower computational requirements than other methods reference ??
% The utilization of the Monte Carlo codes is unattractive because of the tremendous problem size and the need for a large number of neutron histories \cite{lee_status_2006}.
% It is one of the simplest means to solve neutron transport problems \cite{leppanen_development_2007}.
% Here I say the following
% Deterministic diffusion methods are computationally cheaper than the other methods.
% This characteristic makes it a good candidate for coupled calculations.

The history of deterministic diffusion solvers begins in the late 1950s with the \gls{FDM} application to the analysis of \glspl{LWR}.
In \gls{FDM}, mesh spacings are usually of the order of the diffusion length.
While solving large multi-dimensional problems, this feature causes the mesh points to reach intractable numbers \cite{lewis_finite_1986}.
The computational expense of these calculations motivated the generation of more computationally efficient techniques \cite{lawrence_progress_1986}.
Although substantial overlaps exist, the most common techniques fall into two broad categories: nodal methods and \gls{FEM}.

% NODAL
FLARE \cite{delp_flare_1964} is a three-dimensional \gls{BWR} simulator, and it is representative of the first generation of nodal methods.
Such an approach used adjusted parameters to match actual operating data or the results of more accurate calculations.
Most of these methods were implementations of the so-called "1.5 group theory" \cite{gupta_nodal_1981}.
The second generation of nodal methods derived spatial coupling relationships by applying the \gls{TIP}.
Such a procedure obtains equivalent one-dimensional equations by integrating the multi-dimensional diffusion equation over directions transverse to each coordinate axis \cite{lawrence_progress_1986}.
This approach proved to be highly efficient and accurate in Cartesian geometries.

In 1981, a formulation based on the \gls{NEM} first demonstrated the feasibility of nodal methods in hexagonal geometries \cite{duracz_nodal_1981}.
However, this method would introduce non-physical singular terms that required the utilization of discontinuous polynomials.
This drawback motivated the development of more effective formulations.
HEXNOD, introduced in 1988 by Wagner \cite{wagner_three-dimensional_1989}, is an example of such formulations.
This algorithm uses the \gls{TIP} and, in contrast to the \gls{NEM}, solves the resulting differential equation analytically.
Wagner's article demonstrated the method's good accuracy by comparing to \gls{FDM} and Monte Carlo calculations for a few benchmark problems.

Another example of more effective methods is HEXPEDITE \cite{fitzpatrick_hexpedite_1992}.
HEXPEDITE uses the \gls{TIP} formulation to derive a pseudo-one-dimensional equation.
The resulting differential equation is solved analytically.
The difference from HEXNOD is that HEXPEDITE uses a simpler and more efficient coupling scheme.
Different works \cite{fitzpatrick_hexpedite_1992}\cite{fitzpatrick_developments_1995} on the HEXPEDITE methodology tested the approach against the \gls{NEM} and the \gls{FDM}.
These studies established HEXPEDITE’s superiority in terms of accuracy and runtime.
HEXPEDITE's use prevailed in the analysis of \glspl{HTGR} until recently.
In 2010, \gls{INL} conducted a study \cite{ortensi_deterministic_2010-1} in which they compared HEXPEDITE's results against several diffusion codes, as well as the Monte Carlo codes MCNP5 \cite{rsicc_computer_code_collection_mcnp5_2003} and Serpent \cite{leppanen_serpent_2015}.
% In 2010, \gls{INL} conducted a study \cite{ortensi_deterministic_2010-1} in which they compared HEXPEDITE's results against the diffusion codes JAR, CITATION, and CRONOS2, as well as the Monte Carlo codes MCNP5 and Serpent.

DIF3D \cite{lawrence_dif3d_1983} and PARCS \cite{downar_parcs_2004} are other examples of nodal diffusion codes whose use has prevailed until the present.
DIF3D has several solution options such as the diffusion \gls{FDM}, diffusion \gls{NEM} based on \gls{TIP}, and the VARIANT nodal transport method.
% VARIANT: variational nodal \cite{palmiotti_variant_1995}
PARCS has several solution options as well, such as a diffusion \gls{FDM}, diffusion \gls{NEM} based on \gls{TIP}, P$_{N}$ transport methods, and the multigroup transport simplified P$_3$ with \gls{FDM} and \gls{NEM} discretizations.

% from ortensi_deterministic_2010-1 and wang_modified_2018
Nodal methods solve relatively coarse meshes for approximate solutions.
This characteristic makes the process efficient.
On the other hand, the method does not provide detailed point-wise accurate solutions \cite{kang_finite_1973}.
Additionally, the derivation of nodal methods happens in a specific coordinate system for a particular node shape.
The application to complex problems is not flexible as different geometries require the integration over other coordinate systems.
This lack of flexibility limits the applications of nodal methods to regular geometries only.

% FEM
The \gls{FEM} is a well-established method in applied mathematics and engineering.
\gls{FEM} is a numerical technique for finding approximate solutions to partial differential equations by deriving their weak or variational form.
Most applications make \gls{FEM} preferable due to its flexibility in the treatment of curved or irregular geometries.
Also, the use of high order elements attains higher rates of convergence \cite{cavdar_finite_2004}.
The first engineering application of \gls{FEM} was in the field of structural engineering dating back to 1956.
In successive years, \gls{FEM} became the most extensively used technique in almost every branch of engineering.
\glspl{FEM} have several advantages over the nodal methods.
It provides flexibility in the geometry definition, a firm mathematical basis, ease in extension to the multi-group application, and high computational efficiency \cite{lee_development_2008}.

In 1973, Kang et al. \cite{kang_finite_1973} described the first application of \gls{FEM} to the neutron diffusion theory.
The fundamental motivation for this development was the impractical application of the \gls{FDM} to three-dimensional problems.
In this early work, the author compared different \gls{FEM} approaches to the \gls{FDM} in one-dimensional and two-dimensional problems.
The studies showed a higher order of convergence achieved by the \gls{FEM}.

Throughout the last four decades, many codes utilized the \gls{FEM} to solve the diffusion equation.
Some of the most recent software for diffusion simulations are CRONOS2 \cite{lautard_cronos_1990}, CAPP \cite{lee_development_2011}, and Rattlesnake \cite{wang_rattlesnake_2019}.
The list of \gls{FEM} diffusion solvers is more extensive, but we focus on the best-documented software in the open literature.
% We also emphasize that most of the \gls{FEM} diffusion solvers for \glspl{HTGR} were born as \gls{LWR} analysis codes.

% CRONOS
\gls{CEA} developed CRONOS2 \cite{lautard_cronos_1990} as part of the SAPHYR system.
It solves steady-state and transient multi-group calculations, taking into account thermal-fluids feedback effects.
CRONOS2 solves either the diffusion equation or the transport equation through the S$_N$ method.
In 2008, Damian et al. \cite{damian_vhtr_2008} conducted a study aimed at understanding the physical aspects of the annular core and the passive safety features of a standard block type \gls{HTGR}.
For such a study, the authors developed the code suit NEPTHIS/CAST3M which relied on CRONOS2.

% CAPP
In 2008, the \gls{KAERI} published an article \cite{lee_development_2008} that presented CAPP.
Its purposes are to conduct steady-state core physics analysis, core depletion analysis, and core transient analysis.
The article validated the code with two benchmark problems: the IAEA PWR benchmark problem, and Phase I Exercise 1 of the OECD/NEA PBMR-400 Benchmark \cite{reitsma_oecd-neansc_2008}.
The calculations of both problems changed the number of finite elements and the orders of shape functions.

In 2011, Lee et al. published an article \cite{lee_development_2011} in which they extended the functionalities of CAPP to prismatic \glspl{HTGR}.
To take into account the thermal feedback, the authors developed a simplified thermal-fluids analysis tool.
To validate their model, the authors solved a two-dimensional model of the PMR-200 at the beginning of the equilibrium cycle.
The PMR-200 is a pre-conceptual reactor designed by \gls{KAERI}.
Their validation compared the results against HELIOS\cite{stammler_helios_1998}, concluding with a strong accuracy.
Moreover, the authors implemented a depletion solver based on the one-group flux determined by the diffusion solver.
The authors validated the depletion solver by calculating the multiplication factor as a burnup function of a single fuel block of the PMR-200.
They compared the results against HELIOS.
The maximum error was less than 200 pcm.

Tak et al. \cite{tak_cappgamma_2016} developed a coupling between CAPP and GAMMA+ \cite{lim_gamma_2006}.
GAMMA+ is a system code for thermal-fluids analysis and system transients.
In such a study, they studied the steady-state performance of the PMR-200.
The authors conducted several studies, such as a core depletion calculation with and without a critical control rod position search and the analysis of the bypass flow effects on the coupled calculations.
Their results revealed that neglecting the bypass flow decreases the active core temperatures; consequently, the multiplication factor increases by approximately 300 pcm.

A recent article by Yuk et al. \cite{yuk_time-dependent_2020} added to CAPP the capability to conduct transient analyses.
This capability solves the time-dependent neutron diffusion equation with the \gls{FEM}.
The primary motivation behind this feature was to perform reactivity insertion accidents.
Additionally, the article introduced a new method to resolve the control rod cusping effect \cite{joo_resolution_1984}.
% To take into account the thermal feedback, the authors developed a simplified thermal-fluids analysis tool.
The new method integrates over partially rodded computation nodes, and the article referred to it as iPRN.
To test its accuracy, the authors conducted two exercises with several techniques that reduce the rod cusping effect.
The authors used the mesh reconstruction method to obtain the reference results, as such a method eliminates the rod cusping effect by updating the mesh at every time step.
The iPRN technique showed higher accuracy than the other methods.
To test the new transient capabilities, they analyzed two control rod ejection scenarios and compared the results to those of the CAPP/GAMMA+ coupled code.
Both showed similar results.

% Proghorn and Rattlesnake
RattleSnake \cite{wang_rattlesnake_2019} is the MOOSE \cite{gaston_moose_2009} based application for simulating the transport equation.
\gls{INL} had initially developed Pronghorn \cite{strydom_inl_2013} to model \glspl{PBMR}.
The MOOSE neutronics kernel library Yak incorporated the neutron diffusion models initially in Pronghorn.
Currently, RattleSnake is the primary tool for solving the linearized Boltzmann neutron transport equation within MOOSE and relies heavily on Yak's use.
Various solvers are available under RattleSnake, including low-order multigroup diffusion, spherical harmonics transport, and discrete ordinates transport, all solved with the \gls{FEM}.
% Both RattleSnake and Pronghorn yielded the same exact results when using the continuous \gls{FEM} multigroup diffusion option in RattleSnake.

% j_ortensi_relap-7_2012 j_ortensi_initial_2012
In 2012, \gls{INL} published a study \cite{j_ortensi_initial_2012} that coupled Pronghorn and RELAP-7 \cite{andrs_relap-7_2012}.
Pronghorn solved the coupled equations defining the neutron diffusion, fluid flow, and heat transfer in a three-dimensional model.
RELAP-7 is a MOOSE-based system code and solves the one-dimensional continuity, momentum, and energy equations for a compressible fluid.
It was responsible for simulating the plant system layout, including the hot and cold ducts, the helium circulator, and the steam generator.
% This study integrated PRONGHORN and RELAP-7 with the operator split approach (or loose coupling) where each application used an independent mesh.
To test the coupling, INL's team carried out the OECD/NEA MHTGR-350 Benchmark \cite{oecd_nea_coupled_2020}.
The original benchmark provides a set of 26 neutron energy group and temperature dependent cross sections.
To simplify the debugging, the authors collapsed the 26 groups into two groups.
Although using two groups reduces the accuracy of the model, the lower number of groups decreases the calculation time by at least a factor of ten.
In this study, a two-dimensional cylindrical model replaced the three-dimensional geometry defined by the benchmark.
The integrated system testing included two stages: (1) both stand-alone codes underwent several convergence studies, and (2) the integrated system solved the steady-state problem in an integrated manner.
The authors concluded that the coupling between Pronghorn and RELAP-7 was successful.

% strydom_inl_2013
In 2013, \gls{INL} conducted the OECD/NEA MHTGR-350 Benchmark \cite{strydom_inl_2013} without further simplifications.
The \gls{INL} team solved Phase I Exercise 1 using INSTANT-P1 \cite{wang_krylov_2011}, Pronghorn, and RattleSnake.
% They also solved exercises 2 and 3 using RELAP5-3D and PHISICS/RELAP5-3D code suit.
INSTANT-P1 is a transport solver that relies on the spherical harmonics discretization of angles.
The results for Pronghorn and RattleSnake were identical.
By modifying the cross-sections, INSTANT-P1 returned the diffusion solution.
Its results were within 30 pcm from Pronghorn and RattleSnake results.
All presented results exhibited good agreement with the benchmark results.

\subsection{Energy group structure analysis}

% yuk_time-dependent_2020
% The authors recommend homogenizing the group constants using at least 10 energy groups.

% Number of energy groups impact over the calculations
The longer neutron mean free path in \glspl{HTGR} compared to \gls{LWR} increases the spectral interactions between elements.
For this reason, \gls{HTGR} analyses require more energy groups than conventional \gls{LWR} analyses.
\gls{ANL} directed a study \cite{lee_status_2006} to compare the accuracy of nodal diffusion calculations employing different energy group structures to generate the homogenized cross-sections.
The cross-section homogenization used the code DRAGON and the diffusion calculations utilized the code DIF3D.
For the study, the ANL team implemented a one-dimensional fuel-reflector model in which they used 4, 7, 8, 14, and 23 energy groups.
They also used alternative energy group structures for the same number of groups.
For simplicity, the authors used the homogenized fuel compact model and generated all the cross-sections at 300 K.
One of their conclusions was that the number of energy groups should be more than 4, and more than 6 would be sufficient for uranium fueled \glspl{HTGR}.
Another finding was that the accuracy of the diffusion calculation is sensitive to the energy group boundaries.
% Mention something about the metrics of the study? To asses the accuracy, they compared the multiplication factor.

% han_sensitivity_2008
Han's MS thesis \cite{han_sensitivity_2008} focused on selecting energy groups for the reactor analysis of the \gls{PBMR}.
The author used COMBINE6 \cite{grimesey_combinepc-portable_1994} for cross-section generation and the Penn State nodal diffusion code NEM \cite{bandini_three-dimensional_1990} for the reactor analysis.
The author compared the results against MCNP5 reference results.
To simplify the setup, the model used uniformly distributed isotopes in the fuel.
The study performed the calculations at 300 and 1000 K.
To arrive at an optimal group structure, the author compared many combinations of group structures using a trial and error strategy.
One conclusion of this work agrees with the previous bibliography \cite{gulf_oil_company_nuclear_1973} \cite{duderstadt_nuclear_1976} in that the energy spectrum is critical to yield an accurate description of a nuclear reactor using a few groups.

ANL's study helps set up proper nodal diffusion calculations for an \gls{HTGR}.
Although we can extrapolate those conclusions to \gls{FEM} diffusion solvers, such a study might be valuable.
ANL's team conducted the study at 300 K - not in the operational range of any \glspl{HTGR}.
On the contrary, Han's thesis included an analysis at 1000 K, and his results showed that the temperature changes have a non-negligible impact.
Additionally, ANL's study used the simplified model of the homogenized fuel compact.
Han highlighted that homogenized fuel models of the \gls{PBMR} underestimate criticality calculations.
In 2015, \gls{INL} presented their results \cite{strydom_results_2015} for an \gls{IAEA} \gls{CRP} \cite{tyobeka_htgr_2011} and showed that the homogenization of the compact material notably underestimates the multiplication factor.
On the other hand, the open literature has not investigated the impact of such simplification over the homogenized cross-sections.

% left here in the review
\section{Prismatic HTGR Thermal-fluids}

% sort of motivation
Thermal-hydraulic calculations enable the correct design of \glspl{HTGR}.
Predicting the maximum fuel temperature at a steady-state is of paramount importance to succeed in such a task.
We emphasize this statement in the case that hydrogen production is desirable.
Efficient hydrogen production requires higher coolant temperatures, which increases the temperature of the fuel and the reactor pressure vessel.

% sort of intro to simplified models
The complex geometry of the hexagonal fuel assembly requires elaborate numerical calculations for obtaining accurate evaluations.
Thermal-hydraulic studies for early \glspl{HTGR} consisted mainly of support calculations for \gls{NRC} safety analysis reports.
The analyses employed sets of independent codes that relied on simplistic approximations.
Simplified models help understand some fundamental aspects of prismatic \glspl{HTGR} and have the advantage of reducing the computational expense of the calculations.

% shenoy_htgr_1974
General Atomics \cite{shenoy_htgr_1974} developed the first set of simplistic codes.
The following list introduces and summarizes some of these and their features:

\begin{itemize}
\item FLAC: It determines the coolant flow distribution in the coolant channels and gaps.
It solves the one-dimensional momentum equation for incompressible flow and the continuity equations for mass and energy.

\item POKE: It determines the coolant mass flow, coolant temperature, and fuel temperature distribution.
It solves the steady-state mass and momentum conservation equations for parallel channels.

\item DEMISE: It determines the steady-state three-dimensional temperature distribution in a standard element.
It solves the temperature in a network model.

\item TAC2D: It is a general-purpose two-dimensional thermal analysis code.
It solves the two-dimensional heat conduction equation.
\end{itemize}

Several studies have used these codes.
% macdonald_ngnp_2003
For example, \gls{INL} conducted in 2003 a design study \cite{macdonald_ngnp_2003} in support of the \gls{NGNP} project.
Such a study aimed to investigate options for the NGNP that increase the coolant temperature with the lowest possible inlet temperature and the highest overall core power.
The authors conducted several parametric studies whose reference reactor was the GT-MHR \cite{general_atomics_gas_1996}.
Using POKE, they evaluated two major design modifications: reducing the bypass flow and better controlling the inlet coolant flow distribution.
Reducing the bypass flow fraction from 20 to 10$\%$ reduces the peak fuel temperatures by about 50$^{\circ}$C.
Controlling the inlet flow distribution has a stronger effect.
Other studies focused on the dimensions of the reactor and their impact on the maximum fuel temperature.
Using POKE and TAC2D, the authors investigated taller and higher power reactor cores.
The investigation included a 10-block-high 600 MWt, a 12-block-high 720 MWt, and 14-block-high 840 MWt.

Among the simplified approaches, we differentiate the flow network, equivalent cylindrical model, and unit cell model.
% flow network model
% reza_design_2006
Using the network analysis tool RELAP5-3D/ATHENA \cite{inl_relap5-3dathena_2005}, Reza et al. \cite{reza_design_2006} conducted a thermal-hydraulic study of the GT-MHR.
Reza et al. increased the reactor outlet temperature to enable hydrogen production.
Additionally, they evaluated alternative coolant inflow paths in an attempt to reduce the reactor vessel temperatures.
After finding an optimal configuration, they evaluated the fuel and the reactor vessel's maximum temperatures during the \gls{LPCC} and the \gls{HPCC} events.

% equivalent cylindrical model
% no_multi-component_2007
An example of codes using the equivalent cylindrical approach is GAMMA \cite{no_multi-component_2007}.
Its main objective is simlating the air ingress event following a LOCA.
Following the depressurization of helium in the core, there exists the potential for air to enter the core through the break and oxidize the in-core graphite structure.
The oxidation of graphite leads to exothermic chemical reactions and, thus, it is a significant concern.
GAMMA solves heat conduction, fluid flow, chemical reactions, and multi-component molecular diffusion.
The code couples the solid and gas equations using the porous media model.
Together with the multi-dimensional analysis feature, GAMMA has a one-dimensional analysis capability for  modeling a flow network.

% takada_core_2004
Takada et al. \cite{takada_core_2004} carried out another study using the flow network and the equivalent cylindrical model.
Focusing on the \gls{HTTR}, they developed a thermal-hydraulic design code.
This used the flow network analysis code FLOWNET \cite{maruyama_verification_1988} for calculating the coolant flow and temperature distributions.
TEMDIM \cite{maruyama_verification_1988} solved the fuel temperatures using the equivalent cylindrical model.
Finally, the authors validated the calculation scheme by comparing its results with the experimental data from the \gls{HTTR}.

% unit cell model
% nakano_conceptual_2008
Nakano et al. \cite{nakano_conceptual_2008} studied different fuel assembly configurations using several simplistic approximations.
For determining the fuel temperature, they used the TAC2D.
A previous nuclear analysis calculated the power density.
Moreover, a previous study calculated the flow distribution using FLOWNET.
The fuel temperature calculation used the equivalent cylindrical model for a hot channel unit cell.
However, the asymmetry of the unit cell configuration makes the temperature distribution asymmetric in the graphite block.
The equivalent cylindrical model fails to capture this behavior.

% in_three-dimensional_2006
In 2006, In et al. \cite{in_three-dimensional_2006} conducted a more detailed analysis using a three-dimensional model of the unit cell in the hot-spot of an \gls{HTGR}.
The objective of the study was to predict the maximum fuel temperature at steady-state.
The analysis focused on the GT-MHR 600 at the end of the equilibrium cycle.
The CFD code CFX 10 \cite{ansys_inc_cfx_2006} calculated the three-dimensional temperature profile.
In such a study, the results showed that the maximum fuel temperature surpassed the design limits, so the authors proposed a countermeasure accordingly.

% more detailed calculations
Such simplified approaches are helpful to understand essential aspects of prismatic \glspl{HTGR} but they may affect the temperature distribution.
More detailed thermal-hydraulic evaluations were rare in the open literature until the last 15 years.

% cioni_3d_2005
Cioni et al. \cite{cioni_3d_2005} presented an article in 2005 in which they conducted three-dimensional simulations of fuel assemblies of an \gls{HTGR}.
The study's objective was to investigate an emergency situation due to the blocking of cooling channels in the core.
They used the \gls{CFD} code Trio\_U \cite{bieder_priceles_2000} to carry out the analysis.
The numerical scheme solved the three-dimensional conduction equation in the solid coupled to the coolant's one-dimensional thermal-hydraulic equations.
In the preliminary work, the authors conducted a study of the bypass flow's influence on the maximum coolant and fuel temperature.
Another preliminary study analyzed the consequences of two different blocking in a portion of a fuel assembly.
A central blockage exhibits a stronger influence over the assembly's maximum temperatures compared to a peripheral blockage.
Last, they investigated two configurations.
First, six fuel elements surrounded the fuel element with the blockage.
Second, five fuel elements and one reflector element surrounded the fuel element with the blockage.
The results suggest that the blockage increases the temperature on the blocked fuel assembly only, and it does not affect the surrounding elements due to the bypass flow.
The results also show that the fuel temperature surpassed the design limits and that the reactor operators should counteract these effects with active systems.

% simoneau_three-dimensional_2007
Simoneau et al. \cite{simoneau_three-dimensional_2007} analyzed the transient behavior of an \gls{HTGR} during the \gls{DCC} and \gls{HPCC} event.
The CFD code STAR-CD \cite{computational_dynamics_limited_star-cd_2004} performed the calculations.
It solved conductive, convective, and radiation heat transfer in a 30$^{\circ}$ section of the core and reactor vessel.
It uses the porous media model to accommodate the different spatial scales.
The model does not resolve the boundary layer, and the use of coefficients prescribe the solid-fluid heat transfer and pressure drop across the core.
The authors validated their model against explicit calculations using a single fuel block.
One of their results shows that the maximum temperature in the \gls{HPCC} event is lower than in the \gls{DCC} event.
However, the extra convective heat transfer causes a thermal stratification in the surrounding air, causing higher temperatures in the upper reactor structures.

% tak_numerical_2008
In 2008, an article by Tak et al. \cite{tak_numerical_2008} conducted  a three-dimensional \gls{CFD} analysis on a typical prismatic \gls{HTGR} fuel column.
The commercial code CFX 11 \cite{ansys_incorporated_cfx_2006} performed the calculations.
The fuel column under study was from the PMR-600, a pre-conceptual reactor designed by \gls{KAERI} whose reference design is the GT-MHR.
The study considered a one-twelfth section of the fuel due to its symmetry.
The model determined the coolant distribution using the one-dimensional thermal-hydraulic equations.
Such coolant distribution served as input to the CFD code.
The friction in the channels is dependent on the viscosity, which is highly dependent on the temperature.
Therefore, obtaining the mass flow rates from a separate solver may introduce errors \cite{sato_computational_2010}.
As mentioned earlier, the unit cell model may introduce errors in the maximum temperature prediction.
To assess the accuracy of the unit cell model, the authors compared the CFD results against the unit cell model results.
The unit cell model does not consider the bypass flow between assemblies or the radial power distribution within the fuel assembly.
Tak et al. conducted a parametric study that analyzed the bypass gap size's impact on the maximum fuel temperature.
By increasing the bypass gap, the maximum fuel temperature grows.
The results of this study indicate that the accuracy of the unit cell worsens for larger gaps.
Another study imposed different radial peaking factors for the different fuel channels.
Such a study showed the effects of considering a non-flat radial power distribution.
The authors considered a radial power distribution that did not strongly impact on the maximum fuel temperature.

% sato_computational_2010
Another article \cite{sato_computational_2010} carried out \gls{CFD} calculations of a typical prismatic \gls{HTGR} with the commercial code FLUENT \cite{fluent_inc_fluent_2006}.
Their model considered a one-twelfth section of the fuel column of the GT-MHR.
The authors conducted parametric studies changing several factors, such as bypass gap-width, turbulence model, axial heat generation profile, and geometry changes due to irradiation.
Their most relevant results show that the bypass flow causes a large lateral temperature gradient in the block.
Large temperature gradients cause excessive thermal stresses, which raise potential structural issues.
The authors compared the results from different turbulence models: $k \sim \varepsilon$ and $k \sim \omega$.
The $k \sim \omega$ model predicted bulk temperatures that are considerably lower than those from the $k \sim \varepsilon$ model.
The maximum temperature difference was 49$^{\circ}$C.
The overall mass flow rate was about 10$\%$ greater for the $k \sim \omega$ model.
The study suggested that these turbulence models need more verification against prismatic \gls{HTGR} experiments.
Another study analyzed the effect of considering different peak radial factors.
This consideration introduced variations of the maximum fuel temperature of up to 160 $^{\circ}$C.
Their last study focused on the effects of the graphite dimensional changes on the temperature profile.
The shrunken column showed considerably lower temperatures in the fuel.

% travis_thermalhydraulics_2013
Despite the recent developments in CFD tools, a detailed full-core analysis for a prismatic \gls{HTGR} still requires a tremendous computational expense.
This requirement is mostly due to the three-dimensional CFD simulation of the coolant flow.
Travis et al. \cite{travis_thermalhydraulics_2013} developed a method to compute full core thermal-hydraulic analyses of \glspl{HTGR}.
The article presented a simplified method that reduces the computational time and memory requirements while maintaining accurate results.
The method solves the three-dimensional heat conduction in the solid and the one-dimensional thermal-hydraulic equations in the channels.
The fluid one-dimensional approximation avoids finer meshes near the walls as well as turbulence conservation equations \cite{tak_development_2014}.
The method's validation analyzed a fuel column and compared the results to those of a three-dimensional CFD simulation.
The CFD simulation used the commercial software STAR-CCM+ \cite{cd-adapco_star-ccm_2012}.
The new computational scheme reduced the computation time to 2.5\% of the time required by the three-dimensional CFD simulation.
The new method provided good predictions of the temperature distribution and the axial variation of the helium bulk temperature.
However, it failed to resolve the velocity and temperature distribution within the boundary layer properly.
Overall, the method showed good accuracy and less than a 2\% difference to the three-dimensional CFD simulation.

% tak_practical_2012 / tak_development_2014
Tak et al. \cite{tak_practical_2012} \cite{tak_development_2014} developed CORONA, which uses a practical method for the whole core analysis.
CORONA intends to combine the accuracy from CFD tools and the light computational expense of system analysis codes.
The method solves the three-dimensional heat conduction equation in the solid and the one-dimensional thermal-hydraulic equations in the fluid.
To enhance practicability, the code adopts a basic unit cell concept, which eliminates an elaborate grid generation process.
The basic unit cell concept is an extension of the traditional unit cell method, which uses a single triangular unit cell.
This method considers various shapes of unit cells as well as the heat transfer between them.
The method provides a way for fast generating computational grids for modeling the solid regions.
To validate CORONA, the authors compared its results against CFX and experimental results.
The results of the validation showed that the CORONA code provides reasonably accurate results.

% end of section
CFD techniques allow the detailed temperature profile over local models to be computed.
The fine mesh requirement imposes high computational costs for a whole-core CFD analysis, restricting such methods.
However, a whole-core thermal analysis has many advantages over local models.
In general, the problem set up includes more accurate boundary conditions.
Without whole-core modeling, the local models' mass flow distributions are average values of the core flow rate instead of their exact value \cite{huning_novel_2016}.
This simplification leads to under-predicted fuel temperatures for the assemblies with a lower flow rate than the average.
Additionally, a coupled analysis with a reactor physics code requires a full core model \cite{tak_practical_2012}.

% this might appear in a summary of the chapter: see travis_thermalhydraulics_2013 / tak_practical_2012 / tak_development_2014 
% The fine mesh requirement for a whole core CFD analysis restricts the use of such methods.
% Most of the CFD studies limit their application to the local behavior of a fuel column.
% The alternative for an explicit whole-core analysis are system codes that use simplified models.
% However, the simplification of the geometries reduces the fuel block temperature resolution.
% The present method intends to overcome the difficulties in CFD calculation as well as in system calculations.
% The computational expense of a solid heat conduction equation is much less than that of fluid conservation equations in CFD analyses .
% This requirement is mostly due to the three-dimensional CFD simulation of the coolant flow.
% The methodology provides good predictions of the global parameters such as the three-dimensional spatial temperature distribution and the axial variation of the helium bulk temperature \cite{travis_thermalhydraulics_2013}.
% The fluid one-dimensional approximation avoids finer meshes near the walls as well as turbulence conservation equations \cite{tak_development_2014}.

\section{Prismatic HTGR Multi-physics}

% sort of intro
Historically, stand-alone simulations have solved the neutronics and thermal-fluids of \glspl{HTGR}.
Nonetheless, these physical aspects describe processes that rely heavily on one another.
Hence, a coupled analysis is necessary to consider the interaction between the neutronics and thermal-fluids behavior \cite{tak_cappgamma_2016}.

% damian_vhtr_2008
In 2008, Damian et al. \cite{damian_vhtr_2008} conducted a study aimed to understand the physical aspects of the annular core and the passive safety features of a prismatic \gls{HTGR}.
They performed analyses on various geometrical scales, including: unit cell and fuel columns located at the core hot-spot and two-dimensional and three-dimensional core configurations, including the coupling between neutronics and thermal-fluids.
The first part of the assessment concerns thermal calculations on steady-state core configurations.
Such a study used CAST3M \cite{studer_cast3marcturus_2007} to solve the three-dimensional heat conduction in the solid coupled with the one-dimensional thermal-fluids equations in the coolant.
% The authors conducted a parametric study on the maximum fuel temperature by modifying the fuel compact and the fuel element geometries.
% An annular fuel compact and the reduction of the number of fuel compacts in the outer ring yielded the best performance.
The second part of the assessment used the transport code APOLLO2 \cite{sanchez_apollo2_1999} on a two-dimensional core configuration to minimize the radial power peaking factor.
The analysis included the variation of several parameters, such as fuel enrichment, fuel loading, and the fuel management scheme.
The fuel enrichment variation had the most substantial impact.
The last part of the study analyzed a three-dimensional core model using the coupled codes NEPTHIS \cite{cavalier_presentation_2005} and CAST3M/Arcturus.
NEPTHIS and CAST3M/Arcturus calculate the neutronics and the thermal-fluids, respectively.
NEPHTIS uses a transport-diffusion calculation scheme that relies on APOLLO2 and the diffusion code CRONOS2.
CRONOS2 solves either the diffusion equation or the even parity transport equation using an \gls{FDM} or a \gls{FEM} discretization.
The CAST3M/Arcturus model uses a two-level approach.
On the first level, the porous media model solves the homogenized system and the coolant.
On the second level, CAST3M solves the thermal-fluids on the homogenized geometry.
The authors conducted several parametric studies and assessed their impact on the power distribution.
The studies included the variation of the helium bypass fraction, average power density, core geometry, reflector materials, and fuel loading strategy.
One of their results exhibited that with the reduction of the bypass fraction, the average reflector temperature rises.
Another result showed that using magnesium oxide as the reflector material yields lower temperatures for normal operation and transients.

% CAPP
In 2011, Lee et al. published an article \cite{lee_development_2011} in which they extended the functionalities of CAPP to prismatic \glspl{HTGR}.
To take into account the thermal feedback, the authors integrated into CAPP a simplified thermal-fluids tool.
This tool divides a fuel column into six triangular prisms.
Each of them hosts a representative coolant channel.
The code calculates the axial coolant temperature distribution solving the energy equation.
After calculating the coolant temperature, a two-dimensional conduction model solves the moderator and fuel compact temperatures.
Through a TRISO particle conduction model, the model obtains the fuel temperature.
Finally, a three-dimensional conduction model based on the \gls{FDM} allows for solving the reflector temperature.
% The cross-sections are a function of burnup, moderator temperature, and fuel temperature.
To validate this model, the authors solved a two-dimensional model of the PMR-200 at beginning of equilibrium cycle.
They compared the results against the HELIOS reference results.
The results showed good accuracy.

% tak_coupled_2016
Tak et al. \cite{tak_coupled_2016} developed a neutronics/thermal-fluids coupled code using DeCART \cite{kaeri_decart_2007} and CORONA.
DeCART is a whole-core neutron transport code, and it was responsible for calculating the power distribution and the fast neutron fluence.
CORONA calculated the temperature distribution.
To validate the new code, the authors conducted the OECD/NEA MHTGR-350 benchmark.
The exercise's main objective was to validate the code and identify technical challenges for future development.
The authors presented an interesting analysis in which they compared the coupled simulation results and the stand-alone simulations.
The difference in the multiplication factor was as high as 2597 pcm.
The axial offset and maximum fuel temperature exhibited significant differences as well.
Such a result highlights the importance of the integration of both neutronics and thermal-hydraulic solvers.

% tak_cappgamma_2016
Another article \cite{tak_cappgamma_2016} introduced a coupling between CAPP and GAMMA+.
GAMMA+ is a system code for thermo-fluids analysis and system transients.
GAMMA+'s primary motivation was to analyze the air ingress accident and thermo-fluid transients in \glspl{HTGR}.
It uses the one-dimensional form of the mass, momentum, energy, and species conservation equations to solve the fluid's flow and temperature distribution.
For solids, it uses three different models: (1) heat conduction model of a TRISO particle, (2) implicit coupling to consider the heat exchange between a fuel compact and TRISO particle, and (3) multi-dimensional heat conduction model of the hexagonal fuel and reflector blocks.
In such a study, the authors applied the coupled code to study the steady-state performance of the PMR-200.
They analyzed the bypass flow effects on the coupled calculations.
Some of their most relevant results showed that the maximum fuel temperature reaches a peak near the middle of the equilibrium cycle.
Another result revealed that neglecting the bypass flow decreases the active core temperatures and increases the reflector temperatures.
Consequently, the multiplication factor increased by approximately 300 pcm.
On the other hand, the power density changes were not appreciable.

% yuk_time-dependent_2020
A recent article by Yuk et al. \cite{yuk_time-dependent_2020} added the capability to conduct transient analyses to CAPP.
To take into account the thermal feedback, the authors developed a simplified thermal-fluids analysis tool.
The tool divides a fuel column into six triangular prisms.
Each of them hosts a representative coolant channel.
After calculating the coolant temperature, a two-dimensional conduction model solves the moderator and fuel compact temperatures.
CAPP code uses predetermined tables of thermal conductivity for each material.
For a given fast neutron fluence and temperature, it obtains the thermal conductivity by interpolation.
To test the new transient capabilities, they analyzed two control rod ejection scenarios.
They compared the results to those from the coupled CAPP/GAMMA+ code.
Both methods showed similar results.

% Benchmarks Intro
The prismatic \gls{HTGR} tools available have lagged behind state of the art compared to \glspl{LWR}.
This delay drives the development of more accurate and efficient tools to analyze the reactor behavior for design and safety evaluations.
In addition to the development of new methods, it is essential to define appropriate benchmarks to compare various tools' capabilities.
% oecd_nea_coupled_2020
In 2012, the \gls{OECD}/\gls{NEA} defined a benchmark for the \gls{MHTGR}-350 MW reactor \cite{oecd_nea_benchmark_2017}.
The purpose of this benchmarking exercise is to compare various coupled reactor physics and thermal-hydraulic analysis methods.
The MHTGR design serves as a basis for this benchmark.
The scope of the benchmark is twofold: (1) to establish a well-defined problem, based on a common given data set, to compare methods and tools in core simulation and thermal-hydraulic analysis, and (2) to test the depletion capabilities of various lattice physics codes available for prismatic \glspl{HTGR}.
The OECD/NEA MHTGR-350 MW benchmark subdivides the coupled system calculation into three phases.
Phase I corresponds to the stand-alone neutronics and thermal-fluids modeling, as well as the coupled neutronics/thermal-fluids steady-state modeling.
Phase II consists of transient cases.
Phase III focuses on lattice depletion calculations.

% tyobeka_htgr_2011 / strydom_results_2015
Sensitivity analysis and uncertainty analysis methods can assess the predictive capabilities of coupled neutronics/thermal-fluids simulations.
In 2013, the IAEA launched a \gls{CRP} \cite{tyobeka_htgr_2011} on the \gls{HTGR} Uncertainty Analysis in Modeling.
The \gls{CRP} objective was to determine the uncertainty in \gls{HTGR} calculations at all stages of coupled reactor physics/thermal-fluids and depletion calculations.
This \gls{CRP} is a natural continuation of the previous IAEA and OECD/NEA international activities \cite{iaea_evaluation_2003}\cite{reitsma_oecd-neansc_2008} on Verification and Validation of available \gls{HTGR} simulations capabilities.
The technical approach is to establish and utilize a benchmark for uncertainty analysis.
The benchmark defines a series of well-defined problems with complete sets of input specifications and reference experimental data.
The CRP adopted the MHTGR-350 as the reference design and the GT-MHR as a second reference design.
The design specification uses the OECD/NEA MHTGR-350 MW benchmark \cite{oecd_nea_benchmark_2017} code design specifications.
The CRP subdivides the coupled system calculation into three phases.
Phase I corresponds to the stand-alone neutronics and thermal-fluids modeling.
Phase II consists of design calculations, coupled with steady-state neutronics/thermal-fluids calculations with and without a depletion calculation.
Phase III focuses on safety calculations.

\chapter{Methodology}
\section{MOOSE}

This section will briefly talk about MOOSE features.

\section{Moltres}

This section will talk about Moltres features.

\section{Serpent}

This section will talk about Serpent features.

\section{MHTGR-350 Summary}

This section will describe the MHTGR350 main characteristics.



\chapter{Neutronics}
\label{ch:neutronics}

\section{Preliminary studies}

\subsection{Homogeneous vs. heterogeneous isotope distribution}

This section conducted a study to determine the proper way to treat the fuel compact heterogeneities in Serpent.
This study modeled two different isotope distributions in a fuel compact.
First, a homogeneous distribution of the isotopes.
Second, a heterogeneous distribution, in which we explicitly modeled the TRISO particles.
We modeled both cases using Serpent.
We used a hexagonal unit cell model that included the fuel compact, a helium gap, and the surrounding graphite.
Table \ref{tab:compact} specifies the model input parameters.
The material temperature was 1200K, a case that represents the \gls{HFP} core state.
Serpent ran 5$\times 10^4$ neutrons/cycle, 500 active cycles, and 50 inactive cycles for the calculations.
The simulations took 1.73 and 2.21 minutes using 256 cores.
The heterogeneous calculation took 28$\%$ more.

The \gls{Keff} was 1.17523 for the homogeneous distribution and 1.25106 for the heterogeneous distribution.
Using the heterogeneous distribution as a reference, we calculated the relative error of some of the group constants in an eigenvalue calculation.
Serpent generated the group constants using the three energy group structure in Table \ref{tab:energygroups}.
The evaluated parameters were $D_g$, $\Sigma^r_g$, $\nu\Sigma^f_g$, and $\chi^t_g$ (see equation \ref{eq:diffusion}).
Figure \ref{fig:param-comparison} displays the relative errors for $\Sigma^r_g$ (REMXS) and $\nu\Sigma^f_g$ (NSF), which were the group constants that changed the most.
The figure does not include $D_g$ and $\chi^t_g$ because their relative errors were less than 1$\%$.
The relative errors of $\Sigma^r_g$ and $\nu\Sigma^f_g$ were less than 6$\%$.

% \cite{strydom_results_2015}
% The low multiplication factor in the homogeneous case is due to the fuel self-shielding .
% In the heterogeneous compact, the fuel self-shielding leads to a significantly decreased U-238 neutron absorption compared to the homogeneous fuel compact.

\begin{figure}[htbp!]
	\centering
	\includegraphics[width=0.43\linewidth]{figures-neutronics/param-comparison}
	\hfill
	\caption{Relative error of the group constants generated with a homogeneous isotope distribution.}
	\label{fig:param-comparison}
\end{figure}

The results show that the homogenization of the fuel compact isotopes decreases the multiplication factor considerably.
The impact on the group constants does not seem to be substantial.
However, the multiplication factor's considerable difference suggests that the group constants’ small variations combined effect is significant.
Based on these results, Serpent models the TRISO particles explicitly in the following sections.


\subsection{Problem set-up}

% Diffusion calculations: homo vs hetero in LWRs
Diffusion calculations necessitate a spatial homogenization of the group constants.
Depending on the desired level of detail, the type of homogenization could vary.
For example, in PWR core calculations, the homogenization in space could be per assembly or pin-by-pin \cite{krebs_calculational_1990}.
In a per assembly homogenization, the diffusion code models a global neutronic representation of the assembly.
We will refer to diffusion calculations using this type of group constant homogenization scheme as homogeneous calculations.
A pin-by-pin homogenization makes possible the treatment of pin or assembly heterogeneities.
This type of homogenization yields a detailed neutronic representation of the fuel pins.
We will refer to diffusion calculations using this second type of homogenization scheme as heterogeneous calculations.

% Previous works using Moltres
Moltres is a heterogeneous solver.
Previous works \cite{lindsay_introduction_2018}\cite{pater_multiphysics_2019} have used Moltres for the analysis of \glspl{MSR}.
In such calculations, Moltres input files defined two materials: the moderator and the fuel.
For such a configuration, a node in the mesh representing the moderator holds the neutronics and temperature information only for the moderator.
The same is true in the fuel.
A homogeneous calculation would not differentiate between moderator and fuel and would hold the information of both materials in each node.

% What I did in this section:
Keeping in mind Moltres previous works, we aimed for a heterogeneous calculation in a prismatic \gls{HTGR}.
For such a calculation, we modeled a fuel column of the MHTGR-350 and generated the group constants using Serpent.
Figure \ref{fig:fuelcolumn} displays the Serpent model geometry.
We obtained the group constants for three materials: moderator, coolant, and fuel compact.
Serpent ran 5$\times 10^5$ neutrons/cycle, 400 active cycles, and 100 inactive cycles for the calculations.
% The \gls{Keff} was 1.41933.
Taking advantage of the problem's symmetry, Moltres modeled only one-twelfth of the fuel column, Figure \ref{fig:fuelcolumn}.
We made the geometry and mesh using Gmsh \cite{geuzaine_gmsh_2020}.
The diffusion calculation had 183667 \glspl{DoF}/energy-group.
The Moltres input file set an eigenvalue convergence tolerance of 1$\times$10$^{-8}$.
Moltres calculation used a two energy group structure with a thermal cutoff at 0.625eV.
The eigenvalue calculation did not converge.
Although several factors could contribute to this behavior, we focused on the validity of the diffusion calculations in such a system.

% \begin{figure}[htbp!]
% 	\centering
% 	\includegraphics[width=0.45\linewidth]{figures-neutronics/oecd-standard-column-legend}
% 	\hfill
% 	\caption{Serpent model of a MHTGR-350 fuel column.}
% 	\label{fig:fuelcolumn}
% \end{figure}

\begin{figure}[htbp!]
	\centering
    \subfloat[Serpent model geometry.]{
        \includegraphics[width=0.35\textwidth]{figures-neutronics/oecd-standard-column-legend}
    }
    \subfloat[Moltres model geometry.]{
        \includegraphics[width=0.22\textwidth]{figures-neutronics/3D-assembly-30deg-reflec-meshB2}
    }
	\hfill
    \caption{Fuel column of the MHTGR-350. $xy$-plane in the active core region.}
	\label{fig:fuelcolumn}
\end{figure}

% Diffusion validity see bu1/tools-bu.txt
The diffusion theory considers that the current density is proportional to the gradient of the flux \cite{leppanen_development_2007}.
Such an approximation relies on the following assumptions:
\begin{itemize}
	\item The angular flux does not depend strongly on the angular variables.
	\item The fission source is isotropic.
	\item The time derivative of the current density is small compared to the mean collision time.
	\item The anisotropic energy-transfer is negligible in group-to-group scattering.
\end{itemize}

More detailed studies of the transport equation indicate that the following cases violate the assumption of a weak angular dependence \cite{duderstadt_nuclear_1976}:
\begin{itemize}
    \item Regions near vacuum boundaries and low-density material regions.
    \item Regions near strongly absorbing media.
    \item Regions near localized sources.
\end{itemize}

The diffusion theory applies best to geometries consisting of large homogeneous regions where the flux gradient is small.
This is the case for material regions whose geometrical scales are considerably larger than the neutron mean free path.
For this reason, we compared the neutron mean free path in the different fuel assembly materials, Table \ref{tab:mfp}.
The mean free path in the fuel compact and the moderator are in the order of the centimeters.
In the coolant, the mean free path magnitude is comparable to the fuel column dimensions.
These results suggest that a heterogeneous diffusion calculation of the prismatic fuel column violates the diffusion theory assumptions.

\begin{table}[htbp!]
  \centering
  \caption{Neutron mean free path in different materials. Values expressed in $cm$.}
  \begin{tabular}{l|cccc}
  \toprule
              & Fuel compact  & Moderator  & Coolant  & Homogeneous fuel \\
  \midrule
  Fast  		& 2.71 & 2.70 & 1137.31 & 3.37 \\
  Thermal		& 2.22 & 2.36 & 1945.49 & 2.89 \\

  \bottomrule
  \end{tabular}
  \label{tab:mfp}
\end{table}

As the next step, we conducted a feasibility study for the homogeneous calculation of the fuel assembly in Moltres.
Serpent calculated the homogeneous group constants of the fuel assembly.
We homogenized the fuel, coolant, and moderator to create a 'homogeneous fuel.'
This material's mean free path is in the order of the centimeters, Table \ref{tab:mfp}.
Next, Moltres used the homogeneous group constants to carry out an eigenvalue calculation.
We compared Moltres results with Serpent results.
Serpent's \gls{Keff} was 1.41933 while Moltres' was 1.4078788.
Moltres eigenvalue is smaller than Serpent's eigenvalue.
Additionally, we obtained the axial flux in the fuel column.
Figure \ref{fig:prelim} displays a comparison between Serpent and Moltres axial fluxes.
Serpent's flux is the average flux over the fuel column volume.
Moltres flux is the point-wise flux over the $z$-axis.
The fluxes are similar in shape and magnitude.

% Conclusion
We empathize that this was a feasibility study.
The following sections make a more in-depth analysis of more detailed results.
Based on these results and discussion, we use Moltres for carrying out homogeneous calculations in the following sections.

\begin{figure}[htbp!]
	\centering
    \subfloat[Serpent.]{
        \includegraphics[width=0.45\textwidth]{figures-neutronics/standard-column-detector-Axial}
    }
    \subfloat[Moltres.]{
        \includegraphics[width=0.45\textwidth]{figures-neutronics/homo-fuel}
    }
	\hfill
  \caption{Two group axial flux comparison.}
	\label{fig:prelim}
\end{figure}

\section{Serpent-Moltres comparison}

\subsection{Fuel column}

\begin{table}[htbp!]
  \centering
  \caption{Energy group structure.}
  \begin{tabular}{c|l|l|l|l|l|l|l|l|l|l|l|l}
  \toprule
  Upper boundary [eV] & 26    & 21   & 18   & 15a & 15b & 15c & 15d & 15e   & 12  & 9  & 6  & 3 \\
  \midrule
  1.49E+07            & 1     & 1    & 1    & 1   & 1   & 1   & 1   & 1     & 1   & 1  & 1  & 1 \\ \cline{1-2}
  7.41E+06            & 2     &      &      &     &     &     &     &       &     &    &    &   \\ \cline{1-10}
  3.68E+06            & 3     & 2    & 2    & 2   & 2   & 2   & 2   & 2     & 2   &    &    &   \\ \cline{1-2}
  6.72E+05            & 4     &      &      &     &     &     &     &       &     &    &    &   \\ \hline
  1.11E+05            & 5     & 3    & 3    & 3   & 3   & 3   & 3   & 3     & 3   & 2  & 2  & 2 \\ \cline{1-6} \cline{10-10}
  1.93E+04            & 6     & 4    & 4    & 4   & 4   &     &     &       & 4   &    &    &   \\ \cline{1-2}
  3.35E+03            & 7     &      &      &     &     &     &     &       &     &    &    &   \\ \cline{1-4} \cline{7-7}
  1.58E+03            & 8     & 5    & 5    &     &     & 4   &     &       &     &    &    &   \\ \cline{1-6} \cline{8-11}
  7.48E+02            & 9     & 6    & 6    & 5   & 5   &     & 4   & 4     & 5   & 3  &    &   \\ \cline{1-7} \cline{10-11}
  2.75E+02            & 10    & 7    & 7    & 6   & 6   & 5   &     &       & 6   & 4  &    &   \\ \cline{1-6} \cline{8-12}
  1.30E+02            & 11    & 8    & 8    & 7   & 7   &     & 5   & 5     & 7   & 5  & 3  &   \\ \cline{1-3} \cline{6-7}
  6.14E+01            & 12    & 9    &      &     & 8   & 6   &     &       &     &    &    &   \\ \cline{1-6} \cline{8-9}
  2.90E+01            & 13    & 10   & 9    & 8   & 9   &     & 6   & 6     &     &    &    &   \\ \cline{1-5} \cline{10-11}
  1.37E+01            & 14    & 11   & 10   & 9   &     &     &     &       & 8   & 6  &    &   \\ \cline{1-10}
  8.32E+00            & 15    & 12   & 11   & 10  & 10  & 7   & 7   & 7     & 9   &    &    &   \\ \cline{1-2}
  5.04E+00            & 16    &      &      &     &     &     &     &       &     &    &    &   \\ \hline
  2.38E+00            & 17    & 13   & 12   & 11  & 11  & 8   & 8   & 8     & 10  & 7  & 4  & 3 \\ \cline{1-3}
  1.29E+00            & 18    & 14   &      &     &     &     &     &       &     &    &    &   \\ \cline{1-12} 
  6.50E-01            & 19    & 15   & 13   & 12  & 12  & 9   & 9   & 9     & 11  & 8  & 5  &   \\ \cline{1-3} \cline{7-8}
  3.50E-01            & 20    & 16   &      &     &     & 10  & 10  &       &     &    &    &   \\ \cline{1-9}
  2.00E-01            & 21    & 17   & 14   & 13  & 13  & 11  & 11  & 10    &     &    &    &   \\ \cline{1-2} \cline{9-9}
  1.20E-01            & 22    &      &      &     &     &     &     & 11    &     &    &    &   \\ \cline{1-9} 
  8.00E-02            & 23    & 18   & 15   & 14  & 14  & 12  & 12  & 12    &     &    &    &   \\ \cline{1-4} \cline{7-9}
  5.00E-02            & 24    & 19   & 16   &     &     & 13  & 13  & 13    &     &    &    &   \\ \hline
  2.00E-02            & 25    & 20   & 17   & 15  & 15  & 14  & 14  & 14    & 12  & 9  & 6  &   \\ \cline{1-4} \cline{7-9}
  1.00E-02            & 26    & 21   & 18   &     &     & 15  & 15  & 15    &     &    &    &   \\
  \bottomrule
  \end{tabular}
  \label{tab:energygroups}
\end{table}

In this section, we investigated the effects of the energy group structure on the diffusion simulations.
We conducted two analyses.
First, we varied the number of energy groups.
Second, we tried different energy group structures for the same number of groups.
To reduce the computational expense, we narrowed down our focus to a fuel column of the MHTGR-350, Figure \ref{fig:fuelcolumn}.
The fuel column includes the bottom and top reflectors.
Tables \ref{tab:element-characteristics} and \ref{tab:compact} specify the model input parameters.

The first step in the calculation was to obtain the group constants using Serpent.
Figure \ref{fig:fuelcolumn} displays a $xy$-plane of the model.
To simplify Serpent's model, we did not consider the fuel handling holes or the bottom and top reflectors' coolant channels.
\glspl{HTGR} use \glspl{LBP} to reduce the power peaking factors in different active core regions.
Some reactors could have \glspl{LBP} in the rings closer to the reflectors, and no \glspl{LBP} in the middle rings.
This characteristic motivated the analysis of two cases, a fuel column that does not have \glspl{LBP}, and one that has.
The LBP's locations are the six corners of the fuel assembly, Figure \ref{fig:fuelassembly}.
The material temperatures were 600K and 1200K, cases that represent the \gls{CZP} and the \gls{HFP} core states.
Serpent ran 4$\times 10^5$ neutrons/cycle, 360 active cycles, and 40 inactive cycles for the calculations.

Taking advantage of the problem's symmetry, Moltres modeled only one-twelfth of the fuel column.
We made the geometry and mesh using Gmsh.
The mesh had 37120 elements and 22862 nodes.
The diffusion calculations had 22862 \glspl{DoF}/energy-group.
The Moltres input files set an eigenvalue and a flux convergence tolerance of 1$\times$10$^{-8}$.
Moltres calculations used different energy group structures listed in Table \ref{tab:energygroups}.

% This is very important and I should review it carefully
To compare the results from Serpent and Moltres, we present figure comparing the three-group axial fluxes.
Moltres ran the calculations for 26 energy groups and collapsed the results into three energy groups to facilitate the results' visualization.
Note that Serpent's flux is an average over the volume, while the Moltres' flux is the point-wise flux over a line.
Another figure compares the eigenvalue from Serpent and eigenvalues from Moltres for the different energy group structures.
The last analysis is for the Moltres axial flux.
Considering the 26 group structure as the reference value, we obtained the $L_2$-norm of the active core's axial flux relative error.

% Fluxes
To recapitulate, we simulated four operational cases: no \gls{LBP} at 600K, no \gls{LBP} at 1200K, \gls{LBP} at 600K, and \gls{LBP} at 1200K.
Figures \ref{fig:assembly-noLBP-600-flux} to \ref{fig:assembly-LBP-1200-flux} display the axial flux from the Serpent and the Moltres simulations for all cases.
For the no LBP at 600K case, the fluxes are close in shape and magnitude.
For the no LBP at 1200K case, the fluxes look similar.
The flux in Moltres has a straighter shape.
The thermal flux peak in the bottom reflector is bigger.
For the LBP at 600K case, the flux in Moltres has a larger magnitude.
Additionally, the shape of the Moltres flux is concave, while the Serpent flux is convex.
For the LBP at 1200K case, the flux in Moltres is larger.
The Moltres flux is more concave than the Serpent flux.
Overall, the fluxes in Moltres and Serpent are close in shape and magnitude.

% No LBP 600
\begin{figure}[htbp!]
	\centering
    \subfloat[Moltres.]{
        \includegraphics[width=0.45\textwidth]{figures-neutronics/3D-assembly-noLBP-600-26G}
    }
    \subfloat[Serpent.]{
        \includegraphics[width=0.45\textwidth]{figures-neutronics/serpent26G-noLBP-600-collapse}
    }
	\hfill
    \caption{Case no LBP at 600K. 3-group axial neutron flux.}
	\label{fig:assembly-noLBP-600-flux}
\end{figure}

% No LBP 1200
\begin{figure}[htbp!]
  \centering
    \subfloat[Moltres.]{
        \includegraphics[width=0.45\textwidth]{figures-neutronics/3D-assembly-noLBP-1200-26G}
    }
    \subfloat[Serpent.]{
        \includegraphics[width=0.45\textwidth]{figures-neutronics/serpent26G-noLBP-1200-collapse}
    }
  \hfill
    \caption{Case no LBP at 1200K. 3-group axial neutron flux.}
  \label{fig:assembly-noLBP-1200-flux}
\end{figure}

% LBP 600 
\begin{figure}[htbp!]
  \centering
    \subfloat[Moltres.]{
        \includegraphics[width=0.45\textwidth]{figures-neutronics/3D-assembly-LBP-600-26G}
    }
    \subfloat[Serpent.]{
        \includegraphics[width=0.45\textwidth]{figures-neutronics/serpent26G-LBP-600-collapse}
    }
  \hfill
    \caption{Case LBP at 600K. 3-group axial neutron flux.}
  \label{fig:assembly-LBP-600-flux}
\end{figure}

% LBP 1200
\begin{figure}[htbp!]
  \centering
    \subfloat[Moltres.]{
        \includegraphics[width=0.45\textwidth]{figures-neutronics/3D-assembly-LBP-1200-26G}
    }
    \subfloat[Serpent.]{
        \includegraphics[width=0.45\textwidth]{figures-neutronics/serpent26G-LBP-1200-collapse}
    }
  \hfill
    \caption{Case LBP at 1200K. 3-group axial neutron flux.}
  \label{fig:assembly-LBP-1200-flux}
\end{figure}

% eigenvalues
Table \ref{tab:keff} exhibits the reactivity difference ($\Delta \rho$) between the Serpent and Moltres eigenvalues.
We used equation \ref{eq:delta-rho} to obtain $\Delta \rho$.
The eigenvalues in Moltres differ slightly from the eigenvalues in Serpent.
Overall, the reactivity difference is less than 50 pcm.
We note that the number of energy groups does not affect the accuracy of the eigenvalue calculations in Moltres.

\begin{align}
	\Delta \rho &= \left| \frac{k_1-k_2}{k_1 k_2} \right| \label{eq:delta-rho}
  \intertext{where}
  k_1 &= \mbox{Serpent eigenvalue} \notag \\
  k_2 &= \mbox{Moltres eigenvalue.} \notag
\end{align}

\begin{table}[htbp!]
  \centering
  \caption{Serpent and Moltres eigenvalues.}
  \begin{tabular}{l|l|llllllll}
  \toprule
              & Serpent 					& \multicolumn{8}{c}{$\Delta \rho$ [pcm]}            \\ \cline{3-10} 
              &              			& 3   & 6   & 9   & 12   & 15   & 18   & 21   & 26   \\
  \midrule
no LBP, 600K  & 1.43800           & 10  & 7   & 6   & 6    & 5    & 6    & 6    & 12   \\
no LBP, 1200K & 1.37771           & 23  & 15  & 4   & 3    & 2    & 2    & 1    & 11   \\
LBP, 600K     & 1.12861           & 44  & 21  & 24  & 25   & 25   & 24   & 19   & 9    \\
LBP, 1200K    & 1.06554           & 36  & 40  & 29  & 32   & 44   & 43   & 25   & 25   \\
  \bottomrule
  \end{tabular}
  \label{tab:keff}
\end{table}

Figures \ref{fig:assembly-noLBP-er} and \ref{fig:assembly-LBP-er} show the $L_2$-norm of the relative error for the different energy group structures.
The no LBP case's relative error is smaller than the LBP case's relative error.
Overall, the relative error decreases with an increase in the number of energy groups.
Nonetheless, this is not always the case.
For example, in Figure \ref{fig:assembly-noLBP-er-b}, going from 12 to 15 groups, the thermal flux improves, but the fast flux worsens.
Additionally, we observe that a low number of energy groups yields more than 100$\%$ error.
In which case, we can conclude that the solution is wrong.

% No LBP
\begin{figure}[htbp!]
	\centering
    \subfloat[600K.]{
        \includegraphics[width=0.45\textwidth]{figures-neutronics/noLBP-600-er-final}
    }
    \subfloat[1200K.\label{fig:assembly-noLBP-er-b}]{
        \includegraphics[width=0.45\textwidth]{figures-neutronics/noLBP-1200-er-final}
    }
	\hfill
    \caption{No LBP case. L$_2$-norm relative error for different number of energy group structures.}
	\label{fig:assembly-noLBP-er}
\end{figure}

% LBP
\begin{figure}[htbp!]
	\centering
    \subfloat[600K.]{
        \includegraphics[width=0.45\textwidth]{figures-neutronics/LBP-600-er-final}
    }
    \subfloat[1200K.]{
        \includegraphics[width=0.45\textwidth]{figures-neutronics/LBP-1200-er-final}
    }
	\hfill
    \caption{LBP case. L$_2$-norm relative error for different number of energy group structures.}
	\label{fig:assembly-LBP-er}
\end{figure}

We added to the analysis the computational time and the peak memory usage during the simulations, Figure \ref{fig:assembly-time}.
All the simulations used 128 cores.
We present only the cases at 600K because the impact of the temperature change was not significant.
The computational requirements rise with an increase in the number of energy groups.
As the geometry uses a constant number of elements, the DoFs/energy-group is constant for all the simulations.
Thus, the total number of DoFs is proportional to the number of energy groups.
We also discern that the overall time of the LBP cases is higher than the no LBP cases.

% Time and memory
\begin{figure}[htbp!]
	\centering
    \subfloat[No LBP and 600 K.]{
        \includegraphics[width=0.45\textwidth]{figures-neutronics/time-noLBP-600}
    }
    \subfloat[LBP and 600 K.]{
        \includegraphics[width=0.45\textwidth]{figures-neutronics/time-LBP-600}
    }
	\hfill
	\caption{Computational time and memory requirements for different number of energy group structures.}
	\label{fig:assembly-time}
\end{figure}

Finally, we analyzed the impact of changing the energy group structure for a constant number of energy groups.
We chose 15 energy groups, as it yields a good overall accuracy and a not so high computational expense.
Table \ref{tab:energygroups} holds the different energy group structures.
Table \ref{tab:accuracy15} exhibits the $L_2$-norm of the relative error for the different energy group structures.
The energy group structure has an impact on accuracy.
Some energy group structures yield better results for some cases while giving worse results for others.
For example, the structure $15d$ gives better results for the LBP at 600K case than for the no LBP case at 600K.
To choose the best performing structure, we used a weighted average for the different groups.
We arbitrarily chose the weights to be 0.5, 0.3, and 0.2 for the thermal, epithermal, and fast fluxes.
Using this averaging scheme, we determined the group structure $15d$ to be the best one.

\begin{table}[htbp!]
  \centering
  \caption{Axial flux relative error $L_2$-norm for various energy group structures. Values expressed in percentage.}
  \begin{tabular}{@{}l|l|l| S[table-format=2.1] S[table-format=2.1] S[table-format=2.1] S[table-format=2.1] S[table-format=2.1] }
  \toprule
	LBP                  & Temperature {[}K{]}   & Flux       & \multicolumn{1}{c@{}}{15a} & \multicolumn{1}{c@{}}{15b}  & \multicolumn{1}{c@{}}{15c}  & \multicolumn{1}{c@{}}{15d}  & \multicolumn{1}{c@{}}{15e}  \\
	\midrule
	\multirow{6}{*}{No}  & \multirow{3}{*}{600}  & Fast       & 7.9  & 8.0  & 8.2  & 8.1  & 9.1  \\
	                     &                       & Epithermal & 6.6  & 6.5  & 8.6  & 8.2  & 9.2  \\
	                     &                       & Thermal    & 8.8  & 8.5  & 10.6 & 10.7 & 12.9 \\ \cline{2-8}
	                     & \multirow{3}{*}{1200} & Fast       & 7.1  & 7.7  & 5.7  & 5.1  & 4.5  \\
	                     &                       & Epithermal & 3.3  & 3.9  & 6.2  & 5.1  & 3.4  \\
	                     &                       & Thermal    & 5.0  & 4.7  & 8.5  & 8.2  & 8.4  \\ \hline
	\multirow{6}{*}{Yes} & \multirow{3}{*}{600}  & Fast       & 24.0 & 24.8 & 2.6  & 2.3  & 3.7  \\
	                     &                       & Epithermal & 21.0 & 21.7 & 2.0  & 1.6  & 2.7  \\
	                     &                       & Thermal    & 18.1 & 18.8 & 5.2  & 5.5  & 5.7  \\ \cline{2-8}
	                     & \multirow{3}{*}{1200} & Fast       & 36.2 & 37.3 & 6.9  & 6.6  & 25.9 \\
	                     &                       & Epithermal & 33.2 & 34.2 & 6.9  & 6.5  & 25.1 \\
	                     &                       & Thermal    & 29.6 & 30.6 & 8.5  & 8.3  & 20.3 \\
	\midrule
	\multicolumn{2}{l}{Weighted average}         &            & 17.3 & 17.8 & 6.3  & 6.0  & 10.8 \\
	\bottomrule
  \end{tabular}
  \label{tab:accuracy15}
\end{table}


\subsection{Full-core}

In this section, we compared the results from Serpent and Moltres for a full-core simulation.
The first step in the calculation was to obtain the group constants using Serpent.
Figure \ref{fig:fullcoremodel} displays the $xy$-plane of the model.
The model includes the bottom and top reflectors.
Tables \ref{tab:maincharac}, \ref{tab:element-characteristics} and \ref{tab:compact} specify the model input parameters.
All the fuel columns were standard to simplify Serpent’s model.
Additionally, the model did not include the fuel handling holes nor the bottom and top reflectors' coolant channels.
The model considered a fresh core.
Based on the previous section analyses, we chose the energy group structure $15d$ in Table \ref{tab:energygroups}.
The material temperatures were 600K and 1200K, cases that represent the \gls{CZP} and the \gls{HFP} core states.
Serpent ran 8$\times 10^5$ neutrons/cycle, 500 active cycles, and 100 inactive cycles for the calculations.

% Geometries
\begin{figure}[htbp!]
	\centering
    \subfloat[Serpent model geometry.]{
        \includegraphics[width=0.39\textwidth]{figures-neutronics/oecd-fullcore}
    }
    \subfloat[Moltres model geometry.]{
        \includegraphics[width=0.49\textwidth]{figures-neutronics/3D-fullcore-60-homo-meshB2}
    }
	\hfill
  \caption{MHTGR-350 full-core models.}
	\label{fig:fullcoremodel}
\end{figure}

Taking advantage of the problem's symmetry, Moltres modeled only one-twelfth of the fuel column, Figure \ref{fig:fullcoremodel}.
We made the geometry and mesh using Gmsh.
The mesh had 300720 elements and 160035 nodes.
The diffusion calculations had 160035 \glspl{DoF}/energy-group and a total of 2400525 DoFs.
The Moltres input files set an eigenvalue and a flux convergence tolerance of 1$\times$10$^{-8}$.

% Keff
Between Serpent and Moltres, we compared the \gls{Keff}, the power distribution, and the flux shape and magnitude in different zones of the reactor.
Table \ref{tab:full-keff} exhibits the \gls{Keff} from Serpent and Moltres.
Moltres values are larger than Serpent's.
The values are within a 300 pcm difference.

\begin{table}[htbp!]
  \centering
  \caption{Serpent and Moltres eigenvalues.}
  \begin{tabular}{l|lll}
  \toprule
              & Serpent			& Moltres  & $\Delta \rho$ [pcm] 	\\
  \midrule
			 600K  	& 1.10869     & 1.11150	 &	228		\\
			1200K 	& 1.06138     & 1.06468	 &	292   \\

  \bottomrule
  \end{tabular}
  \label{tab:full-keff}
\end{table}

% Power distribution
Figures \ref{fig:fullcore-600-power} and \ref{fig:fullcore-1200-power} show Serpent and Moltres radial power distributions.
The following analysis applies to both temperatures.
The first thing that came to our attention is the symmetry of the power distribution.
The results are symmetric with respect to a 60$^{\circ}$ line.
This result suggests that we could reduce the mesh size by half by considering only one-twelfth of the reactor.
The next observation is that Moltres result exhibits a higher power density than Serpent in the inner and outer rings.
The power density in the middle ring is lower in the Moltres case.
The largest difference is in the inner ring.
Overall, the results differ in less than 0.30 MW.

%Power distribution at 600K
\begin{figure}[htbp!]
	\centering
    \subfloat[Moltres.]{
        \includegraphics[width=0.45\textwidth]{figures-neutronics/3D-fullcore-600-15Gd-power}
    }
    \subfloat[Serpent.]{
        \includegraphics[width=0.45\textwidth]{figures-neutronics/serpent26G-600-power}
    }
	\hfill
	\caption{Radial power distribution at 600 K.}
	\label{fig:fullcore-600-power}
\end{figure}

%Power distribution at 1200K
\begin{figure}[htbp!]
	\centering
    \subfloat[Moltres.]{
        \includegraphics[width=0.45\textwidth]{figures-neutronics/3D-fullcore-1200-15Gc-power}
    }
    \subfloat[Serpent.]{
        \includegraphics[width=0.45\textwidth]{figures-neutronics/serpent26G-1200-power}
    }
	\hfill
	\caption{Radial power distribution at 1200 K.}
	\label{fig:fullcore-1200-power}
\end{figure}

% Fluxes
We placed an axial and a radial flux detectors in arbitrary regions of the reactor to compare the fluxes.
Figure \ref{fig:fullcore-detectors} shows the location of the detectors.
Note that the flux in Serpent is an average over the fuel column's volume, while the flux in Moltres is the point-wise flux over the fuel column's centerline.
The Serpent radial detector's volume had a 2$^{\circ}$-angle and a fuel assembly's height.
Both the Serpent and Moltres radial detector's location was the middle of the active core's height.
Moltres ran the calculations for 15 energy groups and collapsed the results into three groups to facilitate the results' visualization.
Figures \ref{fig:fullcore-600-axial1} and \ref{fig:fullcore-600-radial1} show the axial and radial fluxes at 600K.
Figure \ref{fig:fullcore-600-axial1} shows that the fast and epithermal fluxes in Moltres are larger, while the thermal flux is smaller.
The flux shapes are similar.
The epithermal and thermal fluxes are closer in magnitude in the active core in Serpent's simulation.
In Figure \ref{fig:fullcore-600-radial1}, Serpent fluxes present some 'noise.'
A higher number of generations/cycle in Serpent simulations would solve it.
Another alternative is using a detector with a larger volume.
Additionally, the flux in Serpent shows the location of the LBPs in the fuel assemblies.
A diffusion solver fails to capture such localized effects as the group constants are homogeneous in the fuel assembly.
The fast flux in Moltres is larger, while the epithermal and thermal fluxes have almost the same magnitudes.
Figures \ref{fig:fullcore-1200-axial1} and \ref{fig:fullcore-1200-radial1} display the fluxes at 1200K.
These results differ from the 600K case.
However, we observe the same behavior for both axial and radial fluxes.

%Detectors
\begin{figure}[htbp!]
	\centering
    \subfloat[Moltres model.]{
        \includegraphics[width=0.37\textwidth]{figures-neutronics/3D-fullcore-60-detectors2}
    }
    \subfloat[Serpent model.]{
        \includegraphics[width=0.52\textwidth]{figures-neutronics/oecd-fullcore-detectorsC}
    }
	\hfill
	\caption{Flux detector locations.}
	\label{fig:fullcore-detectors}
\end{figure}

% Axial flux1 at 600K
\begin{figure}[htbp!]
	\centering
    \subfloat[Moltres.]{
        \includegraphics[width=0.45\textwidth]{figures-neutronics/3D-fullcore-600-15Gd-axial1}
    }
    \subfloat[Serpent.]{
        \includegraphics[width=0.45\textwidth]{figures-neutronics/serpent26G-600-collapse-Axial1}
    }
	\hfill
	\caption{Axial flux at 600 K.}
	\label{fig:fullcore-600-axial1}
\end{figure}

%Radial flux at 600 K
\begin{figure}[htbp!]
	\centering
    \subfloat[Moltres.]{
        \includegraphics[width=0.45\textwidth]{figures-neutronics/3D-fullcore-600-15Gd-radial1}
    }
    \subfloat[Serpent.]{
        \includegraphics[width=0.45\textwidth]{figures-neutronics/serpent26G-600-collapse-Radial}
    }
	\hfill
	\caption{Radial flux at 600 K.}
	\label{fig:fullcore-600-radial1}
\end{figure}

% Axial flux1 at 1200K
\begin{figure}[htbp!]
	\centering
    \subfloat[Moltres.]{
        \includegraphics[width=0.45\textwidth]{figures-neutronics/3D-fullcore-1200-15Gc-axial1}
    }
    \subfloat[Serpent.]{
        \includegraphics[width=0.45\textwidth]{figures-neutronics/serpent26G-1200-collapse-Axial1}
    }
	\hfill
	\caption{Axial flux at 1200 K.}
	\label{fig:fullcore-1200-axial1}
\end{figure}

%Radial flux at 1200 K
\begin{figure}[htbp!]
	\centering
    \subfloat[Moltres.]{
        \includegraphics[width=0.45\textwidth]{figures-neutronics/3D-fullcore-1200-15Gc-radial1}
    }
    \subfloat[Serpent.]{
        \includegraphics[width=0.45\textwidth]{figures-neutronics/serpent26G-1200-collapse-Radial}
    }
	\hfill
	\caption{Radial flux at 1200 K.}
	\label{fig:fullcore-1200-radial1}
\end{figure}


\section{OECD/NEA MHTGR-350 MW Benchmark: Phase I Exercise 1}

% Exercise description
This section conducted Phase I Exercise 1 of the benchmark with Moltres and compared the results with the already published results \cite{oecd_nea_coupled_2020}.
The benchmark specifies the group constants required to conduct the exercise.
The group constants provision ensures a common dataset among various benchmark participants and allows for comparing stand-alone neutronic predictions with no thermal-fluids feedback.
% What should be reported \cite{oecd_nea_benchmark_2017}
The exercise requests the reporting of the global parameters: $K_{eff}$, \gls{CR} worth ($\Delta \rho_{CR}$), and axial offset ($AO$).
% It also requires the reporting of a power distribution and a neutron-flux map \cite{oecd_nea_benchmark_2017}.
It also requires the reporting of a power distribution map \cite{oecd_nea_benchmark_2017}.
Equations \ref{eq:controlrod} and \ref{eq:ao} define $\Delta \rho_{CR}$ and $AO$.
\begin{align}
    \Delta \rho_{CR} &= \frac{k_{out}-k_{in}}{k_{out}k_{in}}
		\label{eq:controlrod}
    \intertext{where}
    k_{out} &= \mbox{eigenvalue with \gls{CR} out (at position 1184.8 cm)} \notag \\
    k_{in} &= \mbox{eigenvalue with \gls{CR} in (at position 391.81 cm)} \notag
		\intertext{and}
    AO &= (TP_{top}-TP_{bottom})/(TP_{top}+TP_{bottom})
		\label{eq:ao}
    \intertext{where}
    TP_{top} &= \mbox{total power produced in the top half core} \notag \\
    TP_{bottom} &= \mbox{total power produced in the bottom half core.} \notag
\end{align}

Moltres modeled one-third of the reactor, Figure \ref{fig:bench-mesh}.
The model included the bottom and top reflectors.
The core comprised 232 hexagonal subdomains for which the benchmark provides their group constants.
Table \ref{tab:mac-region} lists the six macroscopic regions that we can differentiate in the model.
The simulations required two meshes: one for the CR out and one for the CR in.
The simulation with the CR out had 268393 \glspl{DoF}/energy-group, and a total of 6978218 DoFs.
The simulation with the CR in had 227592 \glspl{DoF}/energy-group, and a total of 5917392 DoFs.
The Moltres input files set an eigenvalue convergence tolerance of 1$\times$10$^{-8}$.

\begin{figure}[htbp!]
	\centering
	\includegraphics[width=0.55\linewidth]{figures-neutronics/oecd-fullcore-legend}
	\hfill
	\caption{1/3-rd of the MHTGR-350 geometry.}
	\label{fig:bench-mesh}
\end{figure}

\begin{table}[htbp!]
  \centering
  \caption{Macroscopic regions.}
  \label{tab:mac-region}
  \begin{tabular}{@{}l c}
  \toprule
  Macroscopic region    & Subdomains     \\
  \midrule
  Fuel              & 1 to 220      \\
  Bottom reflector  & 221 to 224    \\
  Inner reflector   & 225           \\
  Outer reflector   & 226-227       \\
  Top reflector     & 228 to 231    \\
  Control Rod       & 232           \\
  \bottomrule
  \end{tabular}
\end{table}

The benchmark exercise specifies the group constants and the group constants numbering map.
The benchmark definition used DRAGON-4 \cite{marleau_user_2016} to obtain the group constants from a full block configuration.
The dataset contains 26 energy groups.
Among the benchmark group constants, we find $\Sigma_g^t$, $D_g$, $\nu\Sigma_g^f$, $\Sigma_g^f$, $\chi_g^t$, and $\Sigma_{g'\rightarrow g}^s$ (see equation \ref{eq:diffusion}).
The benchmark group constants format differs from Moltres'.
Hence, we made a python script to handle the formatting differences.

The benchmark exercise sets periodic \glspl{BC} on the sides of the geometry.
However, a memory issue did not allow for implementing those BCs in our 26-group Moltres input file.
We approximated the periodic BC with the reflective BC.
Section \ref{sec:bench-bcs} discusses further the use of periodic and reflective BCs.

% 4.33 h and 4.11 h
On average, the simulations took 4.22 hours using 1024 cores.
Table \ref{tab:globalparam} shows the main results.
Moltres predicted a \gls{Keff} larger than the reference result.
The reactivity difference is of 99 pcm.
Moltres yields a smaller control rod worth.
The difference is of 312 pcm.
The axial offset for the Moltres simulation is 4$\%$ higher than the reference result.
We attribute the discrepancies to the use of the reflective BCs instead of the periodic BCs.
Once again, Section \ref{sec:bench-bcs} discusses further the use of periodic and reflective BCs.

\begin{table}[htbp!]
  \centering
  \caption{Global parameters.}
  \begin{tabular}{l|l|l}
  \toprule
  Parameter 	&  Benchmark  &  Moltres    \\
  \midrule
  K$_{eff}$ 	&  1.06691    &  1.06804    \\
  $\Delta \rho_{CR}$ [pcm]  & 822.1 	& 509.8 \\
  AO        	&  0.168      &  0.1753     \\
  \bottomrule
  \end{tabular}
  \label{tab:globalparam}
\end{table}

Figure \ref{fig:axialpower} shows the radially averaged axial power distribution.
Moltres' result is close in shape and magnitude to the reference result.
Figure \ref{fig:radialpower} shows the axially averaged radial power distribution.
Moltres' values are similar to the reference results.
Moltres' power distribution in the inner ring is larger.
The differences are within 0.25 W/cm$^3$.


\begin{figure}[htbp!]
	\centering
    \subfloat[Moltres result.]{
        \includegraphics[width=0.42\textwidth]{figures-neutronics/3D-fullcore26G-axialpower}
    }
    \subfloat[Benchmark result. Image reproduced from \cite{oecd_nea_coupled_2020}.]{
        \includegraphics[width=0.47\textwidth]{figures-neutronics/benchmark-axialpower}
    }
	\hfill
	\caption{Radially averaged axial power distribution.}
	\label{fig:axialpower}
\end{figure}

\begin{figure}[htbp!]
	\centering
    \subfloat[Moltres result.]{
        \includegraphics[width=0.47\textwidth]{figures-neutronics/3D-fullcore26G-radialpower}
    }
    \subfloat[Benchmark result. Image reproduced from \cite{oecd_nea_coupled_2020}.]{
        \includegraphics[width=0.42\textwidth, height=6.2cm]{figures-neutronics/benchmark-radialpower}
    }
	\hfill
  \caption{Axially averaged radial power distribution.}
  \label{fig:radialpower}
\end{figure}

\subsection{Periodic vs Reflective Boundary Conditions}
\label{sec:bench-bcs}

In the last section, we observed deviations in Moltres results.
In this section, we analyzed the discrepancies that the reflective \gls{BC} approximation may introduce.
As the previous section mentioned, the simulation's memory requirements restrict the use of periodic BCs.
To reduce the memory requirements, we collapsed the group constants to a smaller number of energy groups.
We simulated two cases: one that uses a 3-group structure and one that uses a 6-group structure (see Table \ref{tab:energygroups}).

The simulations required two meshes each, one for the CR out and one for the CR in.
The 3-group simulation had 62118 DoFs/energy-group (total of 186354 DoFs) and 61596 DoFs/energy-group (total of 184788 DoFs) for the CR out and CR in cases, respectively.
The 6-group simulation had 16898 DoFs/energy-group (total of 101388 DoFs) and 19116 DoFs/energy-group (total of 114696 DoFs) for the CR out and CR in cases, respectively.
We highlight that the 6-group simulation had to use a coarser mesh; otherwise, it would not run.
This fact confirms the suspicion that the simulation's memory requirements prevent it from running.

We ran simulations with periodic and reflective boundary conditions for both cases, and we compared their results.
Table \ref{tab:benchmark-bc} presents the results.
Keff rises with the reflective BC.
With the CR out, the raise is small.
However, with the CR in, the increase is considerable.
The combined effect of both increases leads to a decrease in the control rod worth.
The BC approximation barely affects the axial offset.

\begin{table}[htbp!]
  \centering
  \caption{Global parameters comparison for different types of BCs.}
  \begin{tabular}{l|l|l|l|l|l}
  \toprule
  Energy groups       & Type of BCs & K$_{eff, out}$ & K$_{eff, in}$ & $\Delta \rho_{CR}$ [pcm] & AO \\
  \midrule
  \multirow{2}{*}{3}  & Periodic     & 1.07571		& 1.06776		& 692.6		& 0.237		\\
                      & Reflective   & 1.07586	  & 1.07021   & 490.5		& 0.237	  \\ \hline
  \multirow{2}{*}{6}  & Periodic     & 1.07182		& 1.06356		& 724.3	  & 0.185  	\\
                      & Reflective   & 1.07197   	& 1.06610 	& 513.3		& 0.186		\\  
  \bottomrule
  \end{tabular}
  \label{tab:benchmark-bc}
\end{table}

\section{Conclusions}

% Preliminary studies: homogeneous vs heterogeneous isotopic distribution
The preliminary studies focused on several aspects of the simulations.
The first aspect was the effect of distributing the fuel compact isotopes homogeneously in the Serpent model.
The results showed that the homogenization of the fuel compact isotopes decreased the multiplication factor considerably.
The heterogeneous calculation took 28$\%$ more time.
Additionally, the homogeneous distribution appeared not to have a substantial impact on the group constants.
However, the multiplication factor's considerable difference suggested that the group constants’ small variations combined effect was significant.
Although the particles’ explicit modeling is time-consuming, it results necessary.

% Preliminary studies: homogeneous vs heterogeneous diffusion simulation
The next section studied the problem set-up in Moltres.
Different types of reactor technologies allow for homogeneous and heterogeneous diffusion calculations.
Moltres uses a heterogeneous diffusion solver.
In this work, we aim to use Moltres to solve prismatic HTGRs.
Nevertheless, the diffusion approximation fails to model properly regions where the mean free path is comparable to the region's dimensions.
The presence of helium in the fuel assembly of a prismatic \gls{HTGR} prohibits heterogeneous diffusion calculations.
Based on this discussion, we chose to use Moltres as a homogeneous diffusion solver.

% Serpent-Moltres: Fuel column
Focusing on a fuel column of the MHTGR-350, we investigated the effects of the energy group structure on the diffusion calculations.
We considered four different operational cases: a fuel column without LBPs and a fuel column with LBPs, both cases at 600K and 1200K.
Serpent obtained the homogenized group constants of the fuel column.
Moltres took such constants as input with a Gmsh mesh.
The first study compared the Moltres axial flux to Serpent axial flux.
Moltres used a 26 energy group structure to run the simulations.
Overall, the axial fluxes were close in shape and magnitude.
A different study focused on the effects of the energy group structure on the \gls{Keff}.
The number of energy groups did not affect the accuracy of the Moltres eigenvalue calculations.
We also compared the $L_2$-norm of the axial flux relative error in the active core using different energy group structures.
For the four operational cases, increasing the number of energy groups improved the accuracy
Additionally, we presented the simulation's computational expense for the different number of energy groups.
The simulation time and memory requirement rose by increasing the number of energy groups.
The computational time increased as well for the fuel column with LBPs.
Finally, we analyzed the impact of using different 15-group structures on the $L_2$-norm of the axial flux relative error.
We chose $15d$ as the best-performing energy group structure.

% Serpent-Moltres: Full-core
Based on the fuel column analysis results, we compared Moltres full-core results with Serpent reference results.
We considered two operational cases: 600K and 1200K.
Serpent obtained the homogenized group constants of the different regions of the reactor.
Moltres took such constants as input with a Gmsh mesh.
The first analysis compared the Serpent and Moltres eigenvalues.
Moltres results were bigger.
The overall differences were less than 300 pcm.
The second analysis compared the radial power distributions from both codes.
These results showcased the symmetry of the problem.
Reducing the problem size by half other simulations could reduce the computational expense.
For the most part, Moltres radial power distribution showed proximity to the Serpent's result.
We also compared Moltres and Serpent fluxes in two directions (axial and radial) in arbitrary core regions.
The axial fluxes showed small discrepancies, mostly in their magnitude.
The radial fluxes were close in shape and magnitude.
However, the radial flux in the diffusion calculation failed to capture the flux variation near the LBPs.
Overall, the fluxes were similar.

% OECD-Benchmark
The simulation capabilities for prismatic HTGRs have not reached the state of the art of LWRs.
This development delay motivated OECD/NEA to define a benchmark to carry out code-to-code comparisons.
The benchmark uses the MHTGR-350 as the reference design.
We conducted Phase I Exercise 1 with Moltres.
The benchmark defines the group constants for the exercise.
The group constants have a 26-energy group structure.
The benchmark exercise sets periodic \glspl{BC} on the sides of the geometry.
The simulation high memory requirements have challenged such implementation in Moltres. 
To go around such a barrier, we approximated the periodic \gls{BC} with a reflective BC.
Two out of three global parameters exhibited good agreement with the reference results.
However, the control rod worth presented a large discrepancy.
Such a difference was a consequence of the BC approximation.
Reducing the problem's size by collapsing the group constants to 3 and 6-energy groups, we compared the \gls{Keff} using the periodic and reflective BCs.
A reflective BC for the \gls{CR} out case did not substantially impact the \gls{Keff}.
On the other hand, the BC choice for the CR in case had a significant effect.
The combined effect of the approximation led to a large error in the CR worth.
The BC approximation had a small influence on the axial offset.

\chapter{Thermal-fluids}
\section{Preliminary studies}

This section carries out some preliminary studies using Moltres and MOOSE heat conduction module to solve the prismatic HTGR thermal-fluids.

\subsection{Verification of the thermal-fluids model}

To verify our methodology, this section solved a simplified cylindrical model whose analytical solution we know.
Section \ref{appendix:ver} presents the analytical solution of the problem.
Moltres/MOOSE obtained the numerical solution of the thermal-fluid equations from Section \ref{ch3:th}.

Figure \ref{fig:th-ver-mesh} displays the model geometry, which differentiates five subregions: fuel compact, helium gap, moderator, film, and coolant.
Table \ref{tab:th-ver-char} summarizes the geometry dimensions and the input parameters.
The model reference design was the GT-MHR.
The calculated moderator radius is the fuel/coolant pitch minus the fuel compact and coolant channel radii, which is the minimum distance between the fuel and coolant channels in the unit cell.
We obtained the calculated coolant radius by preserves the coolant channel volume.
The model assumed a sinusoidal power profile in the $z$-direction.

\begin{figure}[htbp!]
	\centering
	\includegraphics[width=0.25\linewidth]{figures-thermal/2D-preliminar-mesh2}
	\hfill
	\caption{Scaled-down version of the model geometry.}
	\label{fig:th-ver-mesh}
\end{figure}

\begin{table}[htbp!]
\centering
      \caption{Problem characteristics.}
      \label{tab:th-ver-char}
    % \begin{tabular}{@{}l c S[table-format=2.2] c c}
    \begin{tabular}{@{}l c c c c}
    \toprule
    \multicolumn{1}{c}{Parameter} & \multicolumn{1}{c}{Symbol} & \multicolumn{1}{c@{}}{Value} & \multicolumn{1}{c@{}}{Units} & \multicolumn{1}{c}{Reference} \\
    \midrule
  Fuel compact radius   & R$_f$     & 0.6225  & cm       & \cite{in_three-dimensional_2006} \\
  Fuel channel radius   & R$_g$     & 0.6350  & cm       & \cite{in_three-dimensional_2006} \\
  Coolant channel radius   & - 		& 0.7950  & cm       & \cite{in_three-dimensional_2006} \\
  Fuel/coolant pitch    & -			& 1.8850  & cm       & \cite{in_three-dimensional_2006} \\
  Fuel column height	& L 		& 793 	  & cm 		 & \cite{in_three-dimensional_2006} \\
  Coolant mass flow rate & $\dot{m}$ & 0.0176 & kg/s 	 & \cite{in_three-dimensional_2006} \\
  Average power density & q$_{ave}$ & 35      & W/cm$^3$ & \cite{in_three-dimensional_2006} \\
  Coolant inlet temperature 	& $T_{in}$ & 400  & $^{\circ}C$ & \cite{in_three-dimensional_2006} \\
  Helium inlet pressure & P 		& 70      & bar 	 & \cite{in_three-dimensional_2006} \\
  Helium density		& $\rho_c$  & 4.940 $\times 10^{-6}$ & kg/cm$^3$ & \cite{nist_thermophysical_2020} \\
  Helium heat capacity  & c$_{p,c}$	& 5188 & J/kg/K  & \cite{nist_thermophysical_2020} \\
  Fuel compact thermal conductivity & k$_f$ & 0.07    & W/cm/K & \cite{tak_numerical_2008} \\
  Gap thermal conductivity & k$_g$ & 3 $\times 10^{-3}$ & W/cm/K & \cite{tak_numerical_2008} \\
  Moderator thermal conductivity & k$_m$ & 0.30 & W/cm/K 	& \cite{tak_numerical_2008} \\
    \midrule
  \multicolumn{1}{c}{Calculated parameters} &  &  &  & \\  
    \midrule
  Calculated moderator radius 	& R$_m$ & 1.080  	& cm     & - \\
  Coolant film radius   		& R$_i$ & 1.090  	& cm     & - \\
  Calculated coolant radius 	& R$_c$ & 1.349  	& cm     & - \\
  Coolant average velocity  	& v$_c$ & 1794.33 	& cm/s   & - \\
  Film thermal conductivity  	& k$_i$ & 1.722 $\times 10^{-3}$ & W/cm/K & - \\
  \bottomrule
  \end{tabular}
\end{table}

Figure \ref{fig:th-ver-results} shows the axial and radial temperature profiles.
Both analytical and numerical solutions exhibit good agreement.
The outlet coolant temperature is 770.2 $^{\circ}$C whereas the average outlet coolant temperature of the VHTR is 950 $^{\circ}$C.
Note that this is a simplified model only for verification purposes, and it considers only one fuel channel while in the GT-MHR unit cell two fuel channels deposit their heat into one coolant channel.

\begin{figure}[htbp!]
	\centering
    \subfloat[Fuel centerline and bulk coolant axial temperatures.]{
        \includegraphics[width=0.45\textwidth]{figures-thermal/2D-preliminar-axial}
    }
    \subfloat[Radial temperature at z=L/2=396.5 cm.]{
        \includegraphics[width=0.45\textwidth]{figures-thermal/2D-preliminar-radial2}
    }
	\hfill
    \caption{Temperature profiles.}
	\label{fig:th-ver-results}
\end{figure}

\section{Unit cell problem}

This section solved the unit cell problem in the hot spot of an HTGR.
We intended to reproduce In et al. 2006 \cite{in_three-dimensional_2006} in an effort for validating the model.
We chose this article as it solves a three-dimensional unit-cell model gives one of the most complete descriptions in the open literature.
Table \ref{tab:th-val-unit-char} presents the problem characteristics.
The material properties specification of the solids is missing in the article, and we replaced them with parameters from Tak et al. 2008 \cite{tak_numerical_2008}.
Figure \ref{fig:th-val-unit-model} displays a $XY$-plane of the model geometry and the material properties that depend on the temperature.
Additionally, In et al. used a chopped cosine as the power profile.
To simplify the analysis, we used the average value of the power profile.

\begin{table}[htbp!]
\centering
      \caption{Problem characteristics.}
      \label{tab:th-val-unit-char}
    % \begin{tabular}{@{}l c S[table-format=2.2] c c}
    \begin{tabular}{@{}l c c c c}
    \toprule
    \multicolumn{1}{c}{Parameter} & \multicolumn{1}{c}{Symbol} & \multicolumn{1}{c@{}}{Value} & \multicolumn{1}{c@{}}{Units} & \multicolumn{1}{c}{Reference} \\
    \midrule
  Fuel compact radius       & R$_f$ & 0.6225    & cm   & \cite{in_three-dimensional_2006} \\
  Fuel channel radius       & R$_g$ & 0.6350    & cm   & \cite{in_three-dimensional_2006} \\
  Coolant channel radius    & R$_c$ & 0.7950    & cm   & \cite{in_three-dimensional_2006} \\
  Fuel/coolant pitch        & p     & 1.8850    & cm   & \cite{in_three-dimensional_2006} \\
  Fuel column height        & L     & 793       & cm   & \cite{in_three-dimensional_2006} \\
  % input parameter characteristics
  Coolant channel mass flow rate & $\dot{m}$ & 0.0176 & kg/s & \cite{in_three-dimensional_2006} \\
  Average power density     & q$_{ave}$ & 35    & W/cm$^3$   & \cite{in_three-dimensional_2006} \\
  Inlet coolant temperature & T$_{in}$  & 400   & $^{\circ}$C  & \cite{in_three-dimensional_2006} \\
  Helium inlet pressure & P & 70 & bar & \cite{in_three-dimensional_2006} \\
  Helium density        & $\rho$  & 4.94 $\times 10^{-6}$ & kg/cm$^3$ & \cite{nist_thermophysical_2020} \\
  Helium heat capacity  & c$_p$ & 5188 & J/kg/K & \cite{nist_thermophysical_2020} \\
    \midrule
  \multicolumn{1}{c}{Calculated parameters} &  &  &  & \\  
    \midrule
  Coolant film radius       & R$_i$ & 0.8050    & cm     & -  \\
  Coolant average velocity  & v$_c$ & 1794.33   & cm/s   & -  \\
  Film thermal conductivity & k$_i$ & 1.731 $\times 10^{-3}$ & W/cm/K & -  \\
  \bottomrule
  \end{tabular}
\end{table}

\begin{figure}[htbp!]
	\centering
    \subfloat[Model geometry.]{
        \includegraphics[width=0.40\textwidth]{figures-thermal/val-unit-mesh}
    }
    \subfloat[Material properties.]{
        \includegraphics[width=0.50\textwidth]{figures-thermal/val-unit-matprop}
    }
	\hfill
	\label{fig:th-val-unit-model}
\end{figure}

Figure \ref{fig:th-val-unit-temps} shows the temperature profiles.
The axial temperatures increase from the top to the bottom of the reactor.
The moderator and coolant temperatures are parallel as the model assumes a film thermal conductivity independent of the temperature.
The fuel and moderator temperature difference decreases.
As the thermal conductivities of the different materials increase with temperature, the thermal resistance between the moderator and the fuel decreases.
Table \ref{fig:th-val-unit-results} summarizes the results.
Moltres/MOOSE coolant temperature is smaller by 4$^{\circ}$C.
The moderator temperature is larger by 9$^{\circ}$C.
The fuel temperature is larger by 22$^{\circ}$C.
The cause of this is the power profile simplification.
For a sinusoidal power profile the fuel-to-coolant temperature difference is small in the outlet, as we have seen in the previous section.
The opposite scenario is the uniform power profile, where the fuel-to-coolant temperature difference is larger.
In et al. used a chopped cosine power profile which we can think of it as the case in between a uniform and a sinusoidal power profile.
Overall, our model results showed good agreement with In et al. results.

\begin{figure}[htbp!]
  \centering
    \subfloat[Maximum fuel, moderator, and bulk coolant axial temperatures.]{
        \includegraphics[width=0.45\textwidth]{figures-thermal/in-2006-5-axial}
    }
    \subfloat[Outlet plane temperature z=793 cm.]{
        \includegraphics[width=0.45\textwidth]{figures-thermal/val-unit-outlet-plane}
    }
  \hfill
    \caption{Temperature profiles.}
  \label{fig:th-val-unit-temps}
\end{figure}

\begin{table}[htbp!]
\centering
      \caption{Maximum temperatures.}
      \label{tab:th-val-unit-results}
    \begin{tabular}{@{}l c c}
    \toprule
  Parameter   & In et al. 2006 \cite{in_three-dimensional_2006} & Moltres/MOOSE \\
    \midrule
  Maximum coolant temperature [$^{\circ}$C]   & 1144 & 1140 \\
  Maximum moderator temperature [$^{\circ}$C] & 1250 & 1259 \\
  Maximum fuel temperature [$^{\circ}$C]      & 1295 & 1317 \\
    \bottomrule
  \end{tabular}
\end{table}

\section{Fuel assembly}

This section aims to reproduce some of the analyses in Sato et al. 2010 \cite{sato_computational_2010}.
The study uses the GT-MHR as the reference reactor for the calculations.
The GT-MHR shares the geometry specifications with the MHTGR.
Table \ref{tab:element-characteristics} specifies the fuel element geometry.

Material properties from \cite{johnson_cfd_2009}.

The moderator is graphite H-451.
Fuel compact and moderator thermal conductivities displayed in \ref{tab:th-val-assem-mat}.
Helium thermal conductivity from \cite{nist_thermophysical_2020}.

Figure \ref{fig:th-val-assem-model} displays the model geometry and the model material properties that depend on the temperature.
Table \ref{tab:th-val-assem-massflow} shows the mass flow rate in each coolant channel.

\begin{table}[htbp!]
\centering
      \caption{Problem characteristics.}
      \label{tab:th-val-unit-char}
    % \begin{tabular}{@{}l c S[table-format=2.2] c c}
    \begin{tabular}{@{}l c c c c}
    \toprule
    \multicolumn{1}{c}{Parameter} & \multicolumn{1}{c}{Symbol} & \multicolumn{1}{c@{}}{Value} & \multicolumn{1}{c@{}}{Units} & \multicolumn{1}{c}{Reference} \\
    \midrule
  Inlet coolant temperature & T$_{in}$  & 490   & $^{\circ}$C   & \cite{sato_computational_2010} \\
  Helium inlet pressure     & P         & 70    & bar           & \cite{sato_computational_2010} \\
  Helium density            & $\rho$    & 4.37 $\times 10^{-6}$ & kg/cm$^3$ & \cite{nist_thermophysical_2020} \\
  Helium heat capacity      & c$_p$     & 5188  & J/kg/K        & \cite{nist_thermophysical_2020} \\
  Average power density     & q$_{ave}$ & 27.88 & W/cm$^3$      & \cite{sato_computational_2010} \\
    \midrule
  \multicolumn{1}{c}{Calculated parameters} &  &  &  & \\  
    \midrule
  Coolant film radius       & R$_i$ & 0.804    & cm     & -  \\
  Film thermal conductivity & k$_i$ & 2.09 $\times 10^{-3}$ & W/cm/K & -  \\
  \bottomrule
  \end{tabular}
\end{table}

\begin{table}[htbp!]
\centering
  \caption{Thermal conductivity coefficients.}
  \label{tab:th-val-assem-mat} 
  \begin{tabular}{c|ccc|c}
\toprule
                          & \multicolumn{3}{c|}{Moderator}         & Fuel compact \\ \hline
Temperature range {[}K{]} & 255.6-816 & 816-1644.4 & 1644.4-1922.2 & 255.6-2200   \\
\midrule
A1                        & 28.6      & 1.24E+2    & 41.5          & 3.94         \\
A2                        & -         & -3.32E-1   & -             & 3.59E-3      \\
A3                        & -         & 4.09E-4    & -             & -1.98E-9     \\
A4                        & -         & -2.11E-7   & -             & 3.19E-12     \\
A5                        & -         & 4.02E-11   & -             & -9.77E-16    \\
\bottomrule
  \end{tabular}
\end{table}

\begin{figure}[htbp!]
  \centering
    \subfloat[Model geometry.]{
        \includegraphics[width=0.45\textwidth]{figures-thermal/val-assem-mesh}
    }
    \subfloat[Material properties.]{
        \includegraphics[width=0.45\textwidth]{figures-thermal/val-assem-matprop}
    }
  \hfill
  \label{fig:th-val-assem-model}
\end{figure}

\begin{table}[htbp!]
\centering
  \caption{Mass flow rate [g/s]. Values form \cite{sato_computational_2010}.}
  \label{tab:th-val-assem-massflow}
  \begin{tabular}{lllllllllllllll}
\toprule
Channel & 1 & 2 & 3 & 4 & 5 & 6 & 7 & 8 & 9 & 10 & 11 & 12 & 13 & Gap \\
\midrule
No gap  & 6.18 & 11.34 & 11.37 & 11.38 & 11.43 & 11.33 & 22.70 & 22.73 & 22.73 & 11.38 & 22.77 & 22.91 & 11.44 & -     \\
3mm gap & 5.88 & 10.80 & 10.85 & 10.91 & 11.08 & 10.80 & 21.58 & 21.67 & 21.83 & 10.88 & 21.81 & 22.20 & 11.10 & 16.56 \\
\bottomrule
\end{tabular}
\end{table}

Table \ref{tab:th-val-assem-results}
Figure \ref{fig:th-val-assem-temps}

\begin{table}[htbp!]
  \centering
  \caption{Maximum temperatures.}
  \label{tab:th-val-assem-results}
\begin{tabular}{l|ll|ll}
\toprule
                                   & \multicolumn{2}{l|}{No gap} & \multicolumn{2}{l}{3mm gap} \\ \midrule
                                   & Sato et al. & Moltres/MOOSE & Sato et al. & Moltres/MOOSE \\ \midrule
Maximum fuel temperature           & 1090    & 1094              & 1115     & 1114             \\
Maximum outlet coolant temperature & 985     & 983               & 1007     & 1005               
\bottomrule
\end{tabular}
\end{table}

\begin{figure}[htbp!]
  \centering
    \subfloat[Line A-B.]{
        \includegraphics[width=0.45\textwidth]{figures-thermal/val-assem-lineAB}
    }
    \subfloat[Line A-C.]{
        \includegraphics[width=0.45\textwidth]{figures-thermal/val-assem-lineAC}
    }
  \hfill
  \caption{Outlet plane temperature along the line A-B and line A-C.}
  \label{fig:th-val-assem-temps}
\end{figure}

% \begin{figure}[htbp!]
%   \centering
%     \subfloat[No gap.]{
%         \includegraphics[width=0.45\textwidth]{figures-thermal/val-assem-input}
%     }
%     \subfloat[3mm gap.]{
%         \includegraphics[width=0.45\textwidth]{figures-thermal/val-assem-input}
%     }
%   \hfill
%   \caption{Outlet plane temperature profile.}
%   \label{fig:th-val-assem-temps}
% \end{figure}


\subsection{Flow distribution analysis}

Talk about the necessity for this capability.
Also mention the different exercises from the benchmark.

Compares mass flow and max coolant and fuel temperatures: w/ no gap (it would be better with the gap, but the previous analysis is still wrong sooooo)
- sato et al
- flat (same velocity across all the channels)
- incompressible
- acceleration term

This simplified algorithms allow for solving the mass flow rate distribution in the core for steady-state cases, and for transient cases as an approximation.
However, in coupled analyses, the flow distribution depends on the temperature and will change along time.
This creates the necessity for developing tools integrated into Moltres.
Currently, MOOSE has a module for modeling the incompressible Navier-Stokes equations.
Integrating that module into the solver could improve the accuracy.
This task will be part of the future work.

\section{Full core}

This section will extend the methodology to a full-core problem and it will intend to solve Exercise 2 of Phase I of the OECD/NEA MHTGR-350 Benchmark.

\section{Neutronics and Thermal-fluids Coupling}

3 options:
- Heterogeneous model
- Homogenized media and sub-channel unit cell model: how does it solve advection ?
- Porous media model


\chapter{Hydrogen Production}
% Introduction or Objectives
% Efficient hydrogen production by \glspl{HTGR} motivates the last chapter of the thesis.
% We analyze several hydrogen production methods coupled to different nuclear reactor desings.
% In order to find the most efficient method, we not only consider \glspl{HTGR} but also other type of reactors.

\section{Introduction}

% The energy problem
Energy is one of the most vital contributors to economic growth.
In the future, economies will continue to expand, populations will do so too, and their energy demand will accompany such change \cite{burke_impact_2018} \cite{el-shafie_hydrogen_2019}.
Meeting these future needs requires the development of clean energy sources as environmental concerns continue to rise.

As seen in Figure \ref{fig:ghg}, electricity generation was one of the economic sectors that released the most \glspl{GHG} in the \gls{US} in 2017.
As \gls{CO2} is the main component in \glspl{GHG}, decarbonizing electricity generation will allow us to meet the increases in energy demand and address the environmental concerns simultaneously.

\begin{figure}[htbp!]
	\centering
	\includegraphics[width=0.4\linewidth]{figures-hydro/total-ghg-2017.png}
	\hfill
	\caption{Total \gls{US} \gls{GHG} emissions by economic sector in 2017. Image reproduced from \cite{us_epa_sources_2020}.}
	\label{fig:ghg}
\end{figure}

% word on solar energy and the duck curve
To address these concerns, utility companies are relying more and more on renewable energy resources, such as wind and solar \cite{ming_resource_2019}.
However, high solar adoption creates a challenge.
The need for electricity generators to quickly ramp up increases when the sun sets and the contribution from the \gls{PV} falls \cite{us_department_of_energy_confronting_2017}.
The "duck curve" (or duct chart) depicts this phenomenon, Figure \ref{fig:duck}.
The \gls{CAISO} developed the duck curve to illustrate the grid's net load \cite{bouillon_prepared_2014}.
We define the net load as the difference between forecasted load and expected electricity production from solar.

Moreover, the duck curve reveals another issue.
Over-generation may occur during the middle of the day, and high-levels of non-dispatchable generation may exacerbate the situation.
As a consequence, the market would experience sustained zero or negative prices during the middle of the operating day \cite{bouillon_prepared_2014}.

\begin{figure}[htbp!]
	\centering
	\includegraphics[width=0.75\linewidth]{figures-hydro/caiso-duck.png}
	\hfill
	\caption{The duck curve. Image reproduced from \cite{bouillon_prepared_2014}.}
	\label{fig:duck}
\end{figure}

% solutions to the duck curve
The simplest solution to a demand ramp-up is to increase dispatchable generation, resources with fast ramping and fast starting capabilities such as natural gas and coal \cite{bouillon_prepared_2014}, and, consequently, decrease non-dispatchable generation, such as geothermal, nuclear, and hydro.
Nonetheless, an approach like this is not consistent with the goal of reducing carbon emissions.
Hence, our focus drifts to other potential low-carbon solutions, like nuclear generation and electricity storage through \gls{H2} production.

% Transportation problem
Unfortunately, a carbon-neutral electric grid will be insufficient to halt climate change because transportation is a significant contributor to \gls{GHG} emissions.
As seen in Figure \ref{fig:ghg}, transportation released the most \glspl{GHG} in the \gls{US} in 2017. Thus, decarbonizing transportation underpins global carbon reduction.
One possible strategy is to develop a hydrogen economy, as Japan is currently doing.
Japan's strategy rests on the firm belief that \gls{H2} can be a decisive response to its energy and climate challenges.
It could foster deep decarbonization of the transport, power, industry, and residential sectors while strengthening energy security \cite{nagashima_japans_2018}.
In the transportation sector, Japan plans to deploy fuel cell vehicles, trucks, buses, trains, and ships.

Although \gls{H2} technologies do not release CO$_2$, any \gls{H2} production method is only as carbon-free as the energy source it relies on (electric, heat, or both).
Nuclear reactors introduce a clean energy option to manufacture \gls{H2}.

The \gls{UIUC} is leading by example and actively working to reduce \gls{GHG} emissions from electricity generation and transportation (among other sectors) on its campus.
In pursuance of those efforts, the university has developed the \gls{icap}.

\section{\gls{icap}}
% ICAPP
In 2008, \gls{UIUC} signed the American College and University Presidents' Climate Commitment, formally committing to becoming carbon neutral as soon as possible, no later than 2050.
The university developed the first \gls{icap} in 2010 as a comprehensive roadmap toward a sustainable campus environment \cite{university_of_illinois_at_urbana-champaign_illlinois_2015}.
The \gls{icap} defines a list of goals, objectives, and potential strategies for six topical areas.

\begin{itemize}
	\item Energy Conservation and Building Standards:
\end{itemize}
Focuses on maintaining or reducing campus gross square footage, strengthen conservation efforts, and engage the campus community in energy conservation.

\begin{itemize}
	\item Energy Generation, Purchasing, and Distribution:
\end{itemize}
Efforts towards the exploration of 100\% clean campus energy options.
This includes expanding on-campus solar energy production, the extension of the purchase of clean energy from low-carbon energy sources, and the offset of all emissions from the National Petascale Computing Facility.

\begin{itemize}
	\item Transportation:
\end{itemize}
This area comprises the efforts to reduce air travel emissions, reduce Urbana-Champaign campus fleet emissions, and study scenarios for complete conversion of the campus fleet to renewable fuels.

\begin{itemize}
	\item Water and Stormwater:
\end{itemize}
This area focuses on improving the water efficiency of cooling towers, perform a water audit to establish water conservation targets, determine upper limits for water demand by end-use, and implement projects to showcase the potential of water and stormwater reuse.

\begin{itemize}
	\item Purchasing, Waste, and Recycling:
\end{itemize}
Attempt to standardize office paper purchases, cleaning products, computers, other electronics, and freight delivery services.
It also attempts to foment recycling by reducing non-durable goods purchases and reducing municipal solid waste going to landfills.

\begin{itemize}
	\item Agriculture, Land Use, Food, and Sequestration:
\end{itemize}
This area will perform a comprehensive assessment of \gls{GHG} emissions from agricultural operations, and develop a plan to reduce them, implement a project that examines the foodservice carbon footprint for Dining, and increase carbon sequestration in campus soils.

\section{Objectives}

As mentioned earlier, we place our attention on two areas: electricity generation and transportation.
We will turn our attention to electricity generation and transportation on the \gls{UIUC} campus.
Consequently, this work's objective aligns with the efforts in two of the six target areas defined on the \gls{icap}.

Regarding electricity generation, our analysis focuses on the \gls{UIUC} grid.
The present work quantifies the magnitude of the duck curve in such a grid.
To mitigate the risk of over-generation, we propose to use the over-generated energy to manufacture \gls{H2}.
We chose a nuclear reactor to be the primary source of energy.
The next step is to quantify how much \gls{H2} different production methods can produce.
Section \ref{sec:hydro} discusses a few hydrogen production methods considered for our analysis.
Finally, we will calculate how much electricity we would generate using the \gls{H2} produced.

Regarding transportation, we study the conversion of the \gls{UIUC} fleet on campus to \glspl{FCEV}.
Additionally, the analysis includes the conversion of the \gls{MTD} fleet as well.
The first step is to determine the fuel consumed by both fleets and how much \gls{H2} enables the fleets' complete conversion.
Finally, we consider a few reactor designs and analyze which of them could produce enough \gls{H2} to fulfill both fleet requirements.

Both studies propose the same solution, a nuclear reactor coupled to a hydrogen plant.
In terms of electricity generation, this solution will decrease the need for dispatchable sources and, consequently, reduce carbon emissions.
In terms of transportation, it will eliminate carbon emissions.

%Mention the choice of reactor
In both analyses, many reactor choices can satisfy our needs.
The typical \gls{UIUC}'s grid demand is smaller than 80 MW \cite{dotson_optimal_2020}.
Accordingly, we consider reactors of small capacities, such as microreactors and \glspl{SMR}.
Section \ref{sec:reactors} discusses their characteristics.

\section{Hydrogen production methods}
\label{sec:hydro}

This section introduces several hydrogen production processes and their energy requirements.

\subsection{Electrolysis}

The electrolysis of water is a well-known method whose commercial use began in 1890.
This process produces approximately 4\% of \gls{H2} worldwide.
The process is ecologically clean because it does not emit \glspl{GHG}.
However, in comparison with other methods, electrolysis is a highly energy-demanding technology \cite{kalamaras_hydrogen_2013}.

Three electrolysis technologies exist.
Alkaline-based is the most common, the most developed, and the lowest in capital cost.
It has the lowest efficiency and, therefore, the highest electrical energy cost.
Proton exchange membrane electrolyzers are more efficient but more expensive than Alkaline electrolyzers.
\gls{SOEC} electrolyzers are the most electrically efficient but the least developed.
\gls{SOEC} technology has challenges with corrosion, seals, thermal cycling, and chrome migration \cite{kalamaras_hydrogen_2013}.
As the first two technologies work with liquid water and the latter requires high-temperature steam, we will refer to the first two as \gls{LTE} and the latter as \gls{HTE}.

Water electrolysis converts electric and thermal energy into chemical energy stored in hydrogen.
The process enthalpy change $\Delta H$ determines the required energy for the electrolysis reaction to take place.
Part of the energy corresponds to electric energy $\Delta G$ and its rest to thermal energy $T \cdot \Delta S$, Equation \ref{eq:electrolysis1}.

\begin{align}
	\Delta H &= \Delta G + T \Delta S
\label{eq:electrolysis1}
    \intertext{where}
    \Delta H &= \mbox{Total specific energy [kWh/kg-H$_2$]} \\
    \Delta G &= \mbox{Specific electrical energy [kWh/kg-H$_2$]} \\
    T \Delta S &= \mbox{Specific thermal energy [kWh/kg-H$_2$].}
\end{align}

In \gls{LTE}, electricity generates thermal energy.
Hence, $\Delta H$ alone determines the process required energy.
$\Delta H$ is equal to 60 kWh/kg-H$_2$ considering a 67$\%$ efficiency \cite{usdrive_hydrogen_2017}.

In \gls{HTE}, a high-temperature heat source is necessary to provide thermal energy.
$\Delta G$ decreases with increasing temperature, Figure \ref{fig:electro1}.
Decreasing the electricity requirement results in higher overall production efficiencies since heat-engine-based electrical work has a thermal efficiency of 50$\%$ or less \cite{j_e_obrien_high_2010}.
Figure \ref{fig:electro1} shows $\Delta G$ and $T \Delta$S.
$\Delta G$ considers the \gls{SOEC} to have an electrical efficiency of 88$\%$ \cite{helmeth_high_2020}.
$T \Delta S$ accounts for the latent heat of water vaporization.
Note that the process is at 3.5 MPa.
$\Delta G$ increases with pressure.
However, we chose a high pressure to save energy, as compressing liquid water is cheaper than compressing the hydrogen \cite{obrien_status_2010}.

\begin{figure}[htbp!]
	\centering
	\includegraphics[width=0.6\linewidth]{figures-hydro/hte-energy-P.png}
	\hfill
	\caption{Energy required by \gls{HTE} at 3.5 MPa.}
	\label{fig:electro1}
\end{figure}

Finally, equations \ref{eq:electrolysis2a} and \ref{eq:electrolysis2b} determine the electrical $P_{EH2}$ and thermal power $P_{TH2}$ required by the hydrogen plant.

\begin{align}
	P_{EH2} &= \dot{m}_{H2} \Delta G \label{eq:electrolysis2a} \\
	P_{TH2} &= \dot{m}_{H2} T \Delta S \label{eq:electrolysis2b}
	\intertext{where}
	P_{EH2} &= \mbox{Total electrical power [kW]} \\
	P_{TH2} &= \mbox{Total thermal power [kW]} \\
	\dot{m}_{H2} &= \mbox{\gls{H2} production rate [kg/h].}
\end{align}

\subsection{Sulfur-Iodine Thermochemical Cycle}

Thermochemical water-splitting is converting water into hydrogen and oxygen by a series of thermally driven chemical reactions.
The direct thermolysis of water requires temperatures above 2500 $^{\circ}$C for significant hydrogen generation.
At this temperature, the process can decompose a 10\% of the water.
A thermochemical water-splitting cycle accomplishes the same overall result using much lower temperatures.

General Atomics, Sandia National Laboratories, and the University of Kentucky compared 115 cycles that would use high-temperature heat from an advanced nuclear reactor \cite{brown_high_2003}.
The report specifies a set of screening criteria used to rate each cycle.
The highest scoring method was the \gls{SI} Cycle.

The \gls{SI} cycle consists of the three chemical reactions represented in Figure \ref{fig:sulfur1}.
The whole process takes in water and high-temperature heat and releases hydrogen and oxygen.
The process does not use any electricity.
The process recycles all reagents and does not have any effluents \cite{yildiz_efficiency_2006}.
The chemical reactions are:

\begin{align}
	I_2 + SO_2 + 2H_2O &\rightarrow 2HI + H_2SO_4 \\
	H_2SO4 &\rightarrow SO_2 + H_2O + 1/2O_2 \\
	2HI &\rightarrow I_2 + H_2.
\end{align}

\begin{figure}[htbp!]
	\centering
	\includegraphics[width=0.5\linewidth]{figures-hydro/sulfur1.png}
	\hfill
	\caption{Diagram of the Sulfur-Iodine Thermochemical process. Image reproduced from \cite{benjamin_russ_sulfur_2009}.}
	\label{fig:sulfur1}
\end{figure}

Figure \ref{fig:sulfur2} presents the specific energy requirements of the cycle $\Delta H$.
Several sources disagree on the minimum temperature for the process to be viable.
Our analysis considers the process feasible only for temperatures above 800 $^{\circ}$C.
Finally, equation \ref{eq:sulfur4} determines the thermal power $P_{TH2}$ required by the hydrogen plant.

\begin{figure}[htbp!]
	\centering
	\includegraphics[width=0.55\linewidth]{figures-hydro/si-energy2.png}
	\hfill
	\caption{Energy required by the Sulfur-Iodine Thermochemical Cycle.}
	\label{fig:sulfur2}
\end{figure}

\begin{align}
	P_{TH2} &= \dot{m}_{H2} \Delta H
	\label{eq:sulfur4}
	\intertext{where}
	P_{TH2} &= \mbox{Total thermal power [kW]} \\
	\dot{m}_{H2} &= \mbox{\gls{H2} production rate [kg/h]} \\
	\Delta H &= \mbox{Specific energy [kWh/kg-H$_2$].}
\end{align}

\section{Microreactors and \glspl{SMR}}
\label{sec:reactors}

These reactor concepts share several features.
The reactors require limited on-site preparation as their components are factory-fabricated and shipped out to the generation site.
This feature reduces up-front capital costs, enables rapid deployment, and expedites start-up times.
This reactor concept allows for black starts and islanding operation mode.
They can start up from an utterly de-energized state without receiving power from the grid.
They can also operate connected to the grid or independently.
Moreover, these types of reactors are self-regulating.
They minimize electrical parts and use passive safety systems to prevent overheating and safely shutdown.

Microreactors have the distinction that is transportable.
Small designs make it easy for vendors to ship the entire reactor by truck, shipping vessel, or railcar.
These features make the technology appealing for a wide range of applications, such as deployment in remote residential locations and military bases.

The \gls{DOE} defines a microreactor as a reactor that generates from 1 to 20 MWt \cite{us-doe_ultimate_2019}.
The \gls{IAEA} describes an \gls{SMR} as a reactor whose power is under 300 MWe.
It defines, as well, a very small modular reactor as a reactor that produces less than 15 MWe \cite{world_nuclear_association_small_2020}.
As the definitions of these reactor concepts overlap, we will consider reactors of less than 100 MWt regardless of their specific classification.

\section{Methodology}
\label{sec:metho}

In this analysis, the energy source (electric and thermal) is a nuclear reactor with co-generation capabilities.
The nuclear reactor supplies the grid with electricity $P_E$ while providing a hydrogen plant with electricity $P_{EH2}$ and thermal energy $P_{TH2}$, see the diagram in Figure \ref{fig:cogen}.
$\beta$ and $\gamma$ determine the distribution of the reactor thermal power $P_{th}$ into $P_E$, $P_{EH2}$, and $P_{TH2}$, see Equations \ref{eq:cogen1} to \ref{eq:cogen6}.

\begin{figure}[htbp!]
	\centering
	\includegraphics[height=5.0cm]{figures-hydro/hte-figure0.png}
	\hfill
	\caption{Diagram of a reactor coupled to a hydrogen plant.}
	\label{fig:cogen}
\end{figure}

\begin{align}
	P_{E} &= \eta \beta P_{th} 	\label{eq:cogen1} \\
	P_{EH2} &= \eta \gamma (1-\beta) P_{th} \\
	% \label{eq:cogen2}
	P_{TH2} &= (1-\gamma) (1-\beta) P_{th}
	\label{eq:cogen3}
	%\label{eq:cogen3}
	\intertext{where}
    \eta &= \mbox{thermal-to-electric conversion efficiency} \\
	% \label{eq:cogen4}
	\beta &= \frac{P_{E} / \eta}{P_{E} / \eta + P_{TH2}/(1-\gamma)} \\
	% \label{eq:cogen5}
	\gamma &= \frac{P_{EH2} / \eta}{P_{EH2} / \eta + P_{TH2}}.
	\label{eq:cogen6}
\end{align}

If $\beta = 1$, the reactor only supplies the grid with electricity $P_E$, and the hydrogen plant does not produce \gls{H2}.
If $\beta = 0$, the reactor only supplies the hydrogen plant, and no electricity goes into the grid.
Table \ref{tab:cogen1} summarizes the values that $\gamma$ takes for the different methods.

\begin{table}[htbp!]
    \centering
    \begin{tabular}{l|ccc}
        \hline
        Method    & $\gamma$         & $P_{EH2}$ & $P_{TH2}$ \\ \hline
        \gls{LTE} & 1                & $\ne$ 0   & 0         \\
        \gls{HTE} & $0 < \gamma < 1$ & $\ne$ 0   & $\ne$ 0   \\
        \gls{SI}  & 0                & 0         & $\ne$ 0   \\ \hline
    \end{tabular}
    \caption{Energy requirements of the different methods.}
    \label{tab:cogen1}
\end{table}

\section{Results}
\label{sec:Results}

This section holds the results of the different analyses.

\subsection{Transportation}

This subsection centers its focus on the transportation sector.
Figure \ref{fig:fuel} displays the fuel consumed per day by \gls{MTD} and \gls{UIUC} fleet.
Using the values shown in Table \ref{tab:equiv}, we calculate the \gls{H2} requirement for MTD and UIUC fleets, Figure \ref{fig:hydro-fleet}.
Table \ref{tab:hydro-fleet} summarizes the results.

	\begin{figure}[htbp!]
		\centering
        \subfloat[\gls{MTD} fleet. Data go from July 1, 2018, until June 30, 2019 \cite{mtd_irecords_2019}.]{
            \includegraphics[width=0.45\textwidth]{figures-hydro/mtd2}
        }
        \subfloat[\gls{UIUC} fleet. Data go from January 1, 2019, until December 31, 2019 \cite{uiuc_personnal_communication}.]{
            \includegraphics[width=0.45\textwidth]{figures-hydro/uiuc}
        }
		\hfill
		\caption{Fuel consumption data.}
		\label{fig:fuel}
	\end{figure}

	\begin{table}[htbp!]
	\centering
	\caption{\gls{H2} necessary to replace a gallon of fuel \cite{doe_office_of_energy_efficiency_and_renewable_energy_hydrogen_2020} \cite{alternative_fuels_data_center_fuel_2014}.}
	\begin{tabular}{l|c}
	    \hline
	 	                 & Hydrogen Mass [kg] \\ \hline
	 	Gasoline         & 1                  \\
	 	Diesel           & 1.13               \\
	 	E85              & 0.78               \\ \hline
	\end{tabular}
	\label{tab:equiv}
	\end{table}

	\begin{figure}[htbp!]
	    \centering
		\includegraphics[height=7.0cm]{figures-hydro/hydro-fleet}
		\hfill
		\caption{\gls{H2} requirement for MTD and UIUC fleets.}
		\label{fig:hydro-fleet}
	\end{figure}

	\begin{table}[htbp!]
		\centering
	    \caption{\gls{H2} requirement for MTD and UIUC fleets.}
		\begin{tabular}{l|c}
		\hline
		Total [tonnes/year]     & 943    \\
		Average [kg/day] 	    & 2584   \\
		Average [kg/h] 		    & 108    \\
		Maximum in one day [kg] & 4440   \\ \hline
        \end{tabular}
        \label{tab:hydro-fleet}
	\end{table}

Using Table \ref{tab:co2-eq}, we calculate the \gls{CO2} savings caused by replacing all the fossil fuels by \gls{H2}.
Table \ref{tab:co2} displays the \gls{CO2} savings for both fleets.

	\begin{table}[htbp!]
		\centering
	    \caption{\gls{CO2} savings in lbs per gallon of fuel burned \cite{energy_information_administration_how_2014}.}
		\begin{tabular}{l|c}
		\hline
		              & \gls{CO2} produced [lbs/gallon] \\ \hline
		Gasoline      & 19.64           \\
		Diesel        & 22.38           \\
		E85           & 13.76           \\ \hline
        \end{tabular}
        \label{tab:co2-eq}
	\end{table}

	\begin{table}[htbp!]
		\centering
	    \caption{\gls{CO2} yearly savings.}
		\begin{tabular}{l|c}
		\hline
		            & \gls{CO2} mass [tonnes/year] \\ \hline
		MTD      	  & 7306           \\
		UIUC        & 1143           \\
		Total       & 8449           \\ \hline
        \end{tabular}
        \label{tab:co2}
	\end{table}

We have determined the \gls{H2} requirement by the fleets, and now we seek a microreactor design capable of meeting such demand.
For our analysis, we chose a few microreactor designs summarized in Table \ref{tab:hydro-micro}.
Further studies could include other designs as well.

Figure \ref{fig:hydro-micro} shows the hourly production rates for the different reactors and the \gls{H2} production processes.
The figure includes a continuous line that represents the hydrogen requirement of both fleets.
Note that the \gls{SI} process' required high temperatures allow for the coupling with only one microreactor design, which has an outlet temperature of more than 800$^{\circ}$C.

	\begin{table}[htbp!]
		\centering
	    \caption{Microreactor designs.}
		\begin{tabular}{l|cc}
		\hline
		Reactor                                      & P[MWt] & T$_o$[$^\circ$C] \\ \hline
		MMR \cite{usnc_mmr_2019}  		             & 15           & 640              \\
		eVinci \cite{hernandez_micro_2019}           & 5            & 650              \\
		ST-OTTO \cite{harlan_x-energy_2018}          & 30           & 750              \\
		U-battery \cite{ding_design_2011}            & 10           & 750              \\
		Starcore \cite{star_core_nuclear_star_2015}  & 36           & 850              \\ \hline
        \end{tabular}
        \label{tab:hydro-micro}
	\end{table}

	\begin{figure}[htbp!]
	    \centering
		\includegraphics[height=6.0cm]{figures-hydro/reactors-by-hour1}
		\hfill
		\caption{Hydrogen production rate by the different microreactor designs.}
		\label{fig:hydro-micro}
	\end{figure}

\subsection{Electricity Generation}

This subsection centers its focus on the electricity generation sector and the duck curve problem.
To quantify the duck curve's magnitude, we have to predict the \gls{UIUC} grid's load and the expected electricity production from solar. 
As the \gls{icap}'s main objective is to become carbon neutral before 2050, we make our prediction for that year.
\gls{UIUC} solar farm is relatively new, and the data available is not enough for producing a reliable forecast.
To go around this barrier, we use the available data for the whole \gls{US}.
Figure \ref{fig:prediction} displays the prediction for 2050.
We carry out the prediction using a linear regression that produces the worst-case scenario.
In such a scenario, the total load does not increase considerably, whereas the solar generation does.

\begin{figure}[htbp!]
    \centering
    \subfloat[Total electricity generation.]{
        \includegraphics[width=0.45\textwidth]{figures-hydro/us-prediction1}
    }
    \subfloat[Solar electricity generation.]{
        \includegraphics[width=0.45\textwidth]{figures-hydro/us-prediction2}
    }
    \hfill
    \caption{Prediction of the electricity generation in the \gls{US} for 2050. Data from \cite{us_energy_information_administration_electric_2020}.}
    \label{fig:prediction}
\end{figure}

The next step was to apply the same growth factor from the predictions to the \gls{UIUC} grid's load and solar electricity.
To obtain a prediction for 2050, we apply the growth factor to the hourly data.
We choose a spring day when solar production is higher, as it is sunny, but the total load is low since people are not using electricity for air conditioning or heating \cite{us_department_of_energy_confronting_2017}.
Finally, we subtract the solar production from the total load, obtaining the net load or demand ($D_{NET}$).

We narrowed our analysis' focus to April 4th, when the net demand reached the lowest value in the 2019's spring.
Figure \ref{fig:uiuc-duck1} shows these results.
In 2050, the peak net demand will be 46.9 MWh at 5 PM.
The lowest net demand will be 15 MWh at 11 AM.
These results yield a demand ramp of 31.9 MWh in 4 hours.
These results show that the grid requires an available capacity of dispatchable sources of at least 31.9 MW.

\begin{figure}[htbp!]
		\centering
	\includegraphics[height=7cm]{figures-hydro/uiuc-duck}
	\hfill
	\caption{Prediction of \gls{UIUC}'s net demand for 2050.}
	\label{fig:uiuc-duck1}
\end{figure}

% Big part of this should go in the methodology section
Once we calculated the net demand, the next step was to figure the over-generated electricity.
For that purpose, we arbitrarily chose a reactor of 25 MW.
For the \gls{LTE} case, any reactor is a valid option.
We chose an $\eta$ of 33$\%$, which yields a reactor power of 75.8 MWt.
For the \gls{HTE} case, the reactor's choice is an HTGR with an outlet temperature of 850$^{\circ}$C.
We consider an $\eta$ of 49.8$\%$, which yields a reactor of 50.2 MWt.

The reactor operates at full capacity at all times.
However, the reactor electricity ($P_{E}$) equals the net demand ($D_{NET}$) once smaller than 25 MW.
Note that $P_{E}$ has power units while $D_{NET}$ has energy units.
We chose time steps of 1 hour for our analysis, hence $P_{E}$ and $D_{NET}$ differ by the constant $h$.
As $P_{E}$ is lower than 25 MW, and the reactor is at full thermal capacity, the hydrogen plant takes the excess of thermal energy.
We use equation \ref{eq:demand} with equations \ref{eq:cogen1} to \ref{eq:cogen6} to calculate the hydrogen produced.
Figure \ref{fig:uiuc-duck2} displays the results.
The total \gls{H2} production reaches 660, 1009, and 815 kg for \gls{LTE}, \gls{HTE}, and \gls{SI}.

\begin{align}
	P_{E} &= D_{NET}  \notag \\
  \frac{P_{E}}{25 MW} &= \frac{\eta \beta P_{th}}{\eta P_{th}} = \beta  \label{eq:demand}
\end{align}

% \begin{align}
	% P_{E} &= $D_{NET}$  \notag \\
  % \frac{P_{E}}{25 MW} &= \frac{\eta \beta P_{th}}{\eta P_{th}} = \beta  \label{eq:demand}
% \end{align}

\begin{figure}[htbp!]
		\centering
	\includegraphics[height=7cm]{figures-hydro/uiuc-hydro2B}
	\hfill
	\caption{\gls{H2} production.}
	\label{fig:uiuc-duck2}
\end{figure}

% This should definitely be in Methodology
Our analysis last step is to calculate the peak demand reduction by using the hydrogen to produce electricity.
The energy produced by hydrogen is $285 kJ/mol$, equal to 40 kWh/kg \cite{ursua_hydrogen_2012}.
However, conventional fuel cells can use up to 60$\%$ of that energy \cite{doe_energy_efficiency_and_renewable_energy_fuel_2015}.
Knowing the mass of hydrogen produced, we calculate the total electricity produced.
We now reduce the peak demand by distributing the electricity over a specific range of hours.
We chose to distribute the electricity for over 6 hours.
We calculate the new peak using equation \ref{eq:newpeak}.
Figure \ref{fig:uiuc-duck3} shows these results.
The different \gls{H2} processes can generate 15.84 MWh, 24.2 MWh, and 19.6 MWh, respectively.
This generation accounts for a peak reduction of 5 MW, 6.4 MW, and 5.6 MW, respectively.

\begin{align}
	NP &= \frac{\sum_{i=0}^{N} D_{NET, i} - TH}{N} \\
	\label{eq:newpeak}
	\intertext{where}
		NP &= \mbox{New peak magnitude} \\
		D_{NET, i} &= \mbox{Hourly net demant} \\
		TH &= \mbox{Total mass oh hydrogen} \\
		N &= \mbox{Total number of hours when we use the \gls{H2}}
\end{align}

\begin{figure}[htbp!]
    \centering
	\includegraphics[height=7cm]{figures-hydro/uiuc-hydro3}
	\hfill
	\caption{Peak reduction by using the produced H$_2$.}
	\label{fig:uiuc-duck3}
\end{figure}

\section{Conclusions}

The world faces energy challenges that compromise the efforts to stop climate change.
The electricity generation and transportation sectors are the largest issuers of \glspl{GHG} and, hence, the major contributors to climate change.
These challenges underscore the need for cleaner sources.
Nonetheless, the common belief that renewable energy is the solution to the problem presents several drawbacks.
The duck curve is an example of such drawbacks.
Moreover, a carbon-neutral electric grid will be insufficient to halt climate change.
The transportation sector needs to survey some possible alternatives to become carbon-free as well.
In this work, we proposed combining nuclear energy and hydrogen production that represents a possible solution to these challenges.

To seek a solution for the challenge described above, we narrowed down our focus on a more particular case, the University of Illinois.
Through the implementation of the \gls{icap}, the University of Illinois is actively working to reduce \gls{GHG} emissions on its campus.
This work's objective aligns with the efforts in two of the six target areas defined on the \gls{icap}, electricity generation, and transportation.

Regarding hydrogen production methods, we surveyed three different processes: \gls{LTE}, \gls{HTE}, and \gls{SI}.
We developed a tool to calculate their energy requirements, regarding electricity and heat, and hydrogen production rates.
This tool is applicable to a stand-alone hydrogen plant and a nuclear power plant that produces both electricity and hydrogen.

In the transportation sector analysis, we quantified the fuel requirements of \gls{MTD} and \gls{UIUC} fleets.
We calculated the mass of hydrogen necessary to replace 100$\%$ of the fleet's fossil fuel usage.
Finally, we chose several microreactor designs, and we calculated their hydrogen production rates.
The microreactors that can meet both fleet hydrogen needs are the MMR, ST-OTTO, U-battery, and Starcore.
Starcore design is the only one that could use the \gls{SI} process.

In the electricity generation sector analysis, we predicted the duck curves' magnitude in UIUC's grid in 2050.
This result exhibits how an increased solar penetration into the grid worsens the duck curve.
We proposed a mitigation strategy that uses a microreactor of 25 MWe.
For such a reactor, we calculated the mass of hydrogen produced by the different methods during the day.
Finally, we estimated a peak demand reduction by using the hydrogen produced during the day.
This last result highlights that hydrogen introduces a means to store energy that reduces the reliance on dispatchable sources.
This analysis emphasizes how nuclear energy and hydrogen production are an approach to mitigate climate change.


\chapter{Conclusions}
\section{Contribution}

% Ch1: prismatic htgrs
The development of the HTGR technology begun almost 60 years ago.
HTGRs have several desirable features that make them a good candidate for deployment in the near-term.
Some of those features are reliance on passive heat transfer mechanisms, use of TRISO particles, and high operating temperatures.
Higher temperatures offer increased cycle efficiencies and enable a wide range of process heat applications, such as hydrogen production.
Hydrogen can be a decisive response to energy and climate challenges, as it can decarbonize the transport and power sectors.
Several hydrogen production processes benefit from high temperatures, such as high-temperature electrolysis or thermochemical water splitting.
Utilizing high-temperature nuclear power plants as the energy source of the process eliminates the need to burn fossil fuels.

% Ch1: motivation
To support the evolution of the HTGR technology, this work focused on the extension of Moltres capabilities to prismatic HTGRs.
Modeling and prediction of core thermal-hydraulic behavior is necessary for assessing the safety characteristics of a reactor.
The fuel blocks’ complex geometry requires elaborate numerical calculations.
The characteristics of an HTGR are different from those of conventional Light Water Reactors (LWRs).
Such differences demand for new reactor analysis tools.
Historically, the operator-splitting technique allowed for coupling stand-alone neutronics solver to a thermal-hydraulics solver and modeling a reactor.
However, solving the fully-coupled system is preferable over using the operator-splitting technique.
Moltres can solve systems of equations in a fully-coupled way or solve systems of equations independently. This feature makes Moltres suitable for solving multi-physics problems and a wide range of nuclear engineering problems.

% Ch1: Objectives
This work extends Moltres modeling capabilities to prismatic HTGRs.
% left here

% Ch4: Neutronics
Multi-physics simulators need to resolve the double heterogeneities present in the prismatic HTGR fuel assemblies.
Monte Carlo simulators are capable of explicitly modeling TRISO particles.
Although using such a capability is computationally expensive, Chapter \ref{ch:neutronics} proved it necessary for obtaining group constants for diffusion solvers.
Diffusion solvers rely on different levels of homogenization.
Moltres previous work focused on MSRs which allow for a more heterogeneous homogenization.
Nonetheless, HTGRs require a higher level of homogenization making Moltres not easily applicable to HTGRs.
This work studied using Moltres as a homogeneous solver for carrying out neutronics stand-alone simulations of prismatic HTGRs.
The first study comprised an analysis of the effects of the energy group structure on the simulations of a fuel column.
Such a study compared Moltres results to Serpent results.
Based on that study, the following section conducted a full-core simulation in Moltres using a 15-energy group tructure.
We compared Moltres results to Serpent results, and overall, the results showed good agreement.
Finally, Moltres carried out Phase I Exercise 1 of the OECD/NEA MHTGR-350 MW Benchmark.
The exercise provides the group constants which required the development of a tool to adapt them to Moltres format.
The exercise models one-third of the MHTGR-350 and uses periodic boundary conditions on the sides.
Imposing those boundary conditions in Moltres was not possible because of the simulation high memory requirements.
To circumvent this barrier, Moltres conducted the exercise using reflective boundary conditions.
We also analyzed the effects of such an approximation as some discrepancies in the results arose.
We concluded that the boundary condition approximation was responsible for causing such discrepancies.

% Ch5: Thermal-fluids


% Ch6: Hydrogen
World-wide climate change demands a large scale deployment of \gls{CO2}-free energy sources.
UIUC is fighting climate change by actively reducing GHG emissions on its campus.
This work comprises efforts that align with UIUC's goals to reduce \gls{CO2} emissions from the electricity generation and the transportation sectors.
This work proposed an alternative: the deployment of a nuclear reactor and hydrogen production.
Regarding hydrogen production methods, we surveyed three different processes: LTE, HTE, and SI.
We developed a tool to calculate their energy requirements, regarding electricity and heat, and hydrogen production rates.
This tool is applicable to a stand-alone hydrogen plant and a nuclear power plant that produces both electricity and hydrogen.

\section{Future Work}

This section introduces some possible future work as a continuation of this thesis.

% Moltres extensions
- 3D Thermal-fluids
	- radiative heat transfer
	- porous media model
- Multi-physics
- Thermo-mechanics

% Thermal-fluids
This simplified algorithms allow for solving the mass flow rate distribution in the core for steady-state cases, and for transient cases as an approximation.
However, in coupled analyses, the flow distribution depends on the temperature and will change with respect to time.
This creates the necessity for developing tools integrated into Moltres.
Currently, MOOSE has a module for modeling the incompressible Navier-Stokes equations.
Integrating that module into the solver could improve the accuracy.
This task will be part of the future work.



% Ch4: Neutronics


% Ch6: Hydrogen
As \ref{ch:hydro} mentioned, high temperatures enable efficient hydrogen production.
However, the current fleet of nuclear reactors are LWRs with outlet temperatures around 300$^{\circ}$C.
Additionally, HTE and SI are viable processes for temperatures well above 300$^{\circ}$C.
In the short term, the development of hydrogen economies demand mature technologies.
Related future work will study the steam reforming method, which could use or not carbon-capture sequestration systems.
The future work will analyze the feasibility of using a carbon-capture sequestration system from an economics perspective.
For the case the process does not use a carbon-capture sequestration system, we will compare the \gls{CO2} savings versus the \gls{CO2} production.
Moreover, related future work will study the viability of combining LWRs to HTE and SI processes via steam boosting systems.
These systems rely on the use of electric heaters to enhance the steam temperature.
These steam boosting systems could enable the coupling of hydrogen plants to the LWR fleet, and it is a promising short term solution to ease climate change.


\begin{appendices}
	\chapter{Equations}
	
\section{Neutron flux equations}
\label{appendix:equations-n}


\begin{align}
  \chi_g^t &= \chi_g^p (1 - \beta) + \chi_g^d \sum_i^I \beta_i  \label{eq:chit} \\
  \nabla \cdot D_g \nabla \phi_g - \Sigma_g^r \phi_g & + \sum_{g \ne g'}^G \Sigma_{g'\rightarrow g}^s \phi_{g'} +
  \chi_g^t \sum_{g' = 1}^G \frac{1}{k_{eff}}\nu \Sigma_{g'}^f \phi_{g'} = 0 \label{eq:eigenvalue}
\end{align}

\section{Thermal-fluids equations}
\label{appendix:equations-th}



% In the steady-state limit, the continuity equation becomes:
% \begin{align}
%   \rho_c u & = \frac{\dot{m}}{A} = \mbox{constant} \label{eq:continuity2}
%   \intertext{where}
%   \dot{m} &= \mbox{mass flow rate} \notag
% \end{align}

% Introducing equation \ref{eq:continuity2} into \ref{eq:temperature} gives:
% \begin{align}
%   c_{p,c} \frac{\dot{m}}{A} \frac{\partial}{\partial z} T_c = q'''_{conv}
%   \label{eq:temperature2}
% \end{align}
	\chapter{Group Constants Handling}
	\section{Group constants homogenization}
\label{appendix:group-const-homo}

\begin{align}
  & \Phi_{g'} = \sum_g \phi_{g}
  & \Sigma_{g'}^t = \frac{\sum_g \Sigma_g^t \phi_{g}}{\Phi_{g'}}
  & \nu\Sigma_{g'}^f = \frac{\sum_g \nu\Sigma_g^f \phi_{g}}{\Phi_{g'}}
  & D_{g'}^t = \frac{\sum_g D_g^t \phi_{g}}{\Phi_{g'}}
  & \chi_{g'}^t = \frac{\sum_g D_g^t \phi_{g}}{\Phi_{g'}}
  \intertext{where}
  & \phi_g = \mbox{group $g$ neutron flux } [n \cdot cm^{-2} \cdot s^{-1}] \notag \\
  & t = \mbox{time } [s] \notag \\
  & D_g = \mbox{group $g$ diffusion coefficient } [cm] \notag \\
  & \Sigma_g^r = \mbox{group $g$ macroscopic removal cross-section } [cm^{-1}] \notag \\
  & \Sigma_{g'\rightarrow g}^s = \mbox{group $g'$ to group $g$ macroscopic scattering cross-section } [cm^{-1}] \notag \\
  & \chi_g^p = \mbox{group $g$ prompt fission spectrum } [-] \notag\\
  & G = \mbox{number of discrete energy groups } [-] \notag \\
  & \nu = \mbox{number of neutrons produced per fission } [-] \notag \\
  & \Sigma_g^f = \mbox{group $g$ macroscopic fission cross-section } [cm^{-1}] \notag \\
\end{align}





\section{Group constants condensation}
\label{appendix:group-const-condense}


\section{Benchmark group constants}

% Maybe ?
% \begin{align}
% \Sigma_{r,g} &= \Sigma_{a,g} + \sum_{g' \ne g} \Sigma_{s,g \rightarrow g'} = \Sigma_{t,g} - \Sigma_{s, g \rightarrow g}
% \label{eq:removal}
% \end{align}
	\chapter{Analytical Solutions}
	\section{Verification of the thermal-fluids model}
\label{appendix:ver}

The analytical solution of the problem is
\begin{align}
    T_c (r, z) &= T_{in} + \frac{q_{ave} R_f^2 L}{2 \rho c_p v \pi R_c^2} \left[ 1 + cos \left( \frac{\pi}{L} z \right) \right] \\
    T_3(z) &= T_c(z) + \frac{q_{ave} \pi}{2} sin \left( \frac{\pi}{L} z \right) R_f^2 \frac{ln(R_i/R_m)}{2 k_i} \\
    T_2(z) &= T_3(z) + \frac{q_{ave} \pi}{2} sin \left( \frac{\pi}{L} z \right) R_f^2 \frac{ln(R_m/R_g)}{2 k_m} \\
    T_1(z) &= T_2(z) + \frac{q_{ave} \pi}{2} sin \left( \frac{\pi}{L} z \right) R_f^2 \frac{ln(R_g/R_f)}{2 k_g} \\
    T_f (r=0, z) &= T_1(z) + \frac{q_{ave} \pi}{2} sin \left( \frac{\pi}{L} z \right) R_f^2 \frac{1}{4 k_f} \\
    T_f (r, z=L/2) &= \frac{q_{ave}}{4 k_f} \left(R_f^2 - r^2\right) + T_1 (z=L/2) \\
    T_g (r, z=L/2) &= \frac{T_1 (z=L/2)-T_2 (z=L/2)}{ln (R_f/R_g)} ln (r/R_g) + T_1(z=L/2) \\
    T_m (r, z=L/2) &= \frac{T_2 (z=L/2)-T_3 (z=L/2)}{ln (R_g/R_m)} ln (r/R_m) + T_2(z=L/2) \\
    T_i (r, z=L/2) &= \frac{T_3 (z=L/2)-T_c (z=L/2)}{ln (R_m/R_i)} ln (r/R_i) + T_3(z=L/2) \\
    T_c (r, z=L/2) &= T_c(z=L/2)
    \intertext{where}
    T_{c} &= \mbox{bulk coolant temperature } [^{\circ}C] \notag \\
    T_{in} &= \mbox{inlet coolant temperature } [^{\circ}C] \notag \\
    q_{ave} &= \mbox{average power density } [W \cdot cm^{-3}] \notag \\
    R_f &= \mbox{fuel compact radius } [cm] \notag \\
    L &= \mbox{fuel column height } [cm] \notag \\
    \rho &= \mbox{helium density } [kg \cdot cm^{-3}] \notag \\
    c_p &= \mbox{helium heat capacity } [J \cdot kg^{-1} \cdot K^{-1}] \notag \\
    v &= \mbox{average helium velocity } [cm \cdot s^{-1}] \notag \\
    R_c &= \mbox{coolant channel radius } [cm] \notag \\
    R_g &= \mbox{gap radius } [cm] \notag \\
    R_m &= \mbox{moderator radius } [cm] \notag \\
    R_i &= \mbox{film radius } [cm] \notag \\
    k_f &= \mbox{fuel compact thermal conductivity } [W \cdot cm^{-1} \cdot K^{-1}] \notag \\
    k_g &= \mbox{gap thermal conductivity } [W \cdot cm^{-1} \cdot K^{-1}] \notag \\
    k_m &= \mbox{moderator thermal conductivity } [W \cdot cm^{-1} \cdot K^{-1}] \notag \\
    k_i &= \mbox{film thermal conductivity } [W \cdot cm^{-1} \cdot K^{-1}]. \notag
\end{align}

\end{appendices}

% \chapter*{Appendix A}
% \setcounter{chapter}{8}
% \setcounter{section}{0}
% \setcounter{equation}{0}
% 
\section{Neutron flux equations}
\label{appendix:equations-n}


\begin{align}
  \chi_g^t &= \chi_g^p (1 - \beta) + \chi_g^d \sum_i^I \beta_i  \label{eq:chit} \\
  \nabla \cdot D_g \nabla \phi_g - \Sigma_g^r \phi_g & + \sum_{g \ne g'}^G \Sigma_{g'\rightarrow g}^s \phi_{g'} +
  \chi_g^t \sum_{g' = 1}^G \frac{1}{k_{eff}}\nu \Sigma_{g'}^f \phi_{g'} = 0 \label{eq:eigenvalue}
\end{align}

\section{Thermal-fluids equations}
\label{appendix:equations-th}



% In the steady-state limit, the continuity equation becomes:
% \begin{align}
%   \rho_c u & = \frac{\dot{m}}{A} = \mbox{constant} \label{eq:continuity2}
%   \intertext{where}
%   \dot{m} &= \mbox{mass flow rate} \notag
% \end{align}

% Introducing equation \ref{eq:continuity2} into \ref{eq:temperature} gives:
% \begin{align}
%   c_{p,c} \frac{\dot{m}}{A} \frac{\partial}{\partial z} T_c = q'''_{conv}
%   \label{eq:temperature2}
% \end{align}

% \chapter*{Appendix B}
% \setcounter{chapter}{9}
% \setcounter{section}{0}
% \setcounter{equation}{0}
% \section{Group constants homogenization}
\label{appendix:group-const-homo}

\begin{align}
  & \Phi_{g'} = \sum_g \phi_{g}
  & \Sigma_{g'}^t = \frac{\sum_g \Sigma_g^t \phi_{g}}{\Phi_{g'}}
  & \nu\Sigma_{g'}^f = \frac{\sum_g \nu\Sigma_g^f \phi_{g}}{\Phi_{g'}}
  & D_{g'}^t = \frac{\sum_g D_g^t \phi_{g}}{\Phi_{g'}}
  & \chi_{g'}^t = \frac{\sum_g D_g^t \phi_{g}}{\Phi_{g'}}
  \intertext{where}
  & \phi_g = \mbox{group $g$ neutron flux } [n \cdot cm^{-2} \cdot s^{-1}] \notag \\
  & t = \mbox{time } [s] \notag \\
  & D_g = \mbox{group $g$ diffusion coefficient } [cm] \notag \\
  & \Sigma_g^r = \mbox{group $g$ macroscopic removal cross-section } [cm^{-1}] \notag \\
  & \Sigma_{g'\rightarrow g}^s = \mbox{group $g'$ to group $g$ macroscopic scattering cross-section } [cm^{-1}] \notag \\
  & \chi_g^p = \mbox{group $g$ prompt fission spectrum } [-] \notag\\
  & G = \mbox{number of discrete energy groups } [-] \notag \\
  & \nu = \mbox{number of neutrons produced per fission } [-] \notag \\
  & \Sigma_g^f = \mbox{group $g$ macroscopic fission cross-section } [cm^{-1}] \notag \\
\end{align}





\section{Group constants condensation}
\label{appendix:group-const-condense}


\section{Benchmark group constants}

% Maybe ?
% \begin{align}
% \Sigma_{r,g} &= \Sigma_{a,g} + \sum_{g' \ne g} \Sigma_{s,g \rightarrow g'} = \Sigma_{t,g} - \Sigma_{s, g \rightarrow g}
% \label{eq:removal}
% \end{align}

% \chapter*{Appendix C}
% \setcounter{chapter}{10}
% \setcounter{section}{0}
% \setcounter{equation}{0}
% \section{Verification of the thermal-fluids model}
\label{appendix:ver}

The analytical solution of the problem is
\begin{align}
    T_c (r, z) &= T_{in} + \frac{q_{ave} R_f^2 L}{2 \rho c_p v \pi R_c^2} \left[ 1 + cos \left( \frac{\pi}{L} z \right) \right] \\
    T_3(z) &= T_c(z) + \frac{q_{ave} \pi}{2} sin \left( \frac{\pi}{L} z \right) R_f^2 \frac{ln(R_i/R_m)}{2 k_i} \\
    T_2(z) &= T_3(z) + \frac{q_{ave} \pi}{2} sin \left( \frac{\pi}{L} z \right) R_f^2 \frac{ln(R_m/R_g)}{2 k_m} \\
    T_1(z) &= T_2(z) + \frac{q_{ave} \pi}{2} sin \left( \frac{\pi}{L} z \right) R_f^2 \frac{ln(R_g/R_f)}{2 k_g} \\
    T_f (r=0, z) &= T_1(z) + \frac{q_{ave} \pi}{2} sin \left( \frac{\pi}{L} z \right) R_f^2 \frac{1}{4 k_f} \\
    T_f (r, z=L/2) &= \frac{q_{ave}}{4 k_f} \left(R_f^2 - r^2\right) + T_1 (z=L/2) \\
    T_g (r, z=L/2) &= \frac{T_1 (z=L/2)-T_2 (z=L/2)}{ln (R_f/R_g)} ln (r/R_g) + T_1(z=L/2) \\
    T_m (r, z=L/2) &= \frac{T_2 (z=L/2)-T_3 (z=L/2)}{ln (R_g/R_m)} ln (r/R_m) + T_2(z=L/2) \\
    T_i (r, z=L/2) &= \frac{T_3 (z=L/2)-T_c (z=L/2)}{ln (R_m/R_i)} ln (r/R_i) + T_3(z=L/2) \\
    T_c (r, z=L/2) &= T_c(z=L/2)
    \intertext{where}
    T_{c} &= \mbox{bulk coolant temperature } [^{\circ}C] \notag \\
    T_{in} &= \mbox{inlet coolant temperature } [^{\circ}C] \notag \\
    q_{ave} &= \mbox{average power density } [W \cdot cm^{-3}] \notag \\
    R_f &= \mbox{fuel compact radius } [cm] \notag \\
    L &= \mbox{fuel column height } [cm] \notag \\
    \rho &= \mbox{helium density } [kg \cdot cm^{-3}] \notag \\
    c_p &= \mbox{helium heat capacity } [J \cdot kg^{-1} \cdot K^{-1}] \notag \\
    v &= \mbox{average helium velocity } [cm \cdot s^{-1}] \notag \\
    R_c &= \mbox{coolant channel radius } [cm] \notag \\
    R_g &= \mbox{gap radius } [cm] \notag \\
    R_m &= \mbox{moderator radius } [cm] \notag \\
    R_i &= \mbox{film radius } [cm] \notag \\
    k_f &= \mbox{fuel compact thermal conductivity } [W \cdot cm^{-1} \cdot K^{-1}] \notag \\
    k_g &= \mbox{gap thermal conductivity } [W \cdot cm^{-1} \cdot K^{-1}] \notag \\
    k_m &= \mbox{moderator thermal conductivity } [W \cdot cm^{-1} \cdot K^{-1}] \notag \\
    k_i &= \mbox{film thermal conductivity } [W \cdot cm^{-1} \cdot K^{-1}]. \notag
\end{align}


\backmatter

% \bibliographystyle{apalike}
\bibliographystyle{ieeetr}
\bibliography{bibliography}

\end{document}
\endinput
%%
%% End of file
