\documentclass[edeposit,fullpage]{uiucthesis2018}

\usepackage[acronym,toc]{glossaries}
\newacronym{ANL}{ANL}{Argonne National Laboratory}
\newacronym{B4C}{B4C}{boron carbide}
\newacronym{BC}{BC}{boundary condition}
\newacronym{BOC}{BOC}{beginning of equilibrium cycle}
\newacronym{BSD}{BSD}{Berkeley Software Distribution}
\newacronym{BWR}{BWR}{Boiling Water Reactor}
\newacronym{CAISO}{CAISO}{California ISO}
\newacronym{CEA}{CEA}{Commissariat a l'Energie Atomique}
\newacronym{CFD}{CFD}{computational fluid dynamics}
\newacronym{CO2}{CO$_2$}{carbon dioxide}
\newacronym{CR}{CR}{control rod}
\newacronym{CRP}{CRP}{Coordinated Research Project}
\newacronym{CZP}{CZP}{Cold Zero Power}
\newacronym{DCC}{DCC}{depressurized conduction cool-down}
\newacronym{DOE}{DOE}{Department of Energy}
\newacronym[\glslongpluralkey={degrees of freedom}]{DoF}{DoF}{degree of freedom}
\newacronym{EOC}{EOEC}{end of equilibrium cycle}
\newacronym{FCEV}{FCEV}{Fuel Cell Electric Vehicle}
\newacronym{FDM}{FDM}{Finite Difference Method}
\newacronym{FEM}{FEM}{Finite Element Method}
\newacronym{FVM}{FVM}{Finite Volume Method}
\newacronym{FSV}{FSV}{Fort St. Vrain}
\newacronym[\glslongpluralkey={greenhouse gases}]{GHG}{GHG}{greenhouse gas}
\newacronym{GRS}{GRS}{Gesellschaft für Anlagen und Reaktorsicherheit}
\newacronym{H2}{H$_2$}{hydrogen}
\newacronym{He}{He}{helium}
\newacronym{HFP}{HFP}{Hot Full Power}
\newacronym{HPCC}{HPCC}{high pressure conduction cool-down}
\newacronym{HTE}{HTE}{High-Temperature Electrolysis}
\newacronym{HTGR}{HTGR}{High-Temperature Gas-Cooled Reactor}
\newacronym{HTR}{HTR}{High Temperature Reactor}
\newacronym{HTTR}{HTTR}{High Temperature Test Reactor}
\newacronym{HZDR}{HZDR}{Helmholtz-Zentrum Dresden-Rossendorf}
\newacronym{IAEA}{IAEA}{International Atomic Energy Agency}
\newacronym{icap}{iCAP}{Illinois Climate Action Plan}
\newacronym{INL}{INL}{Idaho National Laboratory}
\newacronym{IPyC}{IPyC}{inner pyrolytic carbon}
\newacronym{JFNK}{JFNK}{Jacobian-Free Newton-Krylov}
\newacronym{KAERI}{KAERI}{Korea Atomic Energy Research Institute}
\newacronym{Keff}{K$_{eff}$}{multiplication factor}
\newacronym{LBP}{LBP}{Lumped Burnable Poison}
\newacronym{LGPL}{LGPL}{Lesser GNU Public License}
\newacronym{LOCA}{LOCA}{loss of coolant accident}
\newacronym{LPCC}{LPCC}{low pressure conduction cool-down}
\newacronym{LTE}{LTE}{Low-Temperature Electrolysis}
\newacronym{LWR}{LWR}{Light Water Reactor}
\newacronym{MC}{MC}{Monte Carlo}
\newacronym{MHTGR}{MHTGR}{Modular High-Temperature Gas-Cooled Reactor}
\newacronym{MOC}{MOC}{middle of equilibrium cycle}
\newacronym{MOOSE}{MOOSE}{Multi-physics Object-Oriented Simulation Environment}
\newacronym{MPI}{MPI}{Message Passing Interface}
\newacronym{MSR}{MSR}{Molten Salt Reactor}
\newacronym{MTD}{MTD}{Champaign-Urbana Mass Transit District}
\newacronym{NEA}{NEA}{Nuclear Energy Agency}
\newacronym{NEM}{NEM}{Nodal Expansion Method}
\newacronym{NGNP}{NGNP}{Next Generation Nuclear Power}
\newacronym{NRC}{NRC}{Nuclear Regulatory Commission}
\newacronym{NSC}{NSC}{Nuclear Science Committee}
\newacronym{OECD}{OECD}{Organisation for Economic Co-operation and Development}
\newacronym{OPyC}{OPyC}{outer pyrolytic carbon}
\newacronym{ORNL}{ORNL}{Oak Ridge National Laboratory}
\newacronym{OS}{OS}{Operator-Splitting}
\newacronym{PBMR}{PBMR}{Pebble Bed Modular Reactor}
\newacronym{PDE}{PDE}{Partial Differential Equation}
\newacronym{PMR}{PMR}{Prismatic Modular Reactor}
\newacronym{PV}{PV}{photovoltaics}
\newacronym{RSC}{RSC}{Reserve Shutdown Control}
\newacronym{RSD}{RSD}{Relative Standard Deviation}
\newacronym{SD}{SD}{Standard Deviation}
\newacronym{SI}{SI}{Sulfur-Iodine}
\newacronym{SiC}{SiC}{silicon carbide}
\newacronym{SMR}{SMR}{Small Modular Reactor}
\newacronym{SNU}{SNU}{Seoul National University}
\newacronym{SOEC}{SOEC}{Solid Oxide Electrolysis Cells}
\newacronym{TIP}{TIP}{transverse integration procedure}
\newacronym{TRISO}{TRISO}{Tristructural Isotropic}
\newacronym{UIUC}{UIUC}{University of Illinois at Urbana-Champaign}
\newacronym{UNIST}{UNIST}{Ulsan National Institute of Science and Technology}
\newacronym{UK}{UK}{United Kingdom}
\newacronym{UMICH}{UMICH}{University of Michigan}
\newacronym{US}{US}{United States}
\newacronym{VHTR}{VHTR}{Very High Temperature Gas Cooled Reactor}
%\newacronym{<++>}{<++>}{<++>}
%\newacronym{<++>}{<++>}{<++>}


\usepackage{xspace}
\usepackage{graphics}

\usepackage{placeins}
\usepackage{booktabs} % nice rules (thick lines) for tables
\usepackage{microtype} % improves typography for PDF

\usepackage[hyphens]{url}
\usepackage{hyperref}
\usepackage{subfig}
\usepackage{hhline}
\usepackage{amsmath}
\usepackage{color}
\usepackage{multirow}
\usepackage{siunitx}
\sisetup{
    input-decimal-markers = .,input-ignore = {,},table-number-alignment = right,
    group-separator={,}, group-four-digits = true
}
\usepackage{fourier}
\usepackage{booktabs}
\newcommand\tab[1][1cm]{\hspace*{#1}}

\usepackage{threeparttable, tablefootnote}

%tikzpicture fit to page width
\usepackage{environ}
\makeatletter
\newsavebox{\measure@tikzpicture}
\NewEnviron{scaletikzpicturetowidth}[1]{%
  \def\tikz@width{#1}%
  \def\tikzscale{1}\begin{lrbox}{\measure@tikzpicture}%
  \BODY
  \end{lrbox}
  \pgfmathparse{#1/\wd\measure@tikzpicture}%
  \edef\tikzscale{\pgfmathresult}%
  \BODY
}

\usepackage{tabularx}
\newcolumntype{b}{>{\hsize=1.0\hsize}X}
\newcolumntype{q}{>{\hsize=0.5\hsize}X}
\newcolumntype{R}{>{\raggedleft\arraybackslash\hsize=0.5\hsize}X}
\newcolumntype{z}{>{\hsize=0.75\hsize}X}
\newcolumntype{s}{>{\hsize=.5\hsize}X}
\newcolumntype{m}{>{\hsize=.75\hsize}X}

\usepackage{cleveref}
\usepackage{datatool}
\usepackage[numbers]{natbib}
\usepackage{notoccite}

\usepackage{tikz}
\usetikzlibrary{positioning, arrows, decorations, shapes}

\usetikzlibrary{shapes.geometric,arrows}
\tikzstyle{process} = [rectangle, rounded corners, minimum width=2.5cm, minimum height=1cm,text centered, draw=black, fill=blue!30]

\tikzstyle{object} = [ellipse, rounded corners, minimum width=3cm, minimum height=1cm,text centered, draw=black, fill=green!30]
\tikzstyle{objectr} = [ellipse, rounded corners, minimum width=3cm, minimum height=1cm,text centered, draw=black, fill=red!30]

\tikzstyle{empty} =  [rectangle, rounded corners, minimum width=2.5cm, minimum height=0.7cm,text centered, draw=black, fill=white!30]
\tikzstyle{arrow} = [thick,->,>=stealth]

%% Added by me
\usepackage{tabularx}
\usepackage{float}
\usepackage{enumitem}

\title{Draft}
\author{Roberto E. Fairhurst Agosta}
\department{Nuclear, Plasma, Radiological Engineering}
\schools{B.S., University of Illinois - Urbana Champaign, 2017}
\msthesis
\advisor{Kathryn D. Huff}
\degreeyear{2020}
\committee{Assistant Professor Kathryn D. Huff, Advisor \\ Professor Segundo Lector}

\begin{document}
\maketitle

\frontmatter
%% Create an abstract that can also be used for the ProQuest abstract.
%% Note that ProQuest truncates their abstracts at 350 words.
\begin{abstract}

Abstract.

\end{abstract}

\chapter*{Acknowledgments}

Acks.

%% The thesis format requires the Table of Contents to come
%% before any other major sections, all of these sections after
%% the Table of Contents must be listed therein (i.e., use \chapter,
%% not \chapter*).  Common sections to have between the Table of
%% Contents and the main text are:
%%
%% List of Tables
%% List of Figures
%% List Symbols and/or Abbreviations
%% etc.

\tableofcontents
\listoftables
\listoffigures

%% Create a List of Abbreviations. The left column
%% is 1 inch wide and left-justified
%\chapter{List of Abbreviations}
%\printglossaries
%% Create a List of Symbols. The left column
%% is 0.7 inch wide and centered

\pagebreak
\mainmatter

% different chapters
\chapter{Introduction}
\section{Prismatic Gas-Cooled Reactors}

This section will talk about different aspects of prismatic HTGRs:

\begin{itemize}
	\item history
	\item features
	\item co-generation possibilities (highlight hydrogen production)
\end{itemize}

\section{Motivation}

This section will express the motivation behind this work.
\section{Objectives}

This section will summarize the objectives of this work.

\chapter{Literature Review}
\section{PMR neutronics}

This section will talk about deterministic diffusion solvers and related past studies.
\section{PMR thermal-hydraulics}

This section will talk about thermal-hydraulic calculations in prismatic HTGRs and related past studies.
\section{PMR multi-physics}

This section will talk about previous efforts on multi-physic couplings for prismatic HTGRs.

\chapter{Methodology}
\section{MOOSE}

This section will briefly talk about MOOSE features.

\section{Moltres}

This section will talk about Moltres features.

\section{Serpent}

This section will talk about Serpent features.


\section{MHTGR-350 Summary}

This section will describe the MHTGR350 main characteristics.



\chapter{Neutronics}
\section{Preliminary studies}

This section will discuss the current capabilities in Moltres and discuss its applicability to PMRs.
\section{OECD/NEA Benchmark}

This section solves the Exercise 1 of Phase I of the OECD/NEA MHTGR-350 benchmark with the current Moltres capabilities.
\section{Serpent-Moltres validation}

This section compares the results from Moltres and Serpent.
Serpent generates the homogenized group constants and also solves the heterogeneous system, which provides the reference solutions for the validation of the calculation scheme.

\chapter{Thermal-hydraulics}
\section{Preliminary studies}

This section carries out some preliminar studies using Moltres and MOOSE heat conduction module.

\section{Unit cell problem}

This section will solve the unit cell problem in the hot spot of an HTGR.

\section{Fuel assembly}

This section will calculate the heat profile of a fuel assembly of a HTGR.

\section{Full core}

This section will extend the methodology to a fullcore problem and it will inted to solve Exercise 2 of Phase I of the OECD/NEA MHTGR-350 Benchmark.

\chapter{Hydrogen Production}
\section{Introduction}

This section introduces the global warming and duck curve problem.
This introduction also provides the motivation for such study.

\section{iCAP}

Brief description of the iCAP and how its goals aligns with the objectives of this work.

\section{Objectives}

This section summarizes the objectives of this study.
Description on how the iCAP's goals align with the objectives of this work.

\section{Hydrogen production methods}

This section summarizes the most important features of the following hydrogen production methods: Electrolysis (LTE and HTE) and Sulfur-Iodine Thermo-chemical Cycle.

\section{Microreactors and \glspl{SMR}}

This section gives a brief description of Microreactors and SMRs.

\section{Methodology}

This section discusses the methodology for:
\begin{itemize}
\item calculating the hydrogen required for transportation
\item distributing the reactor power into electricity and hydrogen generation
\item predicting the duck curve
\item calculating the electricity to generate with the hydrogen
\end{itemize}




\section{Results}

This section holds the results of the different analyses.

\section{Conclusions}

This section holds the conclusions from this chapter.


\chapter{Conclusions}
\section{Contribution}

% Ch1: prismatic htgrs
The development of the HTGR technology begun almost 60 years ago.
HTGRs have several desirable features that make them a good candidate for deployment in the near-term.
Some of those features are reliance on passive heat transfer mechanisms, use of TRISO particles, and high operating temperatures.
Higher temperatures offer increased cycle efficiencies and enable a wide range of process heat applications, such as hydrogen production.
Hydrogen can be a decisive response to energy and climate challenges, as it can decarbonize the transport and power sectors.
Several hydrogen production processes benefit from high temperatures, such as high-temperature electrolysis or thermochemical water splitting.
Utilizing high-temperature nuclear power plants as the energy source of the process eliminates the need to burn fossil fuels.

% Ch1: motivation
To support the evolution of the HTGR technology, this work focused on the extension of Moltres capabilities to prismatic HTGRs.
Modeling and prediction of core thermal-hydraulic behavior is necessary for assessing the safety characteristics of a reactor.
The fuel blocks’ complex geometry requires elaborate numerical calculations.
The characteristics of an HTGR are different from those of conventional Light Water Reactors (LWRs).
Such differences demand for new reactor analysis tools.
Historically, the operator-splitting technique allowed for coupling stand-alone neutronics solver to a thermal-hydraulics solver and modeling a reactor.
However, solving the fully-coupled system is preferable over using the operator-splitting technique.
Moltres can solve systems of equations in a fully-coupled way or solve systems of equations independently. This feature makes Moltres suitable for solving multi-physics problems and a wide range of nuclear engineering problems.

% Ch1: Objectives
This work extends Moltres modeling capabilities to prismatic HTGRs.
% left here

% Ch4: Neutronics
Multi-physics simulators need to resolve the double heterogeneities present in the prismatic HTGR fuel assemblies.
Monte Carlo simulators are capable of explicitly modeling TRISO particles.
Although using such a capability is computationally expensive, Chapter \ref{ch:neutronics} proved it necessary for obtaining group constants for diffusion solvers.
Diffusion solvers rely on different levels of homogenization.
Moltres previous work focused on MSRs which allow for a more heterogeneous homogenization.
Nonetheless, HTGRs require a higher level of homogenization making Moltres not easily applicable to HTGRs.
This work studied using Moltres as a homogeneous solver for carrying out neutronics stand-alone simulations of prismatic HTGRs.
The first study comprised an analysis of the effects of the energy group structure on the simulations of a fuel column.
Such a study compared Moltres results to Serpent results.
Based on that study, the following section conducted a full-core simulation in Moltres using a 15-energy group tructure.
We compared Moltres results to Serpent results, and overall, the results showed good agreement.
Finally, Moltres carried out Phase I Exercise 1 of the OECD/NEA MHTGR-350 MW Benchmark.
The exercise provides the group constants which required the development of a tool to adapt them to Moltres format.
The exercise models one-third of the MHTGR-350 and uses periodic boundary conditions on the sides.
Imposing those boundary conditions in Moltres was not possible because of the simulation high memory requirements.
To circumvent this barrier, Moltres conducted the exercise using reflective boundary conditions.
We also analyzed the effects of such an approximation as some discrepancies in the results arose.
We concluded that the boundary condition approximation was responsible for causing such discrepancies.

% Ch5: Thermal-fluids


% Ch6: Hydrogen
World-wide climate change demands a large scale deployment of \gls{CO2}-free energy sources.
UIUC is fighting climate change by actively reducing GHG emissions on its campus.
This work comprises efforts that align with UIUC's goals to reduce \gls{CO2} emissions from the electricity generation and the transportation sectors.
This work proposed an alternative: the deployment of a nuclear reactor and hydrogen production.
Regarding hydrogen production methods, we surveyed three different processes: LTE, HTE, and SI.
We developed a tool to calculate their energy requirements, regarding electricity and heat, and hydrogen production rates.
This tool is applicable to a stand-alone hydrogen plant and a nuclear power plant that produces both electricity and hydrogen.

\section{Future Work}

This section introduces some possible future work as a continuation of this thesis.

% Moltres extensions
- 3D Thermal-fluids
	- radiative heat transfer
	- porous media model
- Multi-physics
- Thermo-mechanics

% Thermal-fluids
This simplified algorithms allow for solving the mass flow rate distribution in the core for steady-state cases, and for transient cases as an approximation.
However, in coupled analyses, the flow distribution depends on the temperature and will change with respect to time.
This creates the necessity for developing tools integrated into Moltres.
Currently, MOOSE has a module for modeling the incompressible Navier-Stokes equations.
Integrating that module into the solver could improve the accuracy.
This task will be part of the future work.



% Ch4: Neutronics


% Ch6: Hydrogen
As \ref{ch:hydro} mentioned, high temperatures enable efficient hydrogen production.
However, the current fleet of nuclear reactors are LWRs with outlet temperatures around 300$^{\circ}$C.
Additionally, HTE and SI are viable processes for temperatures well above 300$^{\circ}$C.
In the short term, the development of hydrogen economies demand mature technologies.
Related future work will study the steam reforming method, which could use or not carbon-capture sequestration systems.
The future work will analyze the feasibility of using a carbon-capture sequestration system from an economics perspective.
For the case the process does not use a carbon-capture sequestration system, we will compare the \gls{CO2} savings versus the \gls{CO2} production.
Moreover, related future work will study the viability of combining LWRs to HTE and SI processes via steam boosting systems.
These systems rely on the use of electric heaters to enhance the steam temperature.
These steam boosting systems could enable the coupling of hydrogen plants to the LWR fleet, and it is a promising short term solution to ease climate change.


\chapter*{Appendix}
\section{Verification of the thermal-fluids model}
\label{appendix:ver}

The analytical solution of this problem is
\begin{align}
    T_c (r, z) &= T_{in} + \frac{q_{ave} R_f^2 L}{2 \rho c_p v \pi R_c^2} \left[ 1 + cos \left( \frac{\pi}{L} z \right) \right] \\
    T_3(z) &= T_c(z) + \frac{q_{ave} \pi}{2} sin \left( \frac{\pi}{L} z \right) R_f^2 \frac{ln(R_i/R_m)}{2 k_i} \\
    T_2(z) &= T_3(z) + \frac{q_{ave} \pi}{2} sin \left( \frac{\pi}{L} z \right) R_f^2 \frac{ln(R_m/R_g)}{2 k_m} \\
    T_1(z) &= T_2(z) + \frac{q_{ave} \pi}{2} sin \left( \frac{\pi}{L} z \right) R_f^2 \frac{ln(R_g/R_f)}{2 k_g} \\
    T_f (r=0, z) &= T_1(z) + \frac{q_{ave} \pi}{2} sin \left( \frac{\pi}{L} z \right) R_f^2 \frac{1}{4 k_f} \\
    T_f (r, z=L/2) &= \frac{q_{ave}}{4 k_f} \left(R_f^2 - r^2\right) + T_1 (z=L/2) \\
    T_g (r, z=L/2) &= \frac{T_1 (z=L/2)-T_2 (z=L/2)}{ln (R_f/R_g)} ln (r/R_g) + T_1(z=L/2) \\
    T_m (r, z=L/2) &= \frac{T_2 (z=L/2)-T_3 (z=L/2)}{ln (R_g/R_m)} ln (r/R_m) + T_2(z=L/2) \\
    T_i (r, z=L/2) &= \frac{T_3 (z=L/2)-T_c (z=L/2)}{ln (R_m/R_i)} ln (r/R_i) + T_3(z=L/2) \\
    T_c (r, z=L/2) &= T_c(z=L/2)
    \intertext{where}
    T_{c} &= \mbox{bulk coolant temperature} \notag \\
    T_{in} &= \mbox{inlet coolant temperature} \notag \\
    q_{ave} &= \mbox{average power density} \notag \\
    R_f &= \mbox{fuel compact radius} \notag \\
    L &= \mbox{fuel column height} \notag \\
    \rho &= \mbox{helium density} \notag \\
    c_p &= \mbox{helium heat capacity} \notag \\
    v &= \mbox{average helium velocity} \notag \\
    R_c &= \mbox{coolant channel radius} \notag \\
    R_g &= \mbox{gap radius} \notag \\
    R_m &= \mbox{moderator radius} \notag \\
    R_i &= \mbox{film radius} \notag \\
    k_f &= \mbox{fuel compact thermal conductivity} \notag \\
    k_g &= \mbox{gap thermal conductivity} \notag \\
    k_m &= \mbox{moderator thermal conductivity} \notag \\
    k_i &= \mbox{film thermal conductivity} = h R_i ln(R_i/R_m) \notag
\end{align}


\backmatter

\bibliographystyle{apalike}
\bibliography{bibliography}

\end{document}
\endinput
%%
%% End of file
