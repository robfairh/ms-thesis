
The complex geometry of the hexagonal fuel assembly hinders accurate evaluations of the temperature profile  without elaborate numerical calculations.

The unit cell model is an effective approach that reduces the computational efforts.
However, such model neglects the effects of the bypass gap flow and the radial power distribution within the fuel assembly.

HTGR advantages are an inherent safety, high thermal efficiency, and a high temperature process heat potential.
Higher fuel temperature achieve higher outlet temperature.
However, undesirably high temperatures jeopardize the integrity of the TRISO particles (as well as the fission products release).
The design limit in HTGR is 1250 C. (check)

Complex geomtry of the fuel blocks hinders accurate evaluations of the fuel temperatures requiring elaborate numerical calculations.



(To do)
Simplified computational models have been widely used for the analyses and designs of prismatic reactors. 
Popular simplified models are the equivalent cylinder model and the unit cell model.
The equivalent cylinder models, based on one-dimensional assumptions, have been widely adopted in



\cite{tak_numerical_2008}

What they did: \cite{tak_numerical_2008}
* three-dimensional computational fluid dynamics (CFD) analysis carried out on a typical fuel assembly of a prismatic HTGR
* accuracy of the unit cell approach assessed against the CFD solutions
