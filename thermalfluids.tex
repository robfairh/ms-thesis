\label{ch:thermalfluids}

\section{Preliminary studies}

This section carries out some preliminary studies using Moltres and MOOSE heat conduction module to solve the prismatic HTGR thermal-fluids.

\subsection{Verification of the thermal-fluids model}

To verify our methodology, this section solved a simplified cylindrical model whose analytical solution we know.
Section \ref{appendix:ver} presents the problem's analytical solution.
Moltres/MOOSE obtained the numerical solution of the thermal-fluid equations from Section \ref{ch3:th}.

Figure \ref{fig:th-ver-mesh} displays the model geometry, which differentiates five subregions: fuel compact, helium gap, moderator, film, and coolant.
Table \ref{tab:th-ver-char} summarizes the geometry dimensions and the input parameters.
The model reference design was the GT-MHR.
The calculated moderator radius is the fuel/coolant pitch minus the fuel compact and coolant channel radii --- the minimum distance between the fuel and coolant channels in the unit cell.
We obtained the calculated coolant radius by preserving the coolant channel volume.
The model assumed a sinusoidal power profile in the $z$-direction.

\begin{figure}[htbp!]
	\centering
	\includegraphics[width=0.25\linewidth]{figures-thermal/2D-preliminar-mesh2}
	\hfill
	\caption{Scaled-down version of the model geometry.}
	\label{fig:th-ver-mesh}
\end{figure}

\begin{table}[htbp!]
\centering
      \caption{Problem characteristics.}
      \label{tab:th-ver-char}
    % \begin{tabular}{@{}l c S[table-format=2.2] c c}
    \begin{tabular}{@{}l c c c c}
    \toprule
    \multicolumn{1}{c}{Parameter} & \multicolumn{1}{c}{Symbol} & \multicolumn{1}{c@{}}{Value} & \multicolumn{1}{c@{}}{Units} & \multicolumn{1}{c}{Reference} \\
    \midrule
  Fuel compact radius       & R$_f$     & 0.6225  & cm       & \cite{in_three-dimensional_2006} \\
  Fuel channel radius       & R$_g$     & 0.6350  & cm       & \cite{in_three-dimensional_2006} \\
  Coolant channel radius    & - 		    & 0.7950  & cm       & \cite{in_three-dimensional_2006} \\
  Fuel/coolant pitch        & -			    & 1.8850  & cm       & \cite{in_three-dimensional_2006} \\
  Fuel column height	      & L 		    & 793 	  & cm 		   & \cite{in_three-dimensional_2006} \\
  Coolant mass flow rate    & $\dot{m}$ & 0.0176 & kg/s 	   & \cite{in_three-dimensional_2006} \\
  Average power density     & q$_{ave}$ & 35      & W/cm$^3$ & \cite{in_three-dimensional_2006} \\
  Coolant inlet temperature & $T_{in}$  & 400  & $^{\circ}C$ & \cite{in_three-dimensional_2006} \\
  Helium inlet pressure     & P 		    & 70      & bar 	   & \cite{in_three-dimensional_2006} \\
  Helium density		        & $\rho_c$  & 4.940 $\times 10^{-6}$ & kg/cm$^3$ & \cite{nist_thermophysical_2020} \\
  Helium heat capacity      & c$_{p,c}$	& 5188 & J/kg/K      & \cite{nist_thermophysical_2020} \\
  Fuel compact thermal conductivity & k$_f$ & 0.07    & W/cm/K & \cite{tak_numerical_2008} \\
  Gap thermal conductivity  & k$_g$ & 3 $\times 10^{-3}$ & W/cm/K & \cite{tak_numerical_2008} \\
  Moderator thermal conductivity & k$_m$ & 0.30 & W/cm/K 	   & \cite{tak_numerical_2008} \\
    \midrule
  \multicolumn{1}{c}{Calculated parameters} &  &  &  & \\  
    \midrule
  Calculated moderator radius  & R$_m$ & 1.080  	& cm    & - \\
  Coolant film radius   		   & R$_i$ & 1.090  	& cm    & - \\
  Calculated coolant radius 	 & R$_c$ & 1.349  	& cm    & - \\
  Coolant average velocity  	 & v$_c$ & 1794.33 	& cm/s  & - \\
  Film thermal conductivity  	 & k$_i$ & 1.722 $\times 10^{-3}$ & W/cm/K & - \\
  \bottomrule
  \end{tabular}
\end{table}

Figure \ref{fig:th-ver-results} shows the axial and radial temperature profiles.
Both analytical and numerical solutions exhibit good agreement.
The outlet coolant temperature is 770.2 $^{\circ}$C, whereas the average outlet coolant temperature of the reference reactor is 950 $^{\circ}$C.
Note that this is a simplified model only for verifying that the numerical solution agrees with the analytical solution.
The model also considers only one fuel channel, while in the GT-MHR unit cell, two fuel channels deposit their heat into one coolant channel.

\begin{figure}[htbp!]
	\centering
    \subfloat[Fuel centerline and bulk coolant axial temperatures.]{
        \includegraphics[width=0.45\textwidth]{figures-thermal/2D-preliminar-axial}
    }
    \subfloat[Radial temperature at z=L/2=396.5 cm.]{
        \includegraphics[width=0.45\textwidth]{figures-thermal/2D-preliminar-radial2}
    }
	\hfill
    \caption{Temperature profiles.}
	\label{fig:th-ver-results}
\end{figure}

\subsection{Unit cell problem}
\label{sec:unitcell}

This section solved the unit cell problem in the hot spot of an HTGR.
We intended to reproduce In et al. 2006 \cite{in_three-dimensional_2006} to validate the unit-cell model.
We chose this article because it solves a three-dimensional unit-cell model and gives one of the most thorough descriptions in the open literature.
Table \ref{tab:th-val-unit-char} presents the problem characteristics.
The article does not specify the solid's material properties, and we replaced them with parameters from Tak et al. 2008 \cite{tak_numerical_2008}.
Figure \ref{fig:th-val-unit-model} displays an $XY$-plane of the model geometry and the material properties that depend on the temperature.
Additionally, In et al. used a chopped cosine as the power profile.
To simplify the analysis, we used the average value of the power profile.

\begin{table}[htbp!]
\centering
      \caption{Problem characteristics.}
      \label{tab:th-val-unit-char}
    % \begin{tabular}{@{}l c S[table-format=2.2] c c}
    \begin{tabular}{@{}l c c c c}
    \toprule
    \multicolumn{1}{c}{Parameter} & \multicolumn{1}{c}{Symbol} & \multicolumn{1}{c@{}}{Value} & \multicolumn{1}{c@{}}{Units} & \multicolumn{1}{c}{Reference} \\
    \midrule
  Fuel compact radius       & R$_f$ & 0.6225    & cm   & \cite{in_three-dimensional_2006} \\
  Fuel channel radius       & R$_g$ & 0.6350    & cm   & \cite{in_three-dimensional_2006} \\
  Coolant channel radius    & R$_c$ & 0.7950    & cm   & \cite{in_three-dimensional_2006} \\
  Fuel/coolant pitch        & p     & 1.8850    & cm   & \cite{in_three-dimensional_2006} \\
  Fuel column height        & L     & 793       & cm   & \cite{in_three-dimensional_2006} \\
  % input parameter characteristics
  Coolant channel mass flow rate & $\dot{m}$ & 0.0176 & kg/s & \cite{in_three-dimensional_2006} \\
  Average power density     & q$_{ave}$ & 35    & W/cm$^3$   & \cite{in_three-dimensional_2006} \\
  Inlet coolant temperature & T$_{in}$  & 400   & $^{\circ}$C  & \cite{in_three-dimensional_2006} \\
  Helium inlet pressure & P & 70 & bar & \cite{in_three-dimensional_2006} \\
  Helium density        & $\rho$  & 4.94 $\times 10^{-6}$ & kg/cm$^3$ & \cite{nist_thermophysical_2020} \\
  Helium heat capacity  & c$_p$ & 5188 & J/kg/K & \cite{nist_thermophysical_2020} \\
    \midrule
  \multicolumn{1}{c}{Calculated parameters} &  &  &  & \\  
    \midrule
  Coolant film radius       & R$_i$ & 0.8050    & cm     & -  \\
  Coolant average velocity  & v$_c$ & 1794.33   & cm/s   & -  \\
  Film thermal conductivity & k$_i$ & 1.731 $\times 10^{-3}$ & W/cm/K & -  \\
  \bottomrule
  \end{tabular}
\end{table}

\begin{figure}[htbp!]
	\centering
    \subfloat[Model geometry.]{
        \includegraphics[width=0.40\textwidth]{figures-thermal/val-unit-mesh}
    }
    \subfloat[Material properties.]{
        \includegraphics[width=0.50\textwidth]{figures-thermal/val-unit-matprop}
    }
	\hfill
  \caption{Moltres/MOOSE input parameters.}
	\label{fig:th-val-unit-model}
\end{figure}

Figure \ref{fig:th-val-unit-temps} shows the temperature profiles.
The axial temperatures increase from the top to the bottom of the reactor.
The moderator and coolant temperatures are parallel as the model assumes a film thermal conductivity independent of the temperature.
The fuel and moderator temperature difference decreases.
As the different materials' thermal conductivity increases with temperature, the thermal resistance between the moderator and the fuel decreases.
Table \ref{tab:th-val-unit-results} summarizes the results.
Moltres/MOOSE coolant temperature is smaller by 4$^{\circ}$C.
The moderator temperature is larger by 9$^{\circ}$C.
The fuel temperature is larger by 22$^{\circ}$C.
The cause of the fuel temperature discrepancy is the power profile simplification.
As we have seen in the previous section, the fuel-to-coolant temperature difference is small in the outlet for a sinusoidal power profile. 
The opposite extreme scenario is the uniform power profile, where the fuel-to-coolant temperature difference is larger.
In et al. used a chopped cosine power profile, which is the case in between a uniform and a sinusoidal power profile.
Hence, our model yields a larger fuel-to-coolant temperature at the outlet.
Overall, our model results showed good agreement with In et al. results.

\begin{figure}[htbp!]
  \centering
    \subfloat[Maximum fuel, moderator, and bulk coolant axial temperatures.]{
        \includegraphics[width=0.45\textwidth]{figures-thermal/in-2006-5-axial}
    }
    \subfloat[Outlet plane temperature z=793 cm.]{
        \includegraphics[width=0.45\textwidth]{figures-thermal/val-unit-outlet-plane}
    }
  \hfill
  \caption{Temperature profiles.}
  \label{fig:th-val-unit-temps}
\end{figure}

\begin{table}[htbp!]
\centering
    \caption{Comparison between In et al. and Moltres/MOOSE results.}
    \label{tab:th-val-unit-results}
    \begin{tabular}{@{}l c c}
    \toprule
  Parameter                                   & In et al. & Moltres/MOOSE \\
    \midrule
  Maximum coolant temperature [$^{\circ}$C]   & 1144      & 1140    \\
  Maximum moderator temperature [$^{\circ}$C] & 1250      & 1259    \\
  Maximum fuel temperature [$^{\circ}$C]      & 1295      & 1317    \\
    \bottomrule
  \end{tabular}
\end{table}

\section{Fuel column}
\label{sec:fuelcol}

This section solved an HTGR fuel column.
It aimed to reproduce some of Sato et al. 2010 \cite{sato_computational_2010} analyses to validate the fuel column model.
We chose this article because it gives a thorough description of the problem, making it easier to reproduce.
First, we analyzed a column with no bypass-gap.
Second, we studied a column with a 3mm-gap.

The study used the GT-MHR as the reference reactor for the calculations.
Figure \ref{fig:th-val-assem-model-a} exhibits the model geometry.
The model included only a one-twelfth portion of the column due to symmetry.
The GT-MHR shares the geometry dimensions with the MHTGR, which Table \ref{tab:element-characteristics} specifies.

Equation \ref{eq:matcoeff} evaluates the solid material properties \cite{johnson_cfd_2009}.
Table \ref{tab:th-val-assem-mat} displays the fuel compact and moderator thermal conductivity coefficients.
Table \ref{tab:th-val-assem-char} lists the helium properties and several input parameters.
Figure \ref{fig:th-val-assem-model-b} shows the temperature-dependent material properties.

\begin{align}
  \phi(T) = A_1 + A_2 T + A_3 T^2 + A_4 T^3 + A_5 T^4  \label{eq:matcoeff}
\end{align}

The model assigns a number to each coolant channel and the bypass-gap, see Figure \ref{fig:th-val-assem-model-a}.
In this exercise, we use the mass flow distribution from Sato et al.
Table \ref{tab:th-val-assem-massflow} shows the mass flow rate in each channel.

\begin{table}[htbp!]
\centering
      \caption{Problem characteristics.}
      \label{tab:th-val-assem-char}
    % \begin{tabular}{@{}l c S[table-format=2.2] c c}
    \begin{tabular}{@{}l c c c c}
    \toprule
    \multicolumn{1}{c}{Parameter} & \multicolumn{1}{c}{Symbol} & \multicolumn{1}{c@{}}{Value} & \multicolumn{1}{c@{}}{Units} & \multicolumn{1}{c}{Reference} \\
    \midrule
  Inlet coolant temperature & T$_{in}$  & 490   & $^{\circ}$C   & \cite{sato_computational_2010} \\
  Helium inlet pressure     & P         & 70    & bar           & \cite{sato_computational_2010} \\
  Helium density            & $\rho$    & 4.37 $\times 10^{-6}$ & kg/cm$^3$ & \cite{nist_thermophysical_2020} \\
  Helium heat capacity      & c$_p$     & 5188  & J/kg/K        & \cite{nist_thermophysical_2020} \\
  Average power density     & q$_{ave}$ & 27.88 & W/cm$^3$      & \cite{sato_computational_2010} \\
    \midrule
  \multicolumn{1}{c}{Calculated parameters} &  &  &  & \\  
    \midrule
  Coolant film radius       & R$_i$ & 0.804    & cm     & -  \\
  Film thermal conductivity & k$_i$ & 2.09 $\times 10^{-3}$ & W/cm/K & -  \\
  \bottomrule
  \end{tabular}
\end{table}

\begin{table}[htbp!]
\centering
  \caption{Thermal conductivity coefficients.}
  \label{tab:th-val-assem-mat} 
  \begin{tabular}{c|ccc|c}
\toprule
                          & \multicolumn{3}{c|}{Moderator}         & Fuel compact \\ \hline
Temperature range {[}K{]} & 255.6-816 & 816-1644.4 & 1644.4-1922.2 & 255.6-2200   \\
\midrule
A1                        & 28.6      & 1.24E+2    & 41.5          & 3.94         \\
A2                        & -         & -3.32E-1   & -             & 3.59E-3      \\
A3                        & -         & 4.09E-4    & -             & -1.98E-9     \\
A4                        & -         & -2.11E-7   & -             & 3.19E-12     \\
A5                        & -         & 4.02E-11   & -             & -9.77E-16    \\
\bottomrule
  \end{tabular}
\end{table}

% maybe see this to center the figures
% https://tex.stackexchange.com/questions/121824/horizontal-centering-with-subfloat

\begin{figure}[htbp!]
  \centering
    \subfloat[Model geometry. \label{fig:th-val-assem-model-a}]{
        \includegraphics[width=0.25\textwidth]{figures-thermal/val-assem-mesh}
    }
    \subfloat[Material properties. \label{fig:th-val-assem-model-b}]{
        \includegraphics[width=0.45\textwidth]{figures-thermal/val-assem-matprop}
    }
  \hfill
  \label{fig:th-val-assem-model}
\end{figure}

% \begin{table}[htbp!]
% \centering
%   \caption{Mass flow rate [g/s]. Values form \cite{sato_computational_2010}.}
%   \label{tab:th-val-assem-massflow}
%   \begin{tabular}{l|llllllllllllll}
% \toprule
% Channel & 1 & 2 & 3 & 4 & 5 & 6 & 7 & 8 & 9 & 10 & 11 & 12 & 13 & Gap \\
% \midrule
% No gap  & 6.18 & 11.34 & 11.37 & 11.38 & 11.43 & 11.33 & 22.70 & 22.73 & 22.73 & 11.38 & 22.77 & 22.91 & 11.44 & -     \\
% 3mm gap & 5.88 & 10.80 & 10.85 & 10.91 & 11.08 & 10.80 & 21.58 & 21.67 & 21.83 & 10.88 & 21.81 & 22.20 & 11.10 & 16.56 \\
% \bottomrule
% \end{tabular}
% \end{table}

\begin{table}[htbp!]
\centering
  \caption{Mass flow rate [g/s]. Values form \cite{sato_computational_2010}.}
  \label{tab:th-val-assem-massflow}
  \begin{tabular}{l|ccccccc}
\toprule
Channel & 1 & 2 & 3 & 4 & 5 & 6 & 7 \\
\midrule
No gap  & 6.18 & 11.34 & 11.37 & 11.38 & 11.43 & 11.33 & 22.70 \\
3mm gap & 5.88 & 10.80 & 10.85 & 10.91 & 11.08 & 10.80 & 21.58 \\
\midrule
Channel & 8 & 9 & 10 & 11 & 12 & 13 & Gap \\
\midrule
No gap  & 22.73 & 22.73 & 11.38 & 22.77 & 22.91 & 11.44 & -     \\
3mm gap & 21.67 & 21.83 & 10.88 & 21.81 & 22.20 & 11.10 & 16.56 \\
\bottomrule
\end{tabular}
\end{table}

Table \ref{tab:th-val-assem-results} compares Moltres/MOOSE results to Sato's results.
In the no gap case, Moltres/MOOSE's maximum coolant temperature is 2 $^{\circ}$C lower, and the maximum fuel temperature is 4$^{\circ}$C higher.
In the 3mm-gap case, Moltres/MOOSE's maximum coolant temperature is 2 $^{\circ}$C lower, and the maximum fuel temperature is 1$^{\circ}$C lower.
The results showed good agreement.

Figure \ref{fig:th-val-assem-temps} displays the outlet temperature along lines A-B and A-C from Figure \ref{fig:th-val-assem-model-a}.
The temperature is higher closer to the column center.
The bypass-flow causes the temperature in the center to rise while reducing the peripheral temperature.
The presence of the gap produces a larger temperature gradient in the assembly.

\begin{table}[htbp!]
  \centering
  \caption{Maximum temperatures.}
  \label{tab:th-val-assem-results}
\begin{tabular}{l|c|c|c|c}
\toprule
        & \multicolumn{2}{c|}{No gap} & \multicolumn{2}{c}{3mm gap} \\ \cline{2-5}
        & Sato et al. & Moltres/MOOSE & Sato et al. & Moltres/MOOSE \\ \midrule
Bulk coolant & 985     & 983               & 1007     & 1005             \\  
Fuel    & 1090    & 1094              & 1115     & 1114             \\
\bottomrule
\end{tabular}
\end{table}

\begin{figure}[htbp!]
  \centering
  % \hspace*{\fill}
    \subfloat[Line A-B.]{
        \includegraphics[width=0.45\textwidth]{figures-thermal/val-assem-line-AB}
    }
    \subfloat[Line A-C.]{
        \includegraphics[width=0.45\textwidth]{figures-thermal/val-assem-line-AC}
    }
  \hfill
  \caption{Outlet plane temperature along the line A-B and line A-C.}
  \label{fig:th-val-assem-temps}
\end{figure}

% \begin{figure}[htbp!]
%   \centering
%     \subfloat[No gap.]{
%         \includegraphics[width=0.45\textwidth]{figures-thermal/val-assem-input}
%     }
%     \subfloat[3mm gap.]{
%         \includegraphics[width=0.45\textwidth]{figures-thermal/val-assem-input}
%     }
%   \hfill
%   \caption{Outlet plane temperature profile.}
%   \label{fig:th-val-assem-temps}
% \end{figure}

\subsection{Flow distribution analysis}
\label{sec:flowdistrib}

As described in Section \ref{sec:litrev-thermalf}, several authors calculate the flow distribution using various methods.
This section compares the different method results.
The comparison chosen metrics are the maximum coolant and fuel temperatures and the mass flow distribution
The model assigns a number to each coolant channel and the bypass-gap, see Figure \ref{fig:th-val-assem-model-a}.
Note that channel 1 is a small coolant channel while channels 2 to 13 are large coolant channels.

We analyzed the following cases:
\begin{itemize}
    \item Case 1: uses the mass flow distribution from Sato et al.
    \item Case 2: uses a flat velocity profile (every channel and gap have the same velocity).
    \item Case 3: uses the incompressible flow model with a temperature-independent helium viscosity.
    \item Case 4: uses the incompressible flow model with a temperature-dependent helium viscosity.
    \item Case 5: uses the low-Mach number model with a temperature dependent helium viscosity.
\end{itemize}

To calculate the different flow distributions, we used the following equations.
Equation \ref{eq:case2mdot} calculated the flow distribution in case 1.
Equation \ref{eq:case3mdot} \cite{melese_thermal_1984} calculated the pressure drop in cases 3 and 4.
Case 4 differs from case 3 as the friction factor $f$ depends on the average channel temperature.
Equation \ref{eq:case5mdot} \cite{melese_thermal_1984} calculated the pressure drop in case 5.
For cases 3 to 5, the pressure drop is proportional to the channel mass flow, equation \ref{eq:itersolver1}.
Equations \ref{eq:case3mdot} and \ref{eq:case5mdot} calculated $B_i$.
Equations \ref{eq:itersolver2} and \ref{eq:itersolver3} \cite{melese_thermal_1984} solved the mass flow distribution iteratively.
As cases 4 and 5 mass flow distributions depend on the temperature, the calculations required another level of iterations to obtain the final mass flow distribution.
The convergence criteria were 1$^{\circ}$C for the maximum coolant and fuel temperatures.

\begin{align}
  \dot{m}_i &= \frac{A_i}{\sum_j A_j} \label{eq:case2mdot} \\
  \Delta P &= \frac{1}{\rho} \left( \frac{\dot{m}_i}{A_i} \right)^2 f \frac{2 L}{D_h} \label{eq:case3mdot} \\
  \Delta P &= \frac{\dot{m}_i^2}{2 \rho A_i^2} \left[ \frac{4 f L (T_i+T_o)}{2 D T_i} + \frac{T_o-T_i}{T_i} \right]  \label{eq:case5mdot} \\
  \Delta P &= B_i \dot{m}_i^2 \label{eq:itersolver1} \\
  \Delta P &= \left( \frac{\dot{m}_T}{\sum_i \frac{1}{\sqrt{B_i}}} \right)^2 \label{eq:itersolver2} \\
  \dot{m}_i &= \sqrt{\Delta P / B_i} \label{eq:itersolver3}
  \intertext{where}
  \dot{m}_i &= \mbox{channel $i$ mass flow rate} \notag \\
  A_i &= \mbox{channel $i$ cross-sectional area} \notag \\
  \Delta P &= \mbox{pressure drop} \notag \\
  T_i &= \mbox{channel inlet coolant temperature} \notag \\
  T_o &= \mbox{channel outlet coolant temperature} \notag \\
  \dot{m}_T &= \mbox{total mass flow rate} \notag \\
\end{align}

Table \ref{tab:th-assem-flow-massflow} displays the results for the mass flow rates.
% Case 4 and case 5 converged after two and three iterations, respectively.
Case 2 yields the largest small coolant channel mass flow and the smallest large coolant channel mass flow.
Case 3 and 4 barely differ.
Not considering the viscosity's temperature dependency yields a more straightforward method and the accuracy loss is negligible.
Case 5 arrives at the closest values to the reference solution.

Table \ref{tab:th-assem-flow-results} summarizes the maximum temperatures.
Case 2 yields the largest difference for the maximum coolant temperature, which is less than 10$^{\circ}$C.
Case 5 yields the closest results to the reference values.
For the maximum fuel temperature, case 2 and 5 yield the best results.
Again, case 3 and 4 barely differ.

The low-Mach number model yielded the closest results to the reference solution.
However, such a method required a two-level iterative solver.
From a computational point of view, case 2 model is the simplest method as it does not require an iterative solver.
Additionally, case 2 model yields the simplest Moltres input file.
For these reasons, the rest of the thesis uses case 2 model for the fluid flow distribution.

\begin{table}[htbp!]
  \centering
  \caption{Mass flow rates [g/s].}
  \label{tab:th-assem-flow-massflow}
  \begin{tabular}{l|ccccccc}
\toprule
Channel & 1 & 2 & 3 & 4 & 5 & 6 & 7 \\
\midrule
Case 1  & 5.88 & 10.80 & 10.85 & 10.91 & 11.08 & 10.80 & 21.58 \\
Case 2  & 6.66 & 10.41 & 10.41 & 10.41 & 10.41 & 10.41 & 20.82 \\
Case 3  & 5.98 & 10.91 & 10.91 & 10.91 & 10.91 & 10.91 & 21.82 \\
Case 4  & 5.97 & 10.90 & 10.90 & 10.91 & 10.92 & 10.90 & 21.80 \\
Case 5  & 5.83 & 10.75 & 10.81 & 10.90 & 11.09 & 10.73 & 21.58 \\
\midrule
Channel & 8 & 9 & 10 & 11 & 12 & 13 & Half-gap \\
\midrule
Case 1  & 21.67 & 21.83 & 10.88 & 21.81 & 22.20 & 11.10 & 8.28 \\
Case 2  & 20.82 & 20.82 & 10.41 & 20.82 & 20.82 & 10.41 & 8.20 \\
Case 3  & 21.82 & 21.82 & 10.91 & 21.82 & 21.82 & 10.91 & 8.55 \\
Case 4  & 21.81 & 21.83 & 10.91 & 21.82 & 21.85 & 10.92 & 8.55 \\
Case 5  & 21.71 & 21.92 & 10.84 & 21.87 & 22.26 & 11.08 & 8.63 \\
\bottomrule
\end{tabular}
\end{table}

\begin{table}[htbp!]
  \centering
  \caption{Comparison of the maximum temperatures that the different cases yield.}
  \label{tab:th-assem-flow-results}
\begin{tabular}{l|ccccc}
\toprule
        & Case 1 & Case 2 & Case 3 & Case 4 & Case 5 \\
\midrule
Coolant & 1005   &  994   &  999 & 1000 & 1007 \\
Fuel    & 1114   & 1116   & 1109 & 1110 & 1116 \\
\bottomrule
\end{tabular}
\end{table}

\subsection{Mesh convergence analysis}
\label{sec:meshconverge}
% I might not include this in the final document

The remainder of this chapter intends to solve the full-core problem.
This section aims to identify some possible
problems introduced by the full-core problem's large mesh size requirement.
This section conducts a mesh convergence analysis of the full-fuel column problem.
Figure \ref{fig:th-full-assem-mesh} displays the model geometry.
Table \ref{tab:th-full-assem-results} presents the results.
The convergence criteria were 1$^{\circ}$C for the maximum coolant and fuel temperatures.
The coolant temperature converged for the fifth mesh.
The fuel temperature did not reach convergence.
Further refinement of the mesh was not possible as the simulation memory requirements were too high.

This analysis reveals a potential problem.
The high level of detail in our geometry requires a large number of elements in the mesh.
In the full-scale problem the dimensions increase, and the number of elements in the mesh do so too.
Potentially, this method will not be able to solve the three-dimensional full-scale problem due to a high memory requirement.
For this reason, the next section intends to solve the full-scale problem using a two-dimensional cylindrical model.

% Maybe talk about that the number of nodes in the coolant channels could be reduced as we are using the fluid 1D equations.
% However, the coolant channel mesh requires enough points in the circle to properly model the heat transfer between the solid and the coolant.
% The ideal mesh would have many points in the boundary of the cylinder and one point in the cylinder center.
% But that is not numerically good, as the jacobian of those triangles is too as mall and could cause issues on the simulations.

\begin{figure}[htbp!]
  \centering
  \includegraphics[width=0.45\linewidth]{figures-thermal/full-assem-mesh2}
  \hfill
  \caption{Full fuel column model geometry.}
  \label{fig:th-full-assem-mesh}
\end{figure}

\begin{table}[htbp!]
  \centering
  \caption{Maximum temperatures.}
  \label{tab:th-full-assem-results}
\begin{tabular}{l|ccccc}
\toprule
                            & Mesh 1 & Mesh 2 & Mesh 3 & Mesh 4 & Mesh 5 \\
\midrule
Number of elements [$\times 10^{6}$]  & 1.025 & 1.306 & 1.833 & 2.596 & 3.006 \\
Number of DoFs [$\times 10^{6}$]      & 0.524 & 0.666 & 0.932 & 1.317 & 1.525 \\
Maximum coolant temperature & 1060.41 & 1062.23 & 1064.00 & 1065.13 & 1065.32 \\
Maximum fuel temperature    & 1204.49 & 1217.32 & 1225.57 & 1233.44 & 1234.93 \\
\bottomrule
\end{tabular}
\end{table}

% \section{Full-core 3D}
% This section will extend the methodology to a full-core problem and it will intend to solve Exercise 2 of Phase I of the OECD/NEA MHTGR-350 Benchmark.

\section{OECD/NEA MHTGR-350 MW Benchmark: Phase I Exercise 2}
\label{sec:ph1ex2}

This section conducted Phase I Exercise 2 of the benchmark with Moltres/MOOSE.
Phase I Exercise 2 defines a thermal-fluid standalone calculation.
The exercise purpose is to ensure that the thermal-fluid model differences between participants are negligible and will not affect the coupled exercises.
The benchmark specifies the power density of each fuel region and defines the mass flow distribution and material properties for four sub-cases:
\begin{itemize}
  \item Exercise 2a: No bypass flow and fixed thermo-physical properties. The model does not account for the bypass flow, and the thermo-physical properties are constant.
  \item Exercise 2b: Bypass flow type I and fixed thermo-physical properties. This exercise prescribes the bypass flow distribution, and the thermo-physical properties are constant.
  \item Exercise 2c: Bypass flow type I and variable thermo-physical properties. This exercise prescribes the bypass flow distribution and the thermo-physical properties depend on different simulation parameters.
  \item Exercise 2d: Bypass flow type II and variable thermo-physical properties. This exercise solves the bypass flow distribution through the explicit modeling of the bypass gaps. The thermo-physical properties depend on different simulation parameters.
\end{itemize}

The exercise requires reporting the average and maximum values of the reflector, \gls{RPV}, fuel, moderator, and coolant temperatures.
It also requires reporting the RPV heat flux and the mass flow rate distribution in the coolant channels and the bypass gaps.
These data are helpful to trace the source of possible differences in the primary parameters between participants.
OECD/NEA did not publish this exercise's results.
This section compares Moltres/MOOSE results against INL's benchmark results \cite{strydom_inl_2013}.
It presents only a subset of available data that illustrates the main characteristics of the exercise.

Figure \ref{fig:ex2a-1st-model} displays the model geometry.
To simplify the simulation of the exercise, the model uses several simplifications.
In the axial direction, it does not consider the upper plenum and the outlet plenum.
The axial boundaries are the top reflector's upper face and the bottom reflector's lower face.
The top reflector's upper face and the inlet coolant flow are at 259 $^{\circ}$C.
The bottom reflector's lower face is adiabatic.
The outlet coolant uses an outflow boundary condition.
In the radial direction, the model does not consider the core barrel and the helium gap.
After the outer reflector is the RPV, and after it, the outside air
The outer boundary of the air region is at 30 $^{\circ}$C.

The core model geometry uses INL's model as the reference design.
Nine rings define the solid structures in the core, and the other nine rings the coolant flow in the core.
We calculated the radii of the rings by preserving the assemblies' volume.
The coolant thickness is constant for all the rings and preserves the coolant volume in the core.
% missing description of the power density distribution
Table \ref{tab:th-ex2a} presents the material properties.
All the graphite regions assume the grade H-451 graphite material properties.

Table \ref{tab:th-ex2a-1st-results} displays the results.
Our model considerably under-predicts the inner reflector rings temperature, while it over-predicts the coolant rings temperature.
With the purpose of better understanding this behavior, Figure \ref{fig:ex2a-1st-model-across} shows the temperature across the reactor on the bottom of the active core (z=200 cm).
This figure exposes that the temperatures in the active core are well above the other region temperatures.
Although such a behavior is normal; the temperature profile reveals some heat transfer disconnection between the different rings.
In other words, the heat transfer from the fuel rings to the rest of the core structures is smaller than the heat transfer to the coolant rings in the active core region.
This thermal disconnection causes the reflector temperatures to be too low and the coolant temperatures too high.
These results indicate that Moltres model fails to capture some heat transfer mechanisms that INL's model captures.
The most notorious difference between the models is the inclusion of the radiative heat transfer.
Future works will focus on confirming this supposition and adding the capability to model the radiative heat transfer.

% model H figure
\begin{figure}[htbp!]
  \centering
  \includegraphics[width=0.45\linewidth]{figures-thermal/ex2a-meshD2}
  \hfill
  \caption{Model geometry.}
  \label{fig:ex2a-1st-model}
\end{figure}

\begin{table}[htbp!]
\centering
      \caption{Problem characteristics.}
      \label{tab:th-ex2a}
      % \begin{tabular}{@{}l S[table-format=3.5] c c}
      \begin{tabular}{@{}l c c c}
    \toprule
  \multicolumn{1}{c}{Parameter} & \multicolumn{1}{c@{}}{Value} & \multicolumn{1}{c@{}}{Units} & \multicolumn{1}{c}{Reference} \\    
    \midrule
  Fuel compact thermal conductivity & 20    & W/m/K   & \cite{oecd_nea_benchmark_2017} \\
  Fuel block thermal conductivity   & 37    & W/m/K   & \cite{oecd_nea_benchmark_2017} \\
  Graphite thermal conductivity     & 66    & W/m/K   & \cite{oecd_nea_benchmark_2017} \\
  \gls{RPV} thermal conductivity    & 40    & W/m/K   & \cite{oecd_nea_benchmark_2017} \\
  Coolant thermal conductivity      & 0.41  & W/m/K   & \cite{oecd_nea_benchmark_2017} \\
  Air thermal conductivity          & 0.068 & W/m/K   & \cite{oecd_nea_benchmark_2017} \\
  Helium density                    & 5.703 & $\times 10^{-6}$ kg/cm$^3$  & \cite{nist_thermophysical_2020} \\
  Helium heat capacity              & 5188  & J/kg/K  & \cite{nist_thermophysical_2020} \\
  \bottomrule
  \end{tabular}
\end{table}

\begin{table}[htbp!]
\centering
      \caption{First bottom core level average temperatures.}
      \label{tab:th-ex2a-1st-results}
\begin{tabular}{l|ccc|ccc}
    \toprule
              & \multicolumn{3}{c|}{Reflector} & \multicolumn{3}{c}{Coolant} \\ \cline{2-7} 
              & Ring 1   & Ring 2   & Ring 3   & Ring 4   & Ring 5  & Ring 6  \\
    \midrule
INL           & 790      & 794      & 802      & 797      & 636     & 673     \\
Moltres/MOOSE & 268      & 313      & 769      & 1424     & 1597    & 1157    \\
    \bottomrule
  \end{tabular}
\end{table}

\begin{figure}[htbp!]
  \centering
  \includegraphics[width=0.35\linewidth]{figures-thermal/ex2a-across}
  \hfill
  \caption{Model geometry.}
  \label{fig:ex2a-1st-model-across}
\end{figure}

% left here on grammarly review
To circumvent this barrier, we developed a second model based on Stainsby's approach \cite{stainsby_investigation_2008}.
The calculation used a global model to obtain the coolant temperature in the core and then a sub-channel model to get the fuel and moderator average temperatures.
Figure \ref{fig:ex2a-2nd-model} displays the global model geometry.
The sub-channel model used half of the unit cell from Section \ref{sec:unitcell}.
Three rings defined the solid structures in the global model, and another three, the coolant flow in the core.
We calculated the radii of the rings by preserving the assemblies' volume.
The middle ring defines the fuel rings, which encompasses three rings.
The coolant ring volumes preserve the coolant volume in each of the fuel rings.
The model uses the material properties from Table \ref{tab:th-ex2a}.

Figure \ref{fig:ex2a-temps} presents the coolant and solid temperatures in the different fuel rings.
The temperature is constant in the bottom and top reflector regions.
The coolant temperature increases from the top to the bottom of the active core.
The highest coolant, fuel, and moderator temperatures are in fuel ring 1.

Table \ref{tab:th-ex2a-results} compares Moltres/MOOSE results to INL's.
We focus on the coolant temperature of the different rings.
Moltres predicts smaller values in fuel ring 1, while it predicts higher values in fuel rings 2 and 3.
For some reason, the global model fails to correctly distribute into the coolant rings the heat produced in the fuel rings.
A possible cause of this discrepancy is the flat velocity approximation.
Section \ref{sec:flowdistrib} showed that a flat velocity distribution of the coolant is reasonable in the fuel column model.
However, one of the assumptions of that model was that the power density is uniform.
In this exercise, the power density is not uniform, and that could be causing the discrepancies.
Further studies should study the cause of them.

Additionally, Moltres predicts a smaller coolant-to-moderator temperature difference for fuel rings 1 and 2.
The most considerable discrepancy is in fuel ring 1.
INL's model coolant-to-moderator temperature difference is 46$^{\circ}$C while Moltres/MOOSE predicts a difference of 30$^{\circ}$C.
The flat velocity approximation might be again the source of these discrepancies.
Finally, the moderator-to-fuel temperature differences from INL and Moltres/MOOSE are close.
In fuel ring 1, it is 20$^{\circ}$C and 22$^{\circ}$C for INL and Moltres/MOOSE results, respectively.
In fuel ring 2, the difference is 16$^{\circ}$C and 17$^{\circ}$C for INL and Moltres/MOOSE results.

% model G figure
\begin{figure}[htbp!]
  \centering
  \includegraphics[width=0.4\linewidth]{figures-thermal/ex2a-meshC2}
  \hfill
  \caption{Model geometry.}
  \label{fig:ex2a-2nd-model}
\end{figure}

% model G results
\begin{figure}[htbp!]
  \centering
    \subfloat[Coolant.]{
        \includegraphics[width=0.45\textwidth]{figures-thermal/ex2a-fullcore-cool}
    }
    \subfloat[Solid.]{
        \includegraphics[width=0.45\textwidth]{figures-thermal/ex2a-solid}
    }
  \hfill
  \caption{Axial temperatures.}
  \label{fig:ex2a-temps}
\end{figure}

\begin{table}[htbp!]
\centering
      \caption{First bottom core level average temperatures.}
      \label{tab:th-ex2a-results}
\begin{tabular}{l|c|c|c|c|c|c}
    \toprule
          & \multicolumn{2}{c|}{Fuel ring 1} & \multicolumn{2}{c|}{Fuel ring 2} & \multicolumn{2}{c}{Fuel ring 3} \\ \cline{2-7} 
          & INL     & Moltres/MOOSE     & INL     & Moltres/MOOSE     & INL       & Moltres/MOOSE \\
    \midrule
Coolant   & 797     & 718               & 636     & 646               & 673       & 696           \\
Moderator & 843     & 748               & 672     & 669               & Not shown & 721           \\
Fuel      & 863     & 770               & 688     & 686               & 722       & 739           \\
    \bottomrule
  \end{tabular}
\end{table}

\section{OECD/NEA MHTGR-350 MW Benchmark: Phase I Exercise 3}

% description of the exercise
This section sets the road-map for conducting Phase I Exercise 3 of the benchamrk with Moltres/MOOSE.
Exercise 3 combines all the data from the first two exercises, and the participants need to determine a coupled neutronic and thermal-fluids solution.
The exercise requires the reporting of the same parameters reported in Exercises 1 and 2 combined.
The benchmark specifies the group constants required to conduct the exercise.
The group constants depend on four state parameters: moderator and fuel temperature and xenon-135 and hydrogen concentration.
In addition to this data, the benchmark provides fluence maps to determine the thermal conductivity of graphite.

Two sub-cases compose Exercise 3:
\begin{itemize}
  \item Exercise 3a: Same thermal-fluids problem definition from Exercise 2c.
  \item Exercise 3b: Same thermal-fluids problem definition from Exercise 2d.
\end{itemize}


% Heterogeneous modeling dilema
Before continuing with the resolution of this exercise, note a simulation dilemma.

% Usign moltres now
At the moment, Moltres is not capable of solving prismatic HTGRs.
Moltres is a heterogeneous solver.
In a heterogeneous solver, each node of the mesh holds the information of each variable.
As we have seen earlier, the neutronics calculation require the homogenization of the group constants.
Because of this, a node in the mesh holds the information of the homogenized neutronics characteristics.
This characteristics depend on the fuel and moderator assembly.
Then, a node in the mesh requires both temperatures.
This means that in a diffusion simulation, the homogenization should occur for all the variables.
The thermal-fluids solver should obtain the homogenized (average) moderator and the fuel temperature in the assembly.
A heterogeneous strategy solves the thermal-fluids in an HTGR, however, each node in the mesh holds the information of the temperature in that node only.
A node in the fuel holds the information of the homogenized group constants but it holds only the temperature of the fuel.
In order for the simulation to be accurate, the heterogeneous temperature solver should average the temperature in the assembly and provide that information to the diffusion solver.

% solid temperature de-homogenization
A possible solution then, would be to homogenize the materials in the core obtaining an homogeneous temperature.
Such implementation would be more straightforward than using a heterogeneous solver and then averaging the temperatures.
An available approach would be using the porous media model.
In such model, the same mesh holds the information of the fluid and solid temperatures simultaneously.
However, the use of such model is not always correct.
And sometimes the model does not differentiate between the different solid regions.
The mesh should hold the information of both moderator and fuel temperatures.
A coupling model is unable to correctly calculate the thermal feedback if it does not take into account the difference between the moderator and the fuel temperatures \cite{damian_vhtr_2008}.
A de-homogenization calculation is necessary.

% problem about using ex2a model
In exercise 2a, we use a global model to obtain the coolant temperature and a unit cell model to obtain the moderator and fuel temperatures.
In a proper coupling approach, Moltres would use the moderator and fuel temperatures to determine the thermal feedback.


% initial coupling:
Describe what I've done





\section{Conclusions}

The preliminary studies focused on the verification of the thermal-fluids model and the validation of the unit-cell model.
The verification of the thermal-fluids model studied a simplified cylindrical model comparing the Moltres/MOOSE numerical solution to the problem analytical solution.
Both solutions were in agreement.
This study set the basis for the unit cell problem set-up.
% The unit cell problem use thermal-fluids model from the previous verification.
To validate the unit cell model, Section \ref{sec:unitcell} compared Moltres/MOOSE results to In et al. \cite{in_three-dimensional_2006} results.
The results presented some discrepancies.
The outlet coolant and moderator temperatures differ by less than 10$^{\circ}$C, while the fuel temperature differs by 22$^{\circ}$C.
We expected some variability in the results as some of our simulation parameters may vary from the original study.
In's article is missing the materials definition and Section \ref{sec:unitcell} adopted the material properties from Tak et al. \cite{tak_numerical_2008}.
Additionally, In used a chopped cosine power profile and Section \ref{sec:unitcell} model simplified the calculation using a uniform power density.

The following section's focus of the analysis was a one-twelfth section of an HTGR fuel column.
To validate the model, Section \ref{sec:fuelcol} intended to reproduce some of Sato et al. \cite{sato_computational_2010} results.
Section \ref{sec:fuelcol} conducted two studies: one with no bypass-gap, and one with a 3mm-bypass-gap.
To simplify the analysis, our model adopted the mass flow rates from Sato's article.
For both case studies, the maximum temperatures in the coolant and the fuel showed good agreement.
Section \ref{sec:fuelcol} also presented the temperature profile in two of the edges of the geometry.
Such an analysis exhibits the effects of the bypass flow on the temperature.
The presence of the gap makes the center temperature rise while it reduces the peripheral temperature.
The overall consequence is an increase in the temperature gradient inside the column.

The next analysis studied different calculation methods for the mass flow distribution in the fuel column.
Section \ref{sec:flowdistrib} adopted Sato's mass flow distribution as the reference value and calculated the mass flow distribution using four different methods.
From case 2 to case 5, the methods' complexity increases as some require iterative solvers.
Overall, case 2 proved to be the simplest method, and its application did not considerably deteriorate the results' accuracy.
For that reason, the following studies adopted the case 2 mass flow distribution method.

Keeping in mind that the ultimate objective of this work is to conduct full-scale simulations, Section \ref{sec:meshconverge} studied the feasibility of extending this methodology to larger meshes.
Section \ref{sec:meshconverge} conducted a mesh convergence analysis on the full-fuel column problem.
As the model uses the one-dimensional coolant equations, the coolant temperature converges relatively fast compared to the fuel temperature.
The fuel temperature did not reach convergence in this analysis as the mesh discretization increase imposed a high memory requirement on the simulations.
This analysis concluded that modeling the thermal-fluids with such a detailed level is computationally too expensive, and it suggests searching for other methods.
The following sections adopted a two-dimensional cylindrical model for the full-core analyses.

% left here to review
Section \ref{sec:ph1ex2} described Phase I Exercise 2 of the OECD/NEA MHTGR-350 MW Benchmark.
This exercise encompasses four sub-cases with different definitions of bypass-flow and material properties.
The simplest exercise is \textit{2a}, as it does not model the bypass flow and provides the definition of the material properties.
Section \ref{sec:ph1ex2} used Moltres/MOOSE to conduct it.
Additionally, the Moltres thermal-fluids model bases its definition on INL's model.
As OECD/NEA did not publish this exercise’s results, Section \ref{sec:ph1ex2} compared Moltres/MOOSE results to INL benchmark results \cite{strydom_inl_2013}.
The large discrepancies between Moltres/MOOSE and INL results suggest that our model does not capture some heat transfer mechanism that the INL model does.
One of the known differences between the models is the inclusion of the radiative heat transfer mechanism.
As Moltres does not model that mechanism, it could be the cause of the differences.

Section \ref{sec:ph1ex2} used a global and a unit cell model based on Stainsby's approach \cite{stainsby_investigation_2008} to circumvent this drawback.
The global model is responsible for calculating the coolant temperature while the unit cell model focuses on the moderator and fuel temperatures.
With this approach, Moltres/MOOSE results were closer to INL results.
The results showed some discrepancies, which indicates that some assumption/model simplification is not accurate enough.
A flat velocity approximation might be the cause of such discrepancies.
Further studies should analyze the origin of the discrepancies.

