A gap can be created between two layers of the particle fue to the debonding of two adjacent layers during irradiation.
Fast neutron irradiation causes induced creep and irradiation-induced dimensional change.

The irradiation-induced dimensional change produces considerable stresses on the coating layers and the irradiation-induced creep relieves the stresses.

\cite{cho_stress_2009}

It should be better to calculate the thermal stress evolutions of graphite assemblies during their life, in order to avoid local cracking of these components, and thus to prevent any core
damage or performance degradation.

Strength increases with fluence. Irradiation creep may relax the stresses during normal operation, causing residual stresses at reactor shutdown.

Irradiation shrinkage of the graphite may modify the strains and displacements in the assembly, but could also generate stresses if fluence gradients exist in the core, in the same way as the temperature acts.

\cite{lejeail_calculation_2005}
