This chapter introduces past and current developments in \gls{HTGR} technology, discusses the motivation behind the development of this thesis, and summarizes the objectives of the following chapters.

\section{The Prismatic High-Temperature Gas-Cooled Reactor}
\label{sec:pmr}

% History
The history of prismatic \glspl{HTGR} or simply \glspl{PMR} began in the 1960s with the deployment of the Dragon reactor in the \gls{UK} \cite{brey_development_2001}.
Its initial objective was to demonstrate the feasibility of \glspl{HTGR}.
The Dragon reactor experiment first operated in July 1965 and reached its full-power operation of 20 MWth in April 1966.
The reactor operated for 11 years, demonstrating many components' successful operation and providing information on fuel and material irradiation.
Simultaneously, interest in the \gls{US} led to the 40 MWe \gls{HTGR} Peach Bottom Unit 1.
This reactor achieved initial criticality in March 1966 and went into commercial operation in June 1967.
Peach Bottom Unit 1 demonstrated the \gls{HTGR} concept by confirming the core physics calculations, verifying the design analysis methods, and providing a database for further design activities.
Most importantly, the plant demonstrated that \glspl{HTGR} can load follow \cite{brey_development_2001}.
After the deployment of these two demonstration reactors came the first \gls{HTGR} prototype plant - the Fort St. Vrain Generating Station, shown in Figure \ref{fig:fsv-a}.
Its electric power generation started in December 1976, reaching full-power operation in November 1981.
The Fort St. Vrain plant generated 842 MWth to achieve a net output of 330 MWe.
This reactor laid the foundation for future prismatic designs.
Beginning with Fort St. Vrain, prismatic HTGRs in the \gls{US} adopted as their fuel large hexagonal-shaped graphite elements with ceramic coated \gls{TRISO} particles embedded within rods \cite{brey_development_2001}, displayed in Figure \ref{fig:fsv-b}.

\begin{figure}[htbp!]
    \centering
    \subfloat[Reactor layout. Image reproduced from \cite{barre_gas-cooled_2010}. \label{fig:fsv-a}]{
        \includegraphics[width=0.36\textwidth]{figures/fsv-core}
    }
    \subfloat[Fuel assembly. Image reproduced from \cite{melese_thermal_1984}. \label{fig:fsv-b}]{
        \includegraphics[width=0.39\textwidth]{figures/fsv-assembly}
    }
    \hfill
    \caption{Fort St. Vrain Generating Station.}
    \label{fig:fsv}
\end{figure}

% Safety characteristics of HTGRs: TRISO fuel
The HTGR's most fundamental characteristic is its unique safety features.
Radionuclide containment does not rely on active systems or operator actions.
\gls{TRISO} particles, pictured in Figure \ref{fig:triso}, play a significant role in this task.
They consist of various layers acting as containment to limit radioactive product release.
A \gls{TRISO} particle is a microsphere of about 0.8 mm in diameter.
It includes a fuel kernel surrounded by a porous carbon layer (or buffer), followed successively by an \gls{IPyC} layer, a \gls{SiC} layer, and an \gls{OPyC} layer.
The buffer layer limits kernel migration and provides some retention of gas compounds \cite{oecd_nea_benchmark_2017}.
The \gls{IPyC} layer protects the kernel from chloride in the event of \gls{SiC} decomposition and contributes to fission gas retention \cite{demkowickz_paul_triso_2019}.
The \gls{SiC} layer ensures the particle's structural integrity under constant pressure and helps retain non-gaseous fission products.
The \gls{OPyC} layer contributes to fission gas retention and protects the \gls{SiC} layer during handling.
As an additional advantage, \gls{TRISO} particles increase the proliferation resistance of \glspl{HTGR}.
TRISO particles provide significant barriers and technical difficulties to retrieve materials from within the fuel coatings \cite{paviet-hartmann_analysis_2011}.
The particles can sustain high burnup, which causes the used nuclear fuel to have a low volume fraction of plutonium, making the fuel unsuitable for use in weapons \cite{paviet-hartmann_analysis_2011}.

\begin{figure}[htbp!]
	\centering
	\includegraphics[height=3.0cm]{figures/triso}
	\caption{Drawing of a TRISO fuel particle. Image reproduced from \cite{hales_multidimensional_2013}.}
	\label{fig:triso}
\end{figure}

% Safety characteristics of HTGRs: Graphite and negative temp coeff
Graphite is another contributor to the passive safety of the \gls{HTGR} design.
Combining ceramic fuel and a graphite core structure permits high operating temperatures \cite{ballinger_balance_2004}.
Graphite has a high heat capacity and maintains its strength at temperatures beyond 2760 $^{\circ}$C.
Moreover, HTGRs have an inherent negative temperature coefficient of reactivity.
As a result, temperature changes in the core occur slowly and without damage to the core structure during transients.

% Co-generation applications: Rankine vs Bryton
HTGRs higher operating temperatures offer increased thermal conversion efficiencies.
The early \gls{HTGR} designs converted their heat into electricity using the steam-Rankine cycle \cite{herranz_power_2009}.
In such a system, the helium coolant passes through a steam generator, and the steam drives a turbine.
This arrangement is around 38\% efficient \cite{breeze_nuclear_2014}.
However, the steam cycle requires a steam generator and also a gas circulator \cite{no_review_2007}.
This requirement increases the plant's capital cost, and it creates a risk of a water ingress event.
The Brayton cycle is a better option because the helium coolant can directly drive a gas turbine in a closed cycle.
Figure \ref{fig:gt-mhr} exhibits an HTGR coupled to a gas turbine.
A closed-cycle eliminates the need for a steam generator and a gas circulator.
Additionally, it removes external sources of contamination of the nuclear circuit, reducing the need for on-line cleanup systems \cite{iaea_current_2001}.
With the Brayton cycle, the system can achieve an energy conversion efficiency of around 48\% \cite{breeze_nuclear_2014}.

\begin{figure}[htbp!]
	\centering
	\includegraphics[height=9.0cm]{figures/gt-mhr3}
	\caption{Gas turbine coupled to an HTGR. Image reproduced from \cite{baxi_evaluation_2008}.}
	\label{fig:gt-mhr}
\end{figure}

HTGRs higher outlet temperatures and increased thermal conversion efficiencies enable a wide range of process heat applications, such as coal gasification processes, oil refinery processes, and production of synthesis gas, methanol, and hydrogen.
Hydrogen offers a solution to energy and climate challenges, decarbonizing the transport and power sectors \cite{nagashima_japans_2018}.
Several hydrogen production processes benefit from high temperatures, such as high-temperature electrolysis \cite{doenitz_hydrogen_1980} or thermochemical water-splitting \cite{yildiz_efficiency_2006}.
Utilizing the \gls{HTGR} as the process's energy source eliminates the need to burn fossil fuels to generate the steam those processes require \cite{iaea_current_2001}.

% Modular
In the early 1980s, Siemens/Interatom proposed the first modular type HTGR \cite{brey_development_2001}.
The modular HTGR design adds a low core power density to the safety features of HTGRs.
A low core power density enables passive heat transfer mechanisms to remove the decay heat following postulated accidents \cite{neylan_modular_1988}.
These passive heat transfer mechanisms rely primarily on the natural processes of conduction, thermal radiation, and convection.
This characteristic provides the modular HTGR the ability to cool down entirely without exceeding the failure temperature of the TRISO particles \cite{brey_development_2001}.

% MHTGR
% This thesis focuses primarily on the \gls{MHTGR}-350 \cite{neylan_modular_1988} \cite{silady_licensing_1988}.
Under the sponsorship of the \gls{US} \gls{DOE}, a team consisting of General Atomics, Combustion Engineering, General Electric, Bechtel National, Stone \& Webster Engineering, and \gls{ORNL} developed the \gls{MHTGR}-350 \cite{neylan_modular_1988} \cite{silady_licensing_1988}.
They designed the basic module to deliver superheated steam at 17.3 MPa and 538 $^{\circ}$C.
Based on both economic and technological considerations, the optimal configuration was a 350 MWth reactor with an annular core.
The team completed in 1986 the preliminary safety information document for the MHTGR-350 and the complete draft pre-application in 1989 \cite{huning_steady_2014}.

The MHTGR concept formed the basis for the \gls{GT-MHR} reactor design.
Renewed interest in HTGRs in the US resulted in the GT-MHR development program beginning in 1993.
General Atomics (US) and MINATOM (Russia) a cooperation agreement to develop a prototype reactor.
FRAMATOME and Fuji Electric joined the program in 1996 and 1997, respectively.
The final conceptual design was a 600 MWth/293MWe plant to burn weapons plutonium and the long term goal of commercial deployment.

% HTTR
In 1987, the Japanese Atomic Energy Commission proposed the construction of the \gls{HTTR}, a prismatic block type core structure with a power rating of 30 MWth \cite{iaea_current_2001}.
The Japanese government approved proceeding with the HTTR in 1989, and construction began in March 1991.
The reactor reached first criticality in 1998 and full power in 2001.
The HTTR has two operational modes, for which the outlet coolant temperatures are 850 and 950 $^{\circ}$C.
The HTTR program evaluated six major categories of the HTGR concept: safety, thermal-fluids, fuel, high-temperature components, core physics, and control/instrumentation.
With the HTTR program, the Japanese Atomic Energy Agency established a database of operation and maintenance experience.
The HTTR program aims to share the information in the database with the design, construction, and operation of future HTGRs.

In the early 2000s, the Generation IV Roadmap project \cite{doe-ne_technology_2002} identified reactor concepts that could meet the future's energy demands in an efficient, economical, and environmentally safe manner \cite{macdonald_ngnp_2003}.
One of these reactor concepts is the \gls{VHTR}.
A \gls{VHTR} is a type of HTGR whose core outlet temperatures are between 700 and 950 $^{\circ}$C \cite{gif_gif_2019}.
The \gls{DOE} selected this reactor concept for the \gls{NGNP} Project.
This project intended to demonstrate emissions-free nuclear-assisted electricity and hydrogen production by 2015.

% Current
In addition to government-sponsored research, several privately funded gas reactor projects are underway.
StarCore Nuclear \cite{star_core_nuclear_star_2015} (Canada) is developing a 50 MWth HTGR that will produce electricity (20 MWe) and thermal energy (10 MWth) for use in process heat applications.
\gls{USNC}(US) \cite{usnc_mmr_2019} is developing the \gls{MMR}, a 15 or 30 MWth HTGR.
The MMR Energy System works as a stand-alone power plant, as part of microgrids, or to provide process heat for industrial applications \cite{world_nuclear_news_micro_2020}.
USNC has partnered with Ontario Power Generation, \gls{INL}, and the \gls{UIUC}.
Through these partnerships, USNC proposes to deploy its MMR at sites in Ontario, Idaho, and Illinois.

\section{Motivation}

% Why do we focus on HTGRs?
This work's ultimate goal is to support the development of the HTGR concept.
As the last section described, HTGRs have several favorable characteristics that make them the right candidate for large scale deployment in the near term.
For example, some microreactor designs embody this type of reactor technology and may be operational before 2030.

%Why is computational modeling important?
More specifically, this work develops computational methods for modeling prismatic \glspl{HTGR} with \textit{Moltres} \cite{lindsay_introduction_2018}.
Modeling and prediction of core thermal-fluid behavior are necessary for assessing the safety characteristics of a reactor.
Determining the temperature inside a reactor, for both normal and transient operation, is of paramount importance as several materials' integrity depends on it.
Most importantly, undesirably high temperatures endanger the TRISO particles' integrity and, consequently, jeopardize the fission product containment \cite{tak_numerical_2008}.
% The temperatures in the core have to be kept below values that begin to cause damage to fission product barriers, produce stmctural material weakness, and lead to excessive chemical reaction rates.
Furthermore, the complex fuel blocks geometry requires numerical calculations for obtaining the fuel temperatures.

The characteristics of an \gls{HTGR} are different from those of conventional \glspl{LWR}.
Such differences demand reactor analysis tools that capture the following peculiarities of \glspl{HTGR} \cite{rohde_development_2012}\cite{bostelmann_criticality_2016}:
\begin{itemize}
% \item Hexagonal structure: the shape of the fuel blocks hinders conformity to any orthogonal coordinate system.
\item Double heterogeneity: the TRISO particles form the first heterogeneity level, consisting of four
layers.
The second level arises from the fuel elements, as they encompass the compacts, the coolant, and the moderator.
\item Strong temperature dependence: the fuel temperature has a significant effect on the neutron spectrum and the transient feedbacks.
\item High-temperature gradients: the temperature difference between the fuel and the moderator is large during transients.
Large temperature gradients translate into large thermal stresses in the reactor structures.
\item Large time-scale variation: the low heat capacity of the coolant causes short transients while the large heat capacity of the graphite structures causes long ones.
\end{itemize}

%Why use Moltres?
Historically, linking a stand-alone neutronics solver to a thermal-fluid solver allowed for simulating an entire reactor.
The programs' connection occurred in a loose-coupling fashion, such that one code's output served as the other's input and vice versa.
This coupling technique is commonly known as the operator-splitting technique \cite{ragusa_consistent_2009}.
In such an approach, each program uses a physical model that solves some of the problem variables while assuming constant the rest of them.
Nonetheless, these physical models describe processes that rely heavily on the solution of one another's.
The neutron flux determines the power distribution, and the power distribution strongly influences the temperature field.
Due to the temperature feedback, the temperature affects the neutron flux distribution in the core.
Because of a strong temperature coefficient of reactivity present in HTGRs, multi-physics transient simulations coupled via the operator-splitting approach may introduce significant numerical errors \cite{park_tightly_2010}\cite{ragusa_consistent_2009}.

\gls{MOOSE} \cite{gaston_moose_2009} is a computational framework targeted at solving fully coupled systems.
All the software built on the \gls{MOOSE} framework shares an \gls{API}.
These features facilitate relatively easy coupling between separate phenomena and allow for great flexibility, even with a large variance in time scales \cite{novak_pronghorn_2018}.
Additionally, all programs use \gls{MPI} for parallel communication and allow for deployment on massively-parallel cluster-computing platforms.

\textit{Moltres} is an open-source, \gls{FEM} application built within the \gls{MOOSE} framework.
\textit{Moltres} solves arbitrary-group neutron diffusion, delayed neutron precursor concentration, and temperature governing equations.
These characteristics, plus some modifications that this work intends to implement, make \textit{Moltres} suitable for solving the type of physical phenomena described above.

\section{Objectives}

% This thesis focuses on steady-state calculations and also intends to set a roadmap for the transient simulations.
As mentioned earlier, the ultimate goal of this work is to support the development of \gls{HTGR} technology.
The following list of main objectives expands on that goal.

\paragraph{Determine prismatic \glspl{HTGR} essential physics.}
Prismatic HTGRs have inherent physics phenomena unique to their design.
This thesis intends to determine which are those inherent physics crucial for their accurate modeling.

\paragraph{Extend Moltres modeling capabilities to prismatic \glspl{HTGR}.}
Moltres is a multi-physics solver designed to capture critical physics in \glspl{MSR}.
Moltres' current capabilities allow for solving some of the physics in the prismatic HTGR design.
Nevertheless, the solver needs to capture the missing physics in prismatic HTGRs and adequately integrate them into the current capabilities.

\paragraph{Understand HTGRs' contribution to stopping climate change.}
HTGRs are an attractive technology due to attaining high temperatures and high thermal conversion efficiencies. These features yield an high hydrogen production efficiency, which can potentially ease climate change.

\vskip 0.6cm
The main objectives are somewhat broad.
The following list presents secondary objectives which will lead to the fulfillment of the main objectives:

\paragraph{Demonstrate Moltres' ability to predict HTGR neutronics.}
A neutronics solver should predict the flux shape and magnitude accurately during steady-state and transient simulations.
Previous work demonstrated Moltres' ability to solve MSR neutronics.
Chapter \ref{ch:neutronics} illustrates such a capability for prismatic HTGRs.

\paragraph{Understand the impact of the energy group structure on the HTGR diffusion calculations.}
The underlying physics of \glspl{HTGR} differs from the physics of other reactors.
Consequently, the simulation results will be sensitive to different parameters from other reactor type simulations.
Chapter \ref{ch:neutronics} studies the impact of the energy group structure of the group constants on the diffusion calculations.

\paragraph{Calculate power distribution correctly.}
The power distribution is the most influential parameter over the thermal-fluids as it determines the temperature profile in the reactor.
Previous work demonstrated Moltres' ability to calculate the power distribution in MSRs.
Chapter \ref{ch:neutronics} evinces such an ability for prismatic HTGRs.

\paragraph{Predict the prismatic HTGRs temperature profile accurately.}
Undesirably high temperatures endanger the integrity of the reactor structures, and most importantly, the TRISO particles.
Additionally, the temperature influences the neutronics.
Hence, a neutronics calculation will be inaccurate without a correct thermal-fluids calculation. 
Previous work demonstrated Moltres' ability to predict the temperature distribution in MSRs.
Chapter \ref{ch:thermalfluids} examines Moltres' ability to predict the temperature distribution in prismatic HTGRs accurately.

\paragraph{Conduct coupled simulations of prismatic HTGRs.}
An accurate simulation of prismatic HTGRs requires integrating the neutronics and the thermal-fluids through the modeling of the thermal feedback.
Previous work demonstrated Moltres' ability to calculate thermal feedback in MSRs.
Chapter \ref{ch:thermalfluids} examines Moltres' capabilities for calculating the thermal feedback in prismatic HTGRs.

\paragraph{Technical analysis of hydrogen production with HTGRs.}
High temperatures enable high-efficiency hydrogen production methods.
Most of them have different energy requirements and production rates.
Chapter \ref{ch:hydro} analyses such quantities.
